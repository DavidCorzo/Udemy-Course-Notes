\section{Creating and using arrays}
\begin{itemize}
    \item Arrays permit storing many values associated to a single variable. 
    \item Allows us to group values together, under the same name. No longer do you need a separate variable for a new value. 
    \item A limitation is that an array has to be declared with the size from the beginning, and all the elements need to be of the same type. 
    \item Declaring an array, data inside an array are called elements. 
    \item Syntax is: \mintinline{c}{long numbers[10];} an array named numbers with 10 elements. 
    \item To access an array you must write data to one or more of the indexes. Indexes start from 0. 
    \item You can access the array by typing the index number or an equivalent expression that results in a number that corresponds to an index in the array. 
    \item It's common to use a loop to access an array. 
    \item Trying to access an index outside a range of the array you'll get a \verb|array out of bounds error|. An \verb|array out of bounds error| can crash your program, can also not crash your program but make it run with bugs. 
    \item The compiler cannot check for out of bounds errors so your program will still compile. This type of error is a run time error. 
    \item Assigning values to an array, syntax is: \mintinline{c}{grades[100] = 95;} 
\end{itemize}
\subsubsection{Example}
\inputcode{\lang}{\code/index_out_of_bounds.c}

%----------------------------------------------------------------------------------------
\section{Creating and using arrays}
\begin{itemize}
    \item Initialization, you can intialize the array in the same line, using curly braces. 
    \item Example: \mintinline{c}{int counters[5] = {0,0,0,0,0};}
    \item It's not necessary to initialize the entire array. 
    \item Example: \mintinline{c}{float sample_data[500] = {100.0,300.0,500.5};} this will initialize only the first three elements, the remaining elements are set to 0. 
    \item You can also initial only certain indices, such as: \mintinline{c}{float sample_data[500] = {[2]=500.5,[1]=300.0,[0] = 100.0};} not all compilers allow this but some do. 
    \item The array size needs to be a positive number. 
\end{itemize}
\subsection{Example}
\inputcode{\lang}{\code/initialization_arrays.c}


%----------------------------------------------------------------------------------------
\section{Multidimensional arrays}
\begin{itemize}
    \item C allows arrays of any dimension to be defined. 
    \item Two-dimensional array can have a use case for spreadsheet and matrices. 
    \item Syntax is:
        \begin{minted}[autogobble]{c}
            int numbers[3][4] = { 
                {10,20,30,40},
                {15,25,35,45},
                {47,48,49,50}
            }
        \end{minted}
    
    \item The inner brackets are optional, do it for readability. 
    \item You can also initialize only a limited number of elements in you array. In the example we initialize the first three elements out of 5 in the nested array. In this case the inner pairs of braces are required. 
        \begin{minted}[autogobble]{c}
            int matrix[4][5] = { 
                {10,5,-3},
                {9,0,0},
                {32,20,1},
                {0,0,8}
            }; 
        \end{minted}
    
    \item You can also use designated initializers.
        \mint{c}{int matrix[4][3] = {[0][0] = 1, [1][1] = 5, [2][2] = 9};}
    
    \item You can declare a three-dimensional array like this: \mintinline{c}{int box[10][20][30];}
    \item You can visualize dimensions like this: 
        \begin{itemize}
            \item One-dimension: row of data. 
            \item Two-dimensional: table of data, matrix, spreadsheet. 
            \item Three-dimensional: stack of data tables. 
        \end{itemize}
    
    \item The more dimensions you have, the more nested for loops you need to iterate through them.
    \item Example of initialization of a three-dimensional array: 
        \begin{minted}[autogobble]{c}
            int numbers[2][3][4] = {
                {
                    {10,20,30,40},
                    {15,25,35,45},
                    {47,48,49,50}
                },
                {
                    {10,20,30,40},
                    {15,25,35,45},
                    {47,48,49,50}
                }
            }; 
        \end{minted}
    
    \item Iteration in an n dimensional array: The number of nested loops corresponds directly with the dimensions of arrays. 
        \begin{minted}[autogobble]{c}
            int sum = 0;
            for (int i = 0, i = 0; i < 2; ++i) {
                for (int j = 0; j < 3; ++j) {
                    for (int k = 0; k < 4, ++k) {
                        sum += numbers[i][j][k];
                    }
                }
            }
        \end{minted}
    
    \item In C there is no limit to the number of nested loops you can have. 
\end{itemize}


%----------------------------------------------------------------------------------------
\section{Variable length arrays}
\begin{itemize}
    \item Arrays are of a fixed length after they are declared, this is a problem. 
    \item Variable length is introduced in C99, variable length just means that instead of hard-coding the array size, you can have a variable that does that for you, the array is still going to have a fixed size but it won't be hard coded, this was useful when they were transcribing FORTRAN algorithms to C. 
    \item Examples:  
        \begin{minted}[autogobble]{c}
            int n = 5, m = 8; 
            float a9[m][n]; // this wasn't allowed before C99. 
        \end{minted}
    
    \item It was introduced for compatibility with FORTRAN libraries. 
\end{itemize}


%----------------------------------------------------------------------------------------
\section{Challenge: generate prime numbers from 0 to 100}
\inputcode{\lang}{\code/challenge_primes.c}
\inputcode{\lang}{\code/challenge_rainfall.c}


%----------------------------------------------------------------------------------------

