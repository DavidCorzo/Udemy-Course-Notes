\section{Standard header files}
\begin{itemize}
    \item The C standard library is basically all the functionality of the C programming language.
\end{itemize}

\subsection{strdef.h}
\begin{center}
    \begin{tabular}{ |p{6cm}|p{10cm}| }
        \hline
            Define & Meaning \\
        \hline
            \mintinline{c}{NULL}& A null pointer constant. \\ 
            \mintinline{c}{offsetof} (structure member) & The offset in bytes of the member \emph{member} from the start of the structure \emph{structure}; the typeof the result is \emph{size\_t} \\ 
            \mintinline{c}{ptrdiff_t} & The type of integer produced by subtracting two pointers. \\ 
            \mintinline{c}{size_t}& The type of integer produced by the \emph{sizeof} operator. \\ 
            \mintinline{c}{wchar_t}& The type if the integer required to hold a wide character. \\ 
        \hline
    \end{tabular}
    Taken from Programming in C, Kochan
\end{center}

\subsection{limits.h}
\begin{itemize}
    \item \verb|<limits.h>|: contains various implementation-defined limits for character and integer data types.
\end{itemize}
\begin{center}
    \begin{tabular}{ |p{6cm}|p{10cm}| }
        \hline
            Define & Meaning \\
        \hline
            \mintinline{c}{CHAR_BIT} & Number of bits in a \mintinline{c}{char} (8). \\ 
            \mintinline{c}{CHAR_MAX} & Maximum value for object of type \mintinline{c}{char} (127 if sign extension is done n chars, 255 otherwise). \\ 
            \mintinline{c}{CHAR_MIN} & Minimum value for object of type \mintinline{c}{char} (-127 if sign extension is done on chars, 0 otherwise). \\ 
            \mintinline{c}{SCHAR_MAX} & Maximum value for object of type \mintinline{c}{signed char} (127). \\ 
            \mintinline{c}{SCHAR_MIN} & Minimum value for object of type \mintinline{c}{signed char} (-127). \\ 
            \mintinline{c}{UCHAR_MAX} & Maximum value for object of type \mintinline{c}{unsigned char} (255). \\ 
            \mintinline{c}{SHRT_MAX} & Maximum value for object of type \mintinline{c}{short int} (32767). \\ 
            \mintinline{c}{SHRT_MIN} & Minimum value for object of type \mintinline{c}{short int} (-32767). \\ 
            \mintinline{c}{USHRT_MAX} & Maximum value for object of type \mintinline{c}{unsigned short int} (65535). \\ 
            \mintinline{c}{INT_MAX} & Maximum value for object of type \mintinline{c}{int} (32767). \\ 
            \mintinline{c}{INT_MIN} & Minimum value for object of type \mintinline{c}{int} (-32767). \\ 
            \mintinline{c}{UINT_MAX} & Maximum value for object of type unsined int (65535). \\ 
            \mintinline{c}{LONG_MAX} & Maximum value for object of type \mintinline{c}{long int} (2,147,483,647). \\ 
            \mintinline{c}{LONG_MIN} & Minimum value for object of type \mintinline{c}{long int} (-2,147,483,647). \\ 
            \mintinline{c}{ULONG_MAX} & Maximum value for object of type \mintinline{c}{unsigned long int} (4,294,967,295). \\ 
            \mintinline{c}{LLONG_MAX} & Maximum value for object of type \mintinline{c}{long long int} (9,223,372,036,854,775,807). \\ 
            \mintinline{c}{LLONG_MIN} & Minimum value for object of type \mintinline{c}{long long int} (9,223,372,036,854,775,807). \\ 
            \mintinline{c}{ULLONG_MAX} & Maximum value for object of type \mintinline{c}{unsigned long long int} (18,446,744,073,709,551,615). \\ 
        \hline
    \end{tabular}
    Taken from Programming in C, Kochan
\end{center}

\subsection{stdbool.h}
\begin{itemize}
    \item \verb|<stdbool.h>|: file contains definitions for working with boolean variables (type \_Bool).
\end{itemize}
\begin{center}
    \begin{tabular}{ |p{6cm}|p{10cm}| }
        \hline
            Define & Meaning \\
        \hline
            \mintinline{c}{bool} & Substitute name for the basic \mintinline{c}{_Bool} data type. \\ 
            \mintinline{c}{true} & Defined as \mintinline{c}{1}. \\ 
            \mintinline{c}{false} & Defined as \mintinline{c}{0}. \\ 
        \hline
    \end{tabular}
\end{center}


%----------------------------------------------------------------------------------------
\section{Various functions}
\begin{itemize}
    \item Reminder of some various functions.
\end{itemize}

\subsection{String function}
\begin{itemize}
    \item To use any of these, you need to include the header file \verb|<string.h>| .
        
    \item \mintinline{c}{char *strcat(s1,s1)}
        \begin{itemize}
            \item Concatenates the character string s2 to the end of s1, placing a null character at the end of the final string. The function returns s1.
        \end{itemize}
    
    \item \mintinline{c}{char *strchr(s,c)}
        \begin{itemize}
            \item Searches the string s for the first occurrence of the character c, If it is found, a pointer to the character is returned; otherwise, a null pointer is returned.
        \end{itemize}
    
    \item \mintinline{c}{int strcmp(s1, s1)}
        \begin{itemize}
            \item Compares strings s1 and s2 and returns a value less than zero if s1 is less than s2, equal to zero if s1 is equal to s2, and greater than zero if s1 is greater than s2.
            \item Greater means in alphabetic precedence of the ascii table, a is 97 and b is 98, if we were to send a and b to this function it will return a negative because a is less than b.
        \end{itemize}
    
    \item \mintinline{c}{char *strcpy(s1, s2)}
        \begin{itemize}
            \item Copies the string s2 to s1, returning s1.
            \item Integers and doubles don't need a function to reference them, strings do. 
        \end{itemize}
    
    \item \mintinline{c}{size_t strlen(s)}
        \begin{itemize}
            \item Returns the number of characters in s, excluding the null character \verb|\0|.
        \end{itemize}
    
    \item \mintinline{c}{char *strncat(s1, s2, n)}
        \begin{itemize}
            \item Copies s2 to the end of s1 until either the null character is reached or n characters have been copied, whichever occurs first. Returns s1.
            \item Safer for buffer overflows.
        \end{itemize}
    
    \item \mintinline{c}{int strncmp(s1, s2, n)}
        \begin{itemize}
            \item Performs the same function as strcmp, except that at most n characters from strings are compared.
            \item Safer for buffer overflows.
        \end{itemize}
    
    \item \mintinline{c}{char *strncpy(s1, s2, n)}
        \begin{itemize}
            \item Copies s2 to s1 until the null character is reached or n characters have been copied, whichever occurs first. Returns s1. 
            \item Safer for buffer overflows.
        \end{itemize}
    
    \item \mintinline{c}{char *strrchr(s, c)}
        \begin{itemize}
            \item Searches the string s for the last occurrence of the character c. If found, a pointer to the character in s is returned; otherwise, the null pointer is returned.
            \item Safer for buffer overflows.
        \end{itemize}
    
    \item \mintinline{c}{char *strstr(s1, s2)}
        \begin{itemize}
            \item Searches the string s1 for the first occurrence of the string s2. If found a pointer to the tart of where s2 is located inside s1 is returned; otherwise, if s2 is not located inside s1, the null pointer is returned. 
        \end{itemize}
    
    \item \mintinline{c}{char *strtok(s1, s2)}
        \begin{itemize}
            \item Breaks the string s1 into tokens based on delimiter characters in s2.
            \item Helps with parsing data.
        \end{itemize}
\end{itemize}

\subsection{Character functions}s
\begin{itemize}
    \item To use these character fucntions, you must include the file \verb|<ctype.h>|
    \item The character functions we used were: we used to check or test a character, there are also convention functions.
        \begin{itemize}
            \item isalnum
            \item isalpha
            \item isblank
            \item iscntrl
            \item isdigit
            \item isgraph
            \item islower
            \item isspace
            \item ispunct
            \item isupper
            \item isxdigit
        \end{itemize}
    
    \item \mintinline{c}{int tolower(c)}
        \begin{itemize}
            \item Returns the lowercase equivalent of c. If c is not an uppercase letter, c itself is returned.
        \end{itemize}
    
    \item \mintinline{c}{int toupper(c)}
        \begin{itemize}
            \item Returns the uppercase equivalent of c. If c is not a lowercase letter, c itself was returned.
        \end{itemize}
\end{itemize}

\subsubsection{I/O functions}
\begin{itemize}
    \item To use the most common I/O functions from the C library you should include the header file \verb|<stdio.h>| .
    \item Included in this file are declarations for the I/O functions and definitions for the names EOF, NULL, stdin, stdout, stderr (all constant values), and FILE.
    \item \mintinline{c}{int fclose(filePtr)}
        \begin{itemize}
            \item Closes the file identified by filePtr and returns zero if the close was successful, or returns EOF if an error occurs.
        \end{itemize}
    
    \item \mintinline{c}{int feof(filePtr)}
        \begin{itemize}
            \item Returns nonzero if the identified file has reached the end of the file and returns zero otherwise.
            \item We can use this instead of comparing to a constant EOF.
        \end{itemize}
    
    \item \mintinline{c}{int fflush(filePtr)}
        \begin{itemize}
            \item fflushes (writes) any data from internal buffers to the indicated file, returning zero on success and the value EOF if an error occurs.
        \end{itemize}
    
    \item \mintinline{c}{int fgetc(filePtr)}
        \begin{itemize}
            \item Returns the next character from the file identified by the filePtr, or the value of EOF if an end-of-file condition occurs.
            \item Remember that this function returns an int. 
        \end{itemize}
    
    \item \mintinline{c}{int fgetpos(filePtr, fpos)}
        \begin{itemize}
            \item Gets the current position for the file associated wirh filePtr, storing it into fpor\_t (defined in \verb|<stdio.h>|) variable pointed to by fpos. fgetpos returns zero on success, and returns nonzero on failure.
        \end{itemize}
    
    \item \mintinline{c}{char *fgets(buffer, i, filePtr)}
        \begin{itemize}
            \item Reads characters from the indicated file, until either $i-1$ characeters are read or a newline character is read, whichever occurs first.
        \end{itemize}
    
    \item \mintinline{c}{FILE *fopen(fileName, accessMode)}
        \begin{itemize}
            \item Opens the specified file with the indicated access mode.
        \end{itemize}
    
    \item \mintinline{c}{int fprintf(filePtr, format, ...)}
        \begin{itemize}
            \item Writes the specified arguments to the file identified by filePtr, according to the format specified by the character string format. 
        \end{itemize}
    
    \item \mintinline{c}{int fputc(c,filePtr)}
        \begin{itemize}
            \item Writes the value of c to the file identified by the filePtr, returning c if the write is successful, and the value EOF otherwise.
        \end{itemize}
    
    \item \mintinline{c}{int fputs(buffer, filePtr)}
        \begin{itemize}
            \item Writes the characters in the array pointers to by buffer to the indicated file until the terminating null character in buffer is reached.
        \end{itemize}
    
    \item \mintinline{c}{int fscanf(filePtr, format, ...)}
        \begin{itemize}
            \item Reads data items from the file identified by filePtr, according to the format specified by the character string format. 
        \end{itemize}
    
    \item \mintinline{c}{int fseek(filePtr, offset, mode)}
        \begin{itemize}
            \item Positions the indicated file to a point that is offset (a long int) bytes from the beginning of the file, from the current position in the file, or from the end of the file, depending upon the value of mode (an integer.)
            \item Allows you to move to a position. 
        \end{itemize}
    
    \item \mintinline{c}{int fsetpos(filePtr, fpos)}
        \begin{itemize}
            \item Sets the current file position for the file associated with filePtr to the value pointed to by fpos, which is of type fpos\_t (defined in \verb|<stdio.h>|). Returns zero on succes, and non-zero on failure.
        \end{itemize}
    
    \item \mintinline{c}{lont ftell (filePtr)}
        \begin{itemize}
            \item Returns the relative offset in bytes of the current position in the file identified by filePtr, or -1L on error.
        \end{itemize}
    
    \item \mintinline{c}{int printf(format, ...)}
        \begin{itemize}
            \item Writes the specified arguments to stdout, according tot he format specified by the character string. 
            \item Returns the number of characters written. 
        \end{itemize}
    
    \item \mintinline{c}{int remove(fileNam)}
        \begin{itemize}
            \item Removes the specified file. A non-zero value is returned on failure.
            \item The file pointer has to be closed. 
        \end{itemize}
    
    \item \mintinline{c}{int rename(fileName1, fileName2)}
        \begin{itemize}
            \item Renames the file fileName1 to fileName2, returning a non-zero result on failure.
        \end{itemize}
    
    \item \mintinline{c}{int scanf(format, ...)}
        \begin{itemize}
            \item Reads items from stdin according to the format specified by the string format. 
        \end{itemize}
\end{itemize}

\subsection{Conversion functions}
\begin{itemize}
    \item To use these functions that convert character strings to numbers you must include the header file \verb|<stdlib.h>|, look at the stdlib header file for more functions. 
    \item \mintinline{c}{double atof(s)}
        \begin{itemize}
            \item Converts the string pointed to by s into an int, returning the result.
        \end{itemize}
    
    \item \mintinline{c}{int atoi(s)}
        \begin{itemize}
            \item Converts the string pointed to by s into an int, returning the result. 
        \end{itemize}
    
    \item \mintinline{c}{int atol(s)}
        \begin{itemize}
            \item Converts the string pointed to by s into a long int, returning the result.
        \end{itemize}
    
    \item \mintinline{c}{int atoll(s)}
        \begin{itemize}
            \item Converts the string pointed to by s into a long long int, returning the result.
        \end{itemize}
\end{itemize}

\subsection{Dynamic memory allocation function}
\begin{itemize}
    \item To use these functions that allocate and free memory dynamically you must include the \verb|<stdlib.h>| header file.
    \item \mintinline{c}{void *calloc(n,size)}
        \begin{itemize}
            \item Allocates contiguous space for n items of data, where each item is size bytes in length. The allocated space is initially set to all zeroes. On success, a pointer to the allocated space is returned; on failure, the null pointer is returned. 
            \item This is preferred over malloc(). It initializes it as well as allocating it.
        \end{itemize}
    
    \item \mintinline{c}{void free(pointer)}
        \begin{itemize}
            \item Returns a block of memory pointed to by pointer that was previously allocated by a calloc(), malloc(), or realloc() call. 
        \end{itemize}
    
    \item \mintinline{c}{void *malloc(size)}
        \begin{itemize}
            \item Allocates contiguous space of size bytes, returning a pointer to the beginning of the allocated block if successful, and the null pointer otherwise.
        \end{itemize}
    
    \item \mintinline{c}{void *realloc(pointer, size)}
        \begin{itemize}
            \item Changes the size of a previously allocated block to size bytes, returning a pointer to the new block (which might have moved), or null pointer of an error occurs. 
        \end{itemize}
\end{itemize}


%----------------------------------------------------------------------------------------
\section{Math functions}
\begin{itemize}
    \item To use common math functions you must include the \verb|<math.h>| header file and link to the math library.
    \item A lot of different functions here are the main ones, consult the actual header file for more.
    \item \mintinline{c}{double acosh(x)}
        \begin{itemize}
            \item Returns the hyperbolic arc cosine of x, $x \geq 1$ 
        \end{itemize}
    
    \item \mintinline{c}{double asin(x)}
        \begin{itemize}
            \item Returns the arcsine of x as an angle expressed in radians in the range $[-\pi/2,\pi/2]$. x is in the range $[-1,1]$
        \end{itemize}
    
    \item \mintinline{c}{double atan(x)}
        \begin{itemize}
            \item Returns the arctangent of x as an angle expressed in radians in the range $[-\pi/2,\pi/2]$
        \end{itemize}
    
    \item \mintinline{c}{double ceil(x)}
        \begin{itemize}
            \item Returns the smallest integer value grater than or equal to x. Note that the value is returned as a double.
        \end{itemize}
    
    \item \mintinline{c}{double cos(r)}
        \begin{itemize}
            \item Returns the cosine of r.
        \end{itemize}

    \item \mintinline{c}{double floor(x)}
        \begin{itemize}
            \item Returns the largest integer value less than or equal to x. Note that the value is returned as double.
        \end{itemize}
    
    \item \mintinline{c}{double log(x)}
        \begin{itemize}
            \item Returns the natural logarithm of x, $x \geq 0$ 
        \end{itemize}
    
    \item \mintinline{c}{double nan}
        \begin{itemize}
            \item Returns a NaN, if possible, according to the content specified by the string pointed to by s.
        \end{itemize}
    
    \item \mintinline{c}{double pow(x,y)}
        \begin{itemize}
            \item Returns x,y. If x is less than zero, y must be an integer. If x is equal to zero, y must be greater than zero.
        \end{itemize}
    
    \item \mintinline{c}{double remainder(x,y)}
        \begin{itemize}
            \item Returns the remainder of x divided by y.
            \item You can use this instead of the \% operator.
        \end{itemize}
    
    \item \mintinline{c}{double round(x)}
        \begin{itemize}
            \item Returns the value of x rounded to the nearest integer in floating-point format. Halfway values are always rounded away from zero (so 0.5 always rounds to 1.0).
        \end{itemize}
    
    \item \mintinline{c}{double sin(r)}
        \begin{itemize}
            \item Returns the sine of r.
        \end{itemize}
    
    \item \mintinline{c}{double sqrt(x)}
        \begin{itemize}
            \item Returns the square root of x, $x \geq 0$.
        \end{itemize}
    
    \item \mintinline{c}{double tan(r)}
        \begin{itemize}
            \item Returns the tangent of r.
        \end{itemize}
    
    \item And so many more ... 
        \begin{itemize}
            \item Check out the complex arithmetic library.
        \end{itemize}
\end{itemize}


%----------------------------------------------------------------------------------------
\section{Utility functions}
\begin{itemize}
    \item To use these routines, include the header file \verb|<stdlib.h>|
    \item Here are some of the functions.
    \item \mintinline{c}{int abs(n)}
        \begin{itemize}
            \item Returns the absolute value of its argument n.
        \end{itemize}
    
    \item \mintinline{c}{void exit(n)}
        \begin{itemize}
            \item Terminates program execution, closing any returning the exit status specified by its int argument n.
            \item \verb|EXIT_SUCCESS| and \verb|EXIT_FAILURES|, defined in \verb|<stdio.h>|.
            \item Other related routines in the library that you might want to refer to are abort and atexit.
                \begin{itemize}
                    \item abort is usually invoked when you have a fatal error, the good thing about abort is that it will generate a core file and this is useful for debugging.
                \end{itemize}
        \end{itemize}
    
    \item \mintinline{c}{char *getenv(s)}
        \begin{itemize}
            \item Returns a pointer to the value of the environment variable pointed to by s, or a null pointer if the variable doesn't exist.
                \begin{itemize}
                    \item There is a get environment variable, an environment variable is like a global variable for the operating system.
                    \item You can set an environment variable and then use it in you program.
                \end{itemize}
            
            \item Used to get environment variables.
        \end{itemize}
    
    \item \mintinline{c}{void qsort(arr, n, size, comp_fn)}
        \begin{itemize}
            \item Sort the data array pointed to by the void pointer arr. 
            \item There are n elements in the array, each size bytes in length. n and size are of type size\_t.
            \item The fourth argument is of type ``pointer to function that returns int and that takes two void pointers as arguments.''
            \item qsort calls this function whenever it needs to compare two elements int the array, passing it pointer to the elements to compare.
        \end{itemize}
    
    \item \mintinline{c}{int rand(void)}
        \begin{itemize}
            \item Returns a random number in the range \verb|[',RAND_MAX]|, where \verb|RAND_MAX| is defined in \verb|<stdlib.h>| and has a minimum value of 32767.
        \end{itemize}
    
    \item \mintinline{c}{void srand(seed)}
        \begin{itemize}
            \item Seed the random number generator to the unsigned int value seed.
        \end{itemize}
    
    \item \mintinline{c}{int system(s)}
        \begin{itemize}
            \item Gives the command contained in the character array pointed to by s to the system for execution, returning a system-defined value.
            \item \mintinline{c}{system("mkdir /usr/tmp/data);}
        \end{itemize}
\end{itemize}

\subsection{Assert library}
\begin{itemize}
    \item The assert library, supported by the \verb|<assert.h>| header file, is a small one designed to help with debugging.
    \item It consists of a macro named assert()
        \begin{itemize}
            \item It takes as its argument an integer expression. 
            \item If the expression evaluates as false (non-zero), the assert() macro writes an error message to the standard error stream (stderr) and calls the abort() function, which terminates the program.
            \item This is a valuable function from debugging.
        \end{itemize}
        \begin{minted}[autogobble]{c}
            z = x * x - y * y; // should be greater + */
            assert(z >= 0); // asserts that z is greater or equal to 0
        \end{minted}
\end{itemize}

\subsection{Other useful header files}
\begin{itemize}
    \item time.h: defines macros and functions supporting operations with dates and times. 
    \item errno.h: defines macros for the reporting of errors. 
    \item locale.h: defines functions and macros to assist with formatting data such as monetary units for different countries.
    \item signal.h: defines facilities with conditions that arise during a program execution, including error conditions.
    \item stdarg.h: defines facilities that enable a variable number of arguments to be passed to a function. 
\end{itemize}
