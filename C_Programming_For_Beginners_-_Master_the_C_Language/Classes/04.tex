\section{Overview}
\begin{itemize}
    \item The concept of memory is directly related to variables and data types. 
    \item Each byte of a computer has been labeled with a number called an address.
    \item The address of a byte is unique.
    \item This all happens in RAM, which can be thought of as an ordered sequence of boxes with binary placeholders. 
    \item Memory thus consists of a large number of bits that are arranged in to groups of eight called bytes and each byte has a unique address. 
\end{itemize}

\section{Variables}
\begin{itemize}
    \item Programs are only useful if they can manipulate data. 
    \item Constants are the opposite of a variable.
    \item Variables are types of data may change or assigned values as the program runs.
    \item Variables are the names you give to computer memory which are used to store in a computer program, more practical than memorizing all the hexadecimal addresses. 
    \item 
\end{itemize}

\subsection{Naming rules}
Cannot be: 
\begin{itemize}
    \item All variable names must begin with a letter or an underscore.
    \item It can be afterward followed by any combination of letters (upper and lower), underscores or digits. 
\end{itemize}
Examples of invalid variable names: 
\begin{itemize}
    \item \verb|temp$value|: dollar sign isn't valid.
    \item \verb|my flag|: embedded spaces aren't permitted.
    \item \verb|3Jason|: variables can't start with a digit. 
    \item \verb|int|: is a reserved keyword. 
\end{itemize}
Remember: 
\begin{itemize}
    \item Have self documenting variable names.
\end{itemize}

\section{Data types}
\begin{itemize}
    \item A way for the computer to identify and use data types such as integers, strings, chars, etc.
    \item Different data types occupy different memory sizes. 
    \item In C everything is a primitive data type, there are no objects.
    \item Primitive data types are types that are not objects and store all sorts of data. 
\end{itemize}
\subsection{Declaring variables}
\begin{itemize}
    \item Syntax: \placeholder{data type} \placeholder{variable name}
    \item C requires all variables to be declared before use. 
\end{itemize}
\subsection{Initializing variables}
\begin{itemize}
    \item To initialize a variable means to assign a starting, or initial value.
    \item This can be done during the declaration such as \mintinline{c}|int x = 10;| or after the declaration \mintinline{c}{int x; x = 10;}.
    \item Use the \mintinline{c}{=} operator to assign values. 
\end{itemize}

\section{Examples}
\inputcode{\lang}{\code/variables.c}


%----------------------------------------------------------------------------------------
\section{Basic data types}
Data types suported: 
\begin{itemize}
    \item \mintinline{c}{int}
    \item \mintinline{c}{float}
    \item \mintinline{c}{double}
    \item \mintinline{c}{char}
    \item \mintinline{c}{_Bool}
\end{itemize}
What the type means: 
\begin{itemize}
    \item How many memory needs to be allocated for each variable. 
    \item This is system dependent or machine dependent, meaning depending on your machine, the memory allocated is determined. 
    \item An int can take 32 bits or 64 bits depending on you machine. 
\end{itemize}

\subsection{\mintinline{c}{int}}
\begin{itemize}
    \item Can contain integer values.
    \item These values can be negative. 
    \item If the int integer is preceded by a 0x (lower or upper), the value is a hexadecimal one. Ex: \mintinline{c}{int rgbcolor = 0xFFEF0D;}
    \item No embedded spaces are permitted between the digits.
    \item Values greater than 3 digits can't be separated using commas. 
\end{itemize}
\subsection{\mintinline{c}{float}}
\begin{itemize}
    \item Contains: numbers with decimal places. 
    \item Floating points can be also expressed in scientific notation. Ex: 1.7e4 is $1.7e^4$.
\end{itemize}
\subsection{\mintinline{c}{double}}
\begin{itemize}
    \item The same as a float, the difference is that the memory allocated is double a float, this means you can store higher numbers. 
    \item All floating-point constants are taken as double values by the C compiler by default.
    \item To explicitly express a float, append either an f of F to the end of the number. Ex: 12.5f
\end{itemize}
\subsection{\mintinline{c}{_Bool}}
\begin{itemize}
    \item Data type for binary values, true or false. 
    \item Yes or no. 
    \item 0 is a false value, 1 is a true value. 
\end{itemize}
\subsection{Other data types}
C can represent other data types: 
\begin{itemize}
    \item You can have more specified adjectives to describe your data type more specifically. 
    \item Three adjective keywords (called data specifiers), you can also just put short, long of unsigned: syntax is: \placeholder{adjective} \placeholder{datatype} 
        \begin{enumerate}
            \item \mintinline{c}{short}: less memory as an int. 
            \item \mintinline{c}{long}: more memory as an int. Append an l of L at the end of the number. 
            \item \mintinline{c}{unsigned}: is used for variables that have only non-negative values, this is just like having an absolute value for your data type, the accuracy of the integer value is extended. 
            \item *extra: signed: means they can be negative and positive. 
        \end{enumerate}
\end{itemize}
\subsection{Examples}
\inputcode{\lang}{\code/data_types.c}


%----------------------------------------------------------------------------------------
\section{Enums and Chars}
\subsection{\mintinline{c}{enum}}
\begin{itemize}
    \item Data type that allows a programmer to define a variable and specify the valid values that could be stored into that variable. 
    \item You create an enum so that only includes values you predefine. 
    \item Syntax: enum \placeholder{name} \{\placeholder{valid values}\}
    \item Example: \mintinline{c}{enum primaryColor { red,yellow,blue }}

    \item No other values other than the predefined values. 
    \item The compiler treats enum values as integer values.
    \item Enums can still be assigned index numbers using the \mintinline{c}|=| operator. 
\end{itemize}
\subsection{\mintinline{c}{char}}
\begin{itemize}
    \item chars represent \textbf{single} characters.
    \item chars literals use single quotes such as 'a'.
    \item You can declare an unsigned char because of the ASCII table. 
    \item You can used numerical codes and it will take it as an ASCII table value and convert it. 
\end{itemize}
\subsection{Escape character}
\begin{itemize}
    \item Characters that represents an action. 
    \item Also called escape sequences. 
    \item Ex: \verb|char x = '\n'|
\end{itemize}
Escape sequences table: 
\begin{center}
    \begin{tabular}{ |p{7cm}|p{7cm}| }
        \hline
            \textbf{Sequence} & \textbf{Meaning} \\
        \hline \hline
            \mintinline{c}{"\a"}   & Alert (ANSI C)\\ 
        \hline
            \mintinline{c}{"\b"}   & Backspace \\ 
        \hline
            \mintinline{c}{"\f"}   & Form feed \\ 
        \hline
            \mintinline{c}{"\n"}   & Newline \\ 
        \hline
            \mintinline{c}{"\r"}   & Carriage return \\ 
        \hline
            \mintinline{c}{"\t"}   & Horizontal tab \\ 
        \hline
            \mintinline{c}{"\v"}   & Vertical tap \\ 
        \hline
            \mintinline{c}{"\\"}   & Backslash (\verb|\|) \\ 
        \hline
            \mintinline{c}{"\'"}   & Single quote \\ 
        \hline
            \mintinline{c}{"\""}   & Double quote \\ 
        \hline
            \mintinline{c}{"\?"}   & Question mark (?) \\ 
        \hline
            \mintinline{c}{"\0oo"} & Octal value \\ 
        \hline
            \mintinline{c}{"\xhh"} & Hexadecimal value \\  
        \hline
    \end{tabular}
\end{center}
Taken from ``C Primer Plus'', Prata. 
\subsection{Code}
\inputcode{\lang}{\code/enums_chars.c}


%----------------------------------------------------------------------------------------
\section{Format specifiers}
\begin{itemize}
    \item These are used to display the value of variables using printf(). 
    \item Also used to convert data. 
    \item You can format data precision using a format specifier. Syntax: \verb|%.<precision>f|
\end{itemize}
\subsection{Table}
\begin{center}
    \begin{tabular}{ |p{5cm}|p{5cm}| }
        \hline
            \textbf{Type} &  \textbf{printf chars} \\
        \hline \hline
            \mintinline{c}{char}                    & \mintinline{c}{"%c"}                  \\
        \hline
            \mintinline{c}{_Bool}                   & \mintinline{c}{"%i"},  \mintinline{c}{"%u"} \\
        \hline
            \mintinline{c}{short int}               & \mintinline{c}{"%hi"}, \mintinline{c}{"%hx"}, \mintinline{c}{"%ho"}      \\
        \hline
            \mintinline{c}{unsigned short int}      & \mintinline{c}{"%hu"}, \mintinline{c}{"%hx"}, \mintinline{c}{"%ho"}      \\
        \hline
            \mintinline{c}{int}                     & \mintinline{c}{"%i"},  \mintinline{c}{"%x"},  \mintinline{c}{"%o"}   \\
        \hline
            \mintinline{c}{unsigned int}            & \mintinline{c}{"%u"},  \mintinline{c}{"%x"},  \mintinline{c}{"%o"} \\
        \hline
            \mintinline{c}{long int}                & \mintinline{c}{"%li"}, \mintinline{c}{"%lx"}, \mintinline{c}{"%lo"} \\
        \hline
            \mintinline{c}{unsigned long int}       & \mintinline{c}{"%lu"}, \mintinline{c}{"%lx"}, \mintinline{c}{"%lo"} \\
        \hline
            \mintinline{c}{long long int}           & \mintinline{c}{"%lli"},\mintinline{c}{"%llx"},\mintinline{c}{"%llo"} \\
        \hline
            \mintinline{c}{unsigned long long int}  & \mintinline{c}{"%llu"},\mintinline{c}{"%llx"},\mintinline{c}{"%llo"} \\
        \hline
            \mintinline{c}{float}                   & \mintinline{c}{"%f"},  \mintinline{c}{"%e"},  \mintinline{c}{"%g"}, \mintinline{c}{"%a"} \\
        \hline
            \mintinline{c}{double}                  & \mintinline{c}{"%f"},  \mintinline{c}{"%e"},  \mintinline{c}{"%g"}, \mintinline{c}{"%a"} \\
        \hline 
            \mintinline{c}{long double}             & \mintinline{c}{"%Lf"}, \mintinline{c}{"$Le"}, \mintinline{c}{"%Lg"} \\ 
        \hline
    \end{tabular}
\end{center}
Taken from ``Programming in C'', Kochan.

\subsection{Examples}
\inputcode{\lang}{\code/format_specifiers.c}


%----------------------------------------------------------------------------------------
\section{Command line arguments}
\begin{itemize}
    \item Sometimes data is required via the command line. 
    \item In order to send command line arguments add the main() function parameters, and compile. After compiling type in the command line: \placeholder{executable name} \placeholder{command line args}
\end{itemize}
\inputcode{\lang}{\code/command_line_args.c}


%----------------------------------------------------------------------------------------
\section{Challenge 1}
\inputcode{\lang}{\code/challenge_command_line_perimeter.c}
\inputcode{\lang}{\code/challenge_enums.c}
