\section{What beginners don't know}
\begin{itemize}
    \item Command line basics: 
        \begin{itemize}
            \item Windows, macOS, Linux 
        \end{itemize}
    
    \item Installing python and making it a path environment variable. 
    \item Difference between distributions.
    \item How to google search for answering one's own questions. 
    \item How to run python code. 
    \item Difference between an IDE and python. 
    \item What is syntax highlighting. 
    \item Variables and their purpose. 
\end{itemize}


%----------------------------------------------------------------------------------------

\section{Basic data types}
\begin{center}
    \begin{tabular}{ |l|l| }
        \hline
            Name & Type  \\
        \hline
            Integers & int \\ 
        \hline
            Floating point & float \\ 
        \hline
            Strings & str \\ 
        \hline
            Lists & list \\ 
        \hline
            Dictionaries & dict \\ 
        \hline
            Tuples & tup \\ 
        \hline
            Sets & set \\ 
        \hline
            Booleans & bool \\ 
        \hline
    \end{tabular}
\end{center}


%----------------------------------------------------------------------------------------
\section{Numbers}
\begin{center}
    \begin{tabular}{ |p{5cm}|p{5cm}|p{5cm}| }
        \hline\\[0.25cm]
            Modulo & 20 \% 2 & $\displaystyle 20\times\pmod{2}$  \\[0.45cm]
            \hline
            Powers & 2 ** 2  & $\displaystyle 2^2$  \\[0.45cm]
            \hline
            Floor & 10 // 3  & $\displaystyle \floor{\frac{10}{3} } $  \\[0.45cm]
            \hline
            Grouping & (10 + 3) / 3 & $\displaystyle \frac{10+3}{3} $ \\[0.45cm] 
        \hline
    \end{tabular}
\end{center}

%----------------------------------------------------------------------------------------
\section{Variable names}
Rules: 
\begin{itemize}
    \item No spaces 
    \item Can't use these: \mintinline{python}{;"',<>\()!@#$%^&*~-+|}
    \item Don't use keywords. 
\end{itemize}
\subsection{Dynamic typing}
\begin{itemize}
    \item Python is a dynamically typed language.
    \item This means you can reassign variables to different data types. 
    \item This makes Python very flexible in assigning data types, this is different to other languages that are ``Statically-Typed''.
\end{itemize}
\begin{center}
    \begin{tabular}{ |p{8.5cm}|p{8.5cm}| }
        \hline
            Pros & Cons \\
        \hline
            \begin{itemize}
                \item Easy to work with
                \item Faster development process 
            \end{itemize}
            & 
            \begin{itemize}
                \item May result in bugs for unexpected data types.
                \item You need to be aware of \verb|type()| function to check this. 
            \end{itemize}
            \\ 
        \hline
    \end{tabular}
\end{center}



%----------------------------------------------------------------------------------------
\section{Strings}
\begin{itemize}
    \item In Python ' and " are the same in strings. 
    \item String indexing: used to extract a single character in a string. a[index] 
    \item String slicing: grabs an interval of indexes. a[start:stop:step]
        \begin{itemize}
            \item start: numerical index for the slice start. 
            \item stop index you will go up to but not include. 
            \item step: the size of the jump you take. 
        \end{itemize}
    
    \item Special characters: \verb|\n, \t, \d|
    \item the lenght function: len(a) -$>$ how many elements are in the string. 
\end{itemize}
\subsection{Normal and reverse indexes}
\begin{center}
    \begin{tabular}{ |l|l|l|l|l|l|l|l|l|l|l|l|l| }
        \hline
            H & E & L & L & O &   & W & O & R & L & D & ! \\
            \hline
            0 & 1 & 2 & 3 & 4 & 5 & 6 & 7 & 8 & 9 & 10 & 11 \\ 
            \hline
            0 & -11 & -10 & -9 & -8 & -7 & -6 & -5 & -4 & -3 & -2 & -1 \\ 
        \hline
    \end{tabular}
\end{center}
\subsection{Examples}
\begin{minted}[autogobble]{python}
    # STRING INDEXES AND REVERSE INDEXES
    a = "123456"
    a[1:2] 
    # output: "2" -> Index 1 up to but not including 2
    a[1:3]  
    # output: "23" -> Index 1 up to but not including 3
    a[:3]   
    # output: "123" -> Index 0 (default) up to but not including 3
    a[::]   
    # output: "123456" -> Normal string 
\end{minted}


%----------------------------------------------------------------------------------------
\section{String properties}
\begin{itemize}
    \item Strings are inmutable: 
        \begin{minted}[autogobble]{python}
            name = "david"
            name[0] = "b" 
            # OUTPUT: 
            # Traceback (most recent call last):
            #   File "<stdin>" line 1, in <module>
            # TypeError: 'str' object does not support item assignment
        \end{minted}

        
    
    \item You can concactenate using the + and * signs
        \begin{minted}[autogobble]{python}
            2 + 3 
            # output: 5
            "2" + "3" 
            # output: "23"
            "h" * 10 
            # output: "hhhhhhhhhh" 
        \end{minted}
    
    \item Methods: 
        \begin{itemize}
            \item .upper()
            \item .lower()
            \item .replace()
            \item .split()
        \end{itemize}
\end{itemize}


%----------------------------------------------------------------------------------------
\section{Print formating with strings}
\begin{itemize}
    \item .format() for string interpolation:
        \begin{minted}[autogobble]{python}
            print("this is a string {}".format("inserted"))
            # output: This is a string inserted

            print("The {} {} {}".format("fox","brown","quick"))
            # output:  The fox brown quick 

            print("The {2} {1} {0}".format("fox","brown","quick"))
            # output: The quick brown fox

            print("The {a} {b} {c}".format(a="fox",b="brown",c="quick"))
            # output: The quick brown fox

            result = 100/777
            print(result) 
            # output: 0.1287001287001287
            # Float formatting formula "{value:width.precision f}"
            print("The result was {r:1.3f}".format(r=result))
            # output: The result was 0.129 
        \end{minted}

    \item f-strings for string interpolation: 
        \begin{minted}[autogobble]{python}
            # F-STRING FORMATING
            var = "World"
            print(f"Hello {var}")
            # output: Hello World
        \end{minted}
\end{itemize}

%----------------------------------------------------------------------------------------
\section{Lists}
\begin{itemize}
    \item They suport indexing and slicing.
        \begin{minted}[autogobble]{python}
            my_list = [1,2,3] 
            my_list[1:]
            # output: [2,3]
            another_list = [4,5]
            my_list + another_list 
            # output: [1,2,3,4,5]
        \end{minted}
        
    \item You can also concactenate lists together. 
    \item Lists are mutable. 
\end{itemize}
\subsection{Methods}
\begin{itemize}
    \item .append(\placeholder{something}) -$>$ Appends to the list. 
    \item .sort() -$>$ sorts the iterable alphabetically or numericaly, this is a void, it doen't return anything, don't assign it to anything because it returns None.
        \begin{minted}[autogobble]{python}
            new_list = ['a','e','x','b','c']
            new_list.sort()
            new_list
            # output: ['a','b','c','e','x']
        \end{minted}
        
    \item .pop() $\rightarrow$  eliminates last element and prints it.
        \begin{minted}[autogobble]{python}
            my_list.pop(0) 
            # output: pops element 0.
        \end{minted}
        
    \item .reverse() $\rightarrow$ returns the reverse of the list. 
\end{itemize}

%----------------------------------------------------------------------------------------
\section{Dictionaries}
\begin{itemize}
    \item Doesn't need indexes. Just keys and values. 
    \item Dictionaries are unordered and can't be sorted.
    \item Methods:
        \begin{minted}[autogobble]{python}
            d = {"key1":1,"key2":2}
        \end{minted}
\end{itemize}

\subsection{Methods}
\begin{itemize}
    \item d.keys() $\rightarrow$ returns all the keys in a dictionary.
        \begin{minted}[autogobble]{python}
            d.keys() 
            # output:  dict_keys(['key1','key2'])
        \end{minted}

    \item d.values() $\rightarrow$ returns all the values in a dictionary.
        \begin{minted}[autogobble]{python}
            d.values() 
            # output: dict_values([1,2])
        \end{minted}

\end{itemize}

%----------------------------------------------------------------------------------------
\section{Tuples}
\begin{itemize}
    \item They are immutable lists essentially. 
        \begin{minted}[autogobble]{python}
            t = ('a','a','b')
            t[0] = 'NEW'
            # output: 
            # Traceback (most recent call last):
            #   File "<stdin>", line 1, in <module>
            # TypeError: 'tuple' object does not support item assignment
        \end{minted}
    \item Useful for immutable applications and memory efficiency.
\end{itemize}
\subsection{Methods}
\begin{itemize}
    \item t.count('a') $\rightarrow$ counts the occurrences of 'a' in the tuple. 
    \item t.index('a') $\rightarrow$ returns the index in which the first time 'a' appears. 
\end{itemize}

%----------------------------------------------------------------------------------------
\section{Sets}
\begin{itemize}
    \item Unordered collection of unique elements. 
        \begin{minted}[autogobble]{python}
            my_set = set()
        \end{minted}
    \item There can only be one of each element, no duplicates. 
    \item Useful for duplicate deletion. 
\end{itemize}

\subsection{Methods}
\begin{itemize}
    \item .add() $\rightarrow$ inserts 1 to the set. 
        \begin{minted}[autogobble]{python}
            my_set.add(1) # output: {1}
            my_set.add(2) # output: {1,2}
            my_set.add(2) # output: {1,2} -> doesn't add because that is a duplicate. 
        \end{minted}
\end{itemize}

%----------------------------------------------------------------------------------------
\section{Boolans}
\begin{itemize}
    \item Allows only True, False. 
\end{itemize}

%----------------------------------------------------------------------------------------
\section{Files}
\begin{itemize}
    \item Opening syntax: 
        \begin{minted}[autogobble]{python}
            with open('myfile.txt') as f: 
                contents = f.read()
            # or
            contents = f.read('myfile.txt')
        \end{minted}
\end{itemize}
\subsection{Methods}
\begin{itemize}
    \item .read() $\rightarrow$ returns a string with all the contents of the file. 
    \item .readlines() $\rightarrow$ returns a list with the lines in a file. 
    \item .seek(\placeholder{index}) $\rightarrow$ sets the cursor to the specified index (usually 0).
    \item .truncate() $\rightarrow$ deletes all information in a file. 
    \item .close() $\rightarrow$ closes the file, it's a best practice. 
\end{itemize}
\subsection{Permissions on the with open('myfile.txt') statement }
\begin{itemize}
    \item \mintinline{python}{mode='r'} $\rightarrow$ permissions are enabled to read a file. 
    \item \mintinline{python}{mode='w'} $\rightarrow$ permissions are enabled to write and overwrite to a file, or create a new file. 
    \item \mintinline{python}{mode='r+'} $\rightarrow$ reading and writing permissions.
    \item \mintinline{python}{mode='w+'} $\rightarrow$ writing and reading, overwriting or creating new files. 
    \item \mintinline{python}{mode='a'} $\rightarrow$ appends data to the end of a file, overwriting and writing to a file permissions are enabled. 
\end{itemize}

