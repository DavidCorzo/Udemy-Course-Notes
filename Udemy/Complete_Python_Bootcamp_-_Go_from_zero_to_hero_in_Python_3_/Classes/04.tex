\section{Functions on python}
Syntax is: 
\begin{minted}[autogobble]{python}
    def name_of_function(args):
        """ Doc strings """
        pass
\end{minted}


%----------------------------------------------------------------------------------------
\section{*args and **kwargs}
Example of *args: 
\begin{minted}[autogobble]{python}
    def myfunc(*args):
        return sum(args) * 0.05
    myfunc(1,2,3,4,5,6,7,8,9,10)
    # output: 2.75
\end{minted}

Example of **kwargs: 
\begin{minted}[autogobble]{python}
    def myfunckwarg(**kwargs):
        if 'fruit' in kwargs: 
            print(f"My fruit of chice is {kwargs['fruit']}")
        else: 
            print('I didn\'t find anything')
    myfunc(fruit='watermelon')
    # output: My fruit of chice is watermelon
\end{minted}

Using **kwargs like this creates a dictionary: 
\begin{minted}[autogobble]{python}
    myfunc(fruit='apple',veggie='lettuce')
    # output: {'fruit':'apple','veggie':'lettuce'}
\end{minted}

Use them in combination: 
\begin{minted}[autogobble]{python}
    def myfunc(*args, **kwargs): 
        print('I would like {} {}'.format(args[0],kwargs['food']))
    myfunc(1,2,3,4,5,6,7,8,9,'Hello','None') 
    # output: I would like 1 None
\end{minted}


%----------------------------------------------------------------------------------------
\section{Lambda expressions Map and Filter}
\begin{itemize}
    \item map(func, iterable): aplies a function to an iterable.
        \begin{minted}[autogobble]{python}
            def square(num):
                return num**2
            my_nums = [1,2,3,4,5]
            for item in map(square,my_nums):
                print(item)
            # output: 1
            # output: 4
            # output: 9
            # output: 16
            # output: 25
            list(map(square,my_nums))
            # output: [1, 4, 9, 16, 25]
        \end{minted}

    \item filter(func,iterable): keeps element if given function returns true.
        \begin{minted}[autogobble]{python}
            def check_even(num):
                return num%2 == 0
            mynums = [1,2,3,4,5,6]
            list(filter(check_even,mynums))
            # output: [2, 4, 6]
            for n in filter(check_even,mynums):
                print(n)
            # output: 2
            # output: 4
            # output: 6
        \end{minted}

    \item lambda: are one use functions. 
        \begin{itemize}
            \item Syntax is: lambda \placeholder{function arguments}: \placeholder{what will be returned}   
        \end{itemize}
        \begin{minted}[autogobble]{python}
            square = lambda num: num ** 2
            square(5)
            # output: 25
            list(map(lambda num: num**2,mynums))
            # output: [1, 4, 9, 16, 25, 36]
            list(map(lambda num: num%2 == 0,mynums))
            # output: [2, 4, 6]
            names = ['Hello','World','!']
            list(map(lambda string:string[0],names))
            # output: ['H', 'W', '!']
        \end{minted}
\end{itemize}



%----------------------------------------------------------------------------------------
\section{Nested statements and scope}
\subsection{LEGB Rule}
\begin{itemize}
    \item L: Local - all names assigned inside a function (def or lambda).
        \begin{minted}[autogobble]{python}
            lambda num: num**2 # num is local
        \end{minted}
    \item E: Enclosing function locals - all names assigned to functions inside functions. (def or lambda), from inner to outer.  
        \begin{minted}[autogobble]{python}
            name = "Global string"
            def greet():
                name = 'Sammy'
                def hello():
                    print('Hello ' + name)
                hello()
            greet()
            # output: Hello Sammy
        \end{minted}
    \item G: Global (module) - names assigned at the top level of a modile file, or declared global in def within the file.
        \begin{minted}[autogobble]{python}
            # GLOBAL
            name = "Global string"
            def greet():
                # ENCLOSING 
                name = 'Sammy'
                def hello():
                    # LOCAL
                    name = 'Im a local'
                    print('Hello ' + name)
                hello()
            greet()
            # output: Hello Im a local
        \end{minted}
    \item B: Built-in (Python) - names preassigned in the buit-in names module: open,range,SyntaxErrore,etc.
\end{itemize}

\subsection{global keyword}
\begin{itemize}
    \item You must remove the global variable as a parameter, then you declare it as: global \placeholder{var name}  
        \begin{minted}[autogobble]{python}
            x = 50 
            def func():
                global x # go to the global space and grab the variable x
                # Anything that happens with x will be equivalent as it happening on a global scope.
                print(f"x is {x}")
                #LOCAL REASSIGNMENT ON A GLOBAL VARIABLE
                x = 'new value'
                print(f"I JUST LOCALLY CHANGED X TO {x}")
            print(x)
            # output: 50
            func() 
            # output: x is 50
            # output: I JUST LOCALLY CHANGED X TO new value
            print(x)
            # output: new value
        \end{minted}

    \item The global keyword is hard to predict, if you are working with lots of scripts you might make it more dificult to debug or you might overwrite variables in the global space unintentionally. So instead do this: 
        \begin{minted}[autogobble]{python}
            # EQUIVALENT OF ALTERNATIVE
            x = 50 
            def func(x):
                print(f"x is {x}")
                #LOCAL REASSIGNMENT ON A GLOBAL VARIABLE
                x = 'new value'
                print(f"I JUST LOCALLY CHANGED X TO {x}")
                return x
            x = func(x)
            # output: x is 50 
            # output: I JUST LOCALLY CHANGED X TO new value
            print(x)
            # output: new value
        \end{minted}
\end{itemize}
