\section{Create a new project in Code::Blocks}
\begin{itemize}[label=$\downarrow$]
   \item File 
   \item New 
   \item Project
\end{itemize}

\section{Set up a project in vscode}
\begin{itemize}[label=$\downarrow$]
   \item Install C/C++ extension from Microsoft. 
   \item Search in command palette: \verb|C/C++: Edit configurations (UI)|
      \begin{itemize}
         \item Search for the compiler path and set it for: \verb|C:/cygwin64/bin/gcc.exe| 
         \item Set intelligence mode to: gcc-x64
         \item This will create a .vscode folder containing a file called \verb|c_cpp_properties.json|
         \item Open the \verb|c_cpp_properties.json|:
            \inputcode{json}{Code/.vscode/c_cpp_properties.json}
      \end{itemize}
   \item Search in command palette: \verb|Tasks: Configure Default Build Task|
      \begin{itemize}
         \item Create a new \verb|tasks.json| file from template.
         \item Then select the \verb|others| template.
         \item Enter the following arguments: 
            \begin{itemize}
               \item ``label'': ``build \placeholder{name of project}''
               \item ``type'': ``shell''
               \item ``command'': ``gcc''
               \item ``args'': 
                  \begin{itemize}
                     \item -g means global 
                     \item -o means to modify the name of the file to.
                     \item write the source file with and without extension: ``\placeholder{name}'', ``\placeholder{name.c}''
                  \end{itemize}
            \end{itemize}
            \inputcode{json}{Code/.vscode/tasks.json}
      \end{itemize}
   
   \item Search in command palette: \verb|Debug: Open launch.json| then select the \verb|C++(GDB/LLDB)| environment.
      \begin{itemize}
         \item Change the workspace folder in ``program''
         \item set stopAtEntry to true 
         \item set miDebuggerPath to \verb|C:/cygwin64/bin/gdb.exe|
      \end{itemize}
      \inputcode{json}{Code/.vscode/launch.json}

      
   \item Search in command palette: \verb|Preferences: Open Settings (JSON)|
      \begin{itemize}
         \item set file.associations to ``*.c'':``c''
         \item set ``terminal.integrated.shell.windows'':``\verb|C:\\cygwin64\\bin\\bash.exe|'' to specify a default terminal.
      \end{itemize}
      \inputcode{json}{Code/.vscode/settings.json}
   
   \item Exit the IDE and reenter so that the changes specified go in to effect. 
\end{itemize}

%----------------------------------------------------------------------------------------

\subsection{Use the settings and shortcuts}  
\begin{itemize}
   \item F5 to display the debugger. 
\end{itemize}


%%%%%%%%%%%%%%%%%%%%%%%%%%%%%%%%%%%%%%%%%%%%%%%%%%%%%%%%%%%%%%%%%%%%%%%%%%%%%%%%%%%%%%%%%%
\section{Structure of a c programs}
\begin{itemize}
   \item main() is a function, it's the entry point of a program, there should be at least one main in all c programs and only one main function.
   \item C is a case-sensitive language.
   \item C isn't tab sensitive like Python, you can actually not indent code and it will compile, however it's recommended to do so. 
   \item Readability is very important. 
\end{itemize}
