\section{Overview}
\begin{itemize}
    \item Operators are functions that use symbolic name, used for mathematical and logical functions. 
    \item Operators are predefined in C, 
\end{itemize}
\subsection{Operators}
\begin{itemize}
    \item Logical operators. 
    \item Arithmetic operators. 
    \item Relational.
    \item Bitwise. 
\end{itemize}

\subsection{Expressions and statements}
\begin{itemize}
    \item They are different. 
    \item Statements: form the basic program in C, most statements are constructed from expressions. 
        \begin{itemize}
            \item They are the building blocks of a programs. Simple statements are short oneliners that end in a ; 
            \item A complete instruction to the computer.
            \item Examples of simple statements: 
                \begin{itemize}
                    \item declaration statement: \mintinline{c}{int a;}
                    \item Assignment statement: \mintinline{c}{a = 5;}
                    \item Function call statement: \mintinline{c}{funct();}
                    \item Structure statement: \mintinline{c}{while (condition) var = num;}
                    \item return statement: \mintinline{c}{return 0;}
                \end{itemize}
            \item C considers any expression to be a statement if you append a semicolon (expression statements).
        \end{itemize}
    \item Expressions: consists of a combination of operators and operands: example: -6, 4+21
        \begin{itemize}
            \item Operands are what an operator operates on. 
            \item Operands can be constants, variables, or combinations of the two.
            \item Every expression has value. 
            \item Expressions generally return values. 
        \end{itemize}
\end{itemize}
\subsection{Compond statements}
\begin{itemize}
    \item Two or more statements grouped together by enclosing them in braces (block).
    \item Example: \mintinline{c}{int i = 0; while (i < 10) {printf("hello");i++;}} the while statement is enclosing the printf() function statement, they are compound statements. 
\end{itemize}


%----------------------------------------------------------------------------------------
\section{Basic operators}
\begin{itemize}
    \item Arithmetic: performs mathematical operations. 
    \item Logical operator: boolean operator. 
    \item Assignment operators: set variables equal to values. 
    \item Relational operator: compares variables against each other.
\end{itemize}
\begin{center}
    \begin{tabular}{ |p{5cm}|p{7cm}| }
        \hline
            Operator & Type  \\
        \hline
            \mintinline{c}{+} & Addition \\ 
            \mintinline{c}{-} & Subtraction \\ 
            \mintinline{c}{*} & Multiplication \\ 
            \mintinline{c}{/} & Division \\ 
            \mintinline{c}{%} & Modulus \\ 
            \mintinline{c}{++} & Increment by one. \\ 
            \mintinline{c}{--} & Decrement by one. \\ 
            \hline
    \end{tabular}
\end{center}
Taken from tutorial point website.
\subsubsection{Example}
\inputcode{\lang}{\code/arithmetic_operators.c}

\subsection{Logical operators}
\begin{center}
    \begin{tabular}{ |p{5cm}|p{7cm}| }
        \hline
            Operator & Type \\
        \hline
            \mintinline{c}{&&} & and \\ 
            \mintinline{c}{|} & or operator \\ 
            \mintinline{c}{!} & not operator \\ 
        \hline
    \end{tabular}
\end{center}
\subsubsection{Example}
\inputcode{\lang}{\code/logical_operators.c}

\subsection{Assignment operators}
\begin{center}
    \begin{tabular}{ |p{5cm}|p{7cm}| }
        \hline
            Operator & Type \\
        \hline
           \mintinline{c}{=} & assignment operator \\ 
           \mintinline{c}{+=} & add and assignment operator \\ 
           \mintinline{c}{-=} & subtract and assign operator \\ 
           \mintinline{c}{*=} & multiply and assign operator \\ 
           \mintinline{c}{/=} & divide and assign operator \\ 
           \mintinline{c}{%=} & modulus and assign operator \\ 
        \hline
    \end{tabular}
\end{center}

\subsection{Relational operators}
\begin{center}
    \begin{tabular}{ |p{5cm}|p{7cm}| }
        \hline
            Operator & Type \\
        \hline
           \mintinline{c}{==} & equality operator \\ 
           \mintinline{c}{!=} & inequality operator \\ 
           \mintinline{c}{>} & greater than \\ 
           \mintinline{c}{<} & less than \\ 
           \mintinline{c}{>=} & greater or equal than \\ 
           \mintinline{c}{<=} & less than or equal than \\ 
        \hline
    \end{tabular}
\end{center}

\section{Bitwise operators}
\begin{itemize}
    \item C offers bitwise logical operators and shift operators. 
    \item They operate of bits, inside the integers. 
    \item Useful for: turning on and off bits inside data types. 
    \item Example: using just one int variable store binary data such as (using 0 bit, or 8 bits): makes you more memory efficient. 
        \begin{itemize}
            \item First bit store weather a person is female or male. 
            \item Second bit if they can speak English. 
            \item Third bit to represent if they can speak German. 
            \item Fourth bit ... eight bit. 
        \end{itemize}
\end{itemize}
\subsection{Binary numbers}
\begin{itemize}
    \item A binary number is a number that includes only ones or zeros. 
    \item Each position of a binary number has value. 
\end{itemize}
Bitwise: 
\begin{center}
    \begin{tabular}{ |p{5cm}|p{7cm}| }
        \hline
            \mintinline{c}{&} & Binary AND operator \\
            \mintinline{c}{|} & Binary OR operator \\
            \mintinline{c}{^} & Binary XOR operator \\
            \mintinline{c}{~} & Binary ones component operator \\
            \mintinline{c}{<<} & Binary Left Shift operator \\
            \mintinline{c}{>>} & Binary Right Shift operator \\
        \hline
    \end{tabular}
\end{center}
Truth table: 
\begin{center}
    \begin{tabular}{ |p{2cm}|p{2cm}|p{2cm}|p{2cm}|p{2cm}| }
        \hline
    $p$ & $q$ & $p$\&$q$ & $p$\textbar$q$  & $p$\verb|^|$q$   \\
        \hline
        0& 0& 0& 0& 0\\ 
        0& 1& 0& 1& 1\\ 
        1& 1& 1& 1& 0\\ 
        1& 0& 0& 1& 1\\ 
        \hline
    \end{tabular}
\end{center}
\subsection{Examples}
\inputcode{\lang}{\code/bitwise_operators.c}


%----------------------------------------------------------------------------------------
\section{The cast and sizeof operators}
\begin{itemize}
    \item These look like functions, but they are not, they are actually operators.
\end{itemize}
\subsection{Type conversions}
\begin{itemize}
    \item You can convert data automatically, known as implicit conversion. 
    \item Converted data can be truncated or promoted, truncated is become less precise, promoted is become more precise. 
    \item Example: a float is converted to an int, decimal portion will be truncated. \mintinline{c}{int x = 0; float f = 12.125; x = f} x will only keep the integer part. This is an example of implicit conversion. 
    \item By contrast if you go from int to float, nothing will happen, this will be promoted. 
    \item Two ints are doing division, the result is going to be truncated to an int if it is a decimal. 
    \item Explicit conversions are done with the casting operator, you want to do this to accurately transcribe the data to the desired type. 
    \item Type cast operator has higher precedence than all the arithmetic operators except unary minus and unary plus. 
    \item Example: \mintinline{c}{(int)21.51 + (int)26.99} is going to be \mintinline{c}{21 + 26}
\end{itemize}
\subsection{sizeof}
\begin{itemize}
    \item Tells you how many bytes are occupied in memory. 
    \item sizeof() is an operator not a function. 
    \item sizeof() is evaluated in compile time, not in runtime, unless a variable-length array is used in its argument. 
    \item Arguments of sizeof can be: basic data types, array names, variables, name derived from a data type, or an expression (such as \mintinline{c}{x+y}). 
    \item Use the sizeof as much as you can, sizeof(int) can be 4 in one system or 8 in another, you want programs that aren't system dependent. 
\end{itemize}
Other operators: 
\begin{itemize}
    \item \mintinline{c}{*} is used for pointers. 
    \item \mintinline{c}{?:} is an operator used for comparisons: 
        \begin{itemize}
            \item if condition is true ? then value x : otherwise value y.
            \item name is the ternary operator. 
        \end{itemize}
\end{itemize}


%----------------------------------------------------------------------------------------
\section{Operator precedence}
\begin{itemize}
    \item Determines how an expression is evaluated. 
    \item Determines the order of evaluation in the expression. 
    \item If parentheses are provided to prioritize an expression to evaluate. 
    \item Associativity: rules that determine what to do in a situation such as: two expressions have the same precedence, associativity is used to apply additional rules in order to determine which goes first. 
    \item Associativity usually favors what comes first. \mintinline{c}{1 == 2 != 3}: what do you evaluate first?, the prior expression is equal to \mintinline{c}{((1 == 2) != 3)}, this is implicit, provide the parentheses in order to prioritize an expression.  
\end{itemize}
Operator precedence (highest to lowest):
\begin{center}
    \begin{tabular}{ |p{4cm}|p{5cm}|p{6cm}| }
        \hline
            Category & Operator & Associativity  \\
        \hline
            Postfix & \mintinline{c}{()[]->.++--} & left to right \\ 
            Unary & \mintinline{c}{+-!~++--(type)*&sizeof} & right to left \\ 
            Multiplicative & \mintinline{c}{*/%} & left to right \\ 
            Additive & \mintinline{c}{+-} & left to right \\ 
            Shift & \mintinline{c}{<<>>} & left to right \\ 
            Relational & \mintinline{c}{<<=>>=} & left to right \\ 
            Equality & \mintinline{c}{== !=} & left to right \\ 
        \hline
    \end{tabular}
\end{center}
Bitwise precedence (highest to lowest):
\begin{center}
    \begin{tabular}{ |p{4cm}|p{5cm}|p{6cm} }
        \hline
            Category & Operator & Associativity \\
        \hline
            Bitwise AND & \mintinline{c}{&} & left to right  \\ 
            Bitwise XOR & \mintinline{c}{^} & left to right  \\ 
            Bitwise OR & \mintinline{c}{|} & left to right \\ 
            Logical AND & \mintinline{c}{&&} & left to right  \\ 
            Logical OR & \mintinline{c}{||} &  left to right \\ 
            Conditional & \mintinline{c}{?:} & right to left  \\ 
            Assignment  & \mintinline{c}{=+=-=*=/=%/=>>=<<=&=^=|=} & right to left  \\ 
            Comma & \mintinline{c}{,} & left to right \\ 
        \hline
    \end{tabular}
\end{center}

\section{Challenge years to days provided minutes}
\inputcode{\lang}{\code/challenge_years_min_days.c}

\section{Challenge on printing the bytesize of data}
\inputcode{\lang}{\code/bytesize_of_data.c}
