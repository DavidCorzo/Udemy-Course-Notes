
%----------------------------------------------------------------------------------------
\section{Writting our first program}
\begin{itemize}
    \item Create a project.
    \item Create a file and type the following code. 
    \item This program is going to take in a number and then display "Wow that is my favorite number".
\end{itemize} 
\begin{minted}[autogobble]{cpp}
    #include <iostream>
    int main() {
        int favorite_number; // stores what the user will enter.
        std::cout << "Enter your favorite number between 1 and 100"; // prints.
        std::cin >> favorite_number;
        std::cout << "Amazing!! That's my favorite number too!" << std::endl; // prints that line. endl adds "\n" and flushes the buffer.
    }
\end{minted}


%----------------------------------------------------------------------------------------
\section{Building our first program}
\begin{itemize}
    \item Building involves compiling and linking.
    \item In vscode you can run the compile task by pressing ctrl+b.
    \item Linking means grabbing all the dependencies the main function needs, making .o or object files and adding them to the executable or the .exe.
    \item Tipically modern compilers have the option to not produce the object files and go ahead and just produce a single executable. IDEs tipically also hide the object files if they are produced.
    \item By \emph{cleaning} a project what we mean is the object files are deleted and an executable will be produced.
\end{itemize}

%----------------------------------------------------------------------------------------
\section{What are compiler errors?}
\begin{itemize}
    \item Programming languages have rules.
    \item Syntax errors: something wrong with the structure.
        \begin{minted}[autogobble]{cpp}
            std::cout << "Errors << std::endl; // the string is never terminated.
        \end{minted}
    
    \item Semantic errors: something wrong with the meaning:
        \begin{minted}[autogobble]{cpp}
            a + b; // to sum a and b when it doesn't make sense to add them, maybe they are not numbers for example.
        \end{minted}
\end{itemize}

\subsection{Examples of errors}
\noindent
Not enclosing a string with the " characters.
\begin{minted}[autogobble]{cpp}
    int main() {
        std::cout << "Hello world << std::endl; // string is not terminated with the other ".
        return 0;
    }
\end{minted}
\noindent
Typos in your program: 
\begin{minted}[autogobble]{cpp}
    int main() {
        std::cout << "Hello world" << std::endll; // endll doesn't exist, this is syntaxis errors.
        return 0;
    }
\end{minted}
\noindent
Missing semi-colons: 
\begin{minted}[autogobble]{cpp}
    int main() {
        std::cout << "Hello world" << std::endl // missing semicolon. 
        return 0;
    }
\end{minted}
\noindent
Function doesn't return the type specified, in this case the function doesn't return an integer.
\begin{minted}[autogobble]{cpp}
    int main() {
        std::cout << "Hello" << std::endl;
        return; // main needs to return an integer and it is returning a void.
    }
\end{minted}
\noindent
Not returning the specified type.
\begin{minted}[autogobble]{cpp}
    int main() {
        std::cout << "Hello World" << std::endl;
        return "Hello"; // "Hello" is not an integer. Error.
    }
\end{minted}
\noindent
Missing Curly brace: 
\begin{minted}[autogobble]{cpp}
    int main() // opening curly brace missing.
        std::cout << "Hello World" << std::endl;
    }
\end{minted}
\noindent
Semantic error (example: adding something when it doesn't make sense).
\begin{minted}[autogobble]{cpp}
    int main() {
        std::cout << ("Hello world" / 125) << std::endl; // dividing a string by a number, this doesn't make sense.
        return 0;
    }
\end{minted}


%----------------------------------------------------------------------------------------
\section{What are compiler warnings?}
\begin{itemize}
    \item It is good practice to never ignore compiler warnings.
    \item The compiler will recognize a potential issue but is still able to produce object code from the source code, things such as uninitialized variables.
    \item It's only a warning because the compiler is still able to generate correct machine code.
    \item Example: 
        \begin{minted}[autogobble]{cpp}
            int miles_driven; // never initialized, this value could be anything.
            std::cout << miles_driven << std::endl;
            /* Warning: 'miles_driven' is used uninitialized in this function. */
        \end{minted}
    
    \item Another example is when you declare variables and never use them.
        \begin{minted}[autogobble]{cpp}
            int miles_driven = 100; 
            std::cout << "Hello world" << std::endl;
            /* Warning: unused variable 'miles_driven'. */
        \end{minted}
    
    \item As a rule you want to produce warning free source code.
\end{itemize}


%----------------------------------------------------------------------------------------
\section{Linked errors}
\begin{itemize}
    \item The linker is having trouble linking all the object files together to create an executable.
    \item Usually there is a library or object file that is missing.
\end{itemize}
\subsection{Example}
\begin{minted}[autogobble]{cpp}
    #include <iostream>
    extern int x; // this means the variable is defined outside this file.
    int main() {
        std::cout << "Hello world" << std::endl;
        std::cout << x;
        return 0;
    }
    /* This program will compile, but in runtime you will get a linker error. */
\end{minted}
\begin{figure}[H]
    \centering
    \includegraphics[width=0.5\textwidth]{\figs/linker_e}
\end{figure}

%----------------------------------------------------------------------------------------
\section{Runtime Errors}
\begin{itemize}
    \item Errors that occur when the program is executing.
    \item Some typical runtime errors include:
        \begin{itemize}
            \item Divide by zero.
            \item File not found.
            \item Out of memory.
        \end{itemize}
    
    \item Can cause your program to crash.
    \item Exception handling can help deal with runtime errors.
\end{itemize}

%----------------------------------------------------------------------------------------
\section{What are logic errors?}
\begin{itemize}
    \item Errors or bugs in your code that cause your program to run incorrectly.
    \item Logic errors are mistakes made by the programmer.
\end{itemize}
Suppose we have a program that determines if a person can vote in an election and you must be 18 years or older to vote.
\begin{minted}[autogobble]{cpp}
    if (age > 18) { // This means that age cannot be 18 thus 18 yearolds would not be able to vote. 18 is not included.
        std::cout << "Yes you can vote" << endl;
    }
\end{minted}

%----------------------------------------------------------------------------------------
\section{Section challenge solution}
\begin{minted}[autogobble]{cpp}
    #include <iostream>
    int main() {
        int favorite_number;
        std::cout << "Enter your favorite number between 1 and 100: ";
        std::cin >> favorite_number;
        std::cout << "Amazing!! Thats my favorite number too!" << std::endl;
        std::cout << "No really!!, " << favorite_number << " is my favorite number!" << std::endl;
    }
    /* Output:
    Enter your favorite number between 1 and 100: 67
    Amazing!! Thats my favorite number too!
    No really!!, 67 is my favorite number!
    */
\end{minted}
