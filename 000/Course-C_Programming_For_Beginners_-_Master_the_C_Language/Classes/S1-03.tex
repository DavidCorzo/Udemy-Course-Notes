\subsection{Basics}
\begin{itemize}
    \item For all the diferent problems that we need solved in the world the computer runs an algorithm.
    \item An algorithm is a set of instructions that tell the CPU what to do.
    \item To write a program you need to implement the instructions correctly in any kind of programming languages, Java, Python, Objective-C, C \& C++.
\end{itemize}
%%%%%%%%%%%%%%%%%%%%%%%%%%%%%%%%%%%%%%%%%%%%%%%%%%%%%%%%%%%%%%%%%%%%%%%%%%%%%%%%%%%%%%%%%%%%%%%%
\subsection{Terminology}
\begin{itemize}
    \item CPU:
        \begin{itemize}
            \item Central Processing Unit
            \item Excecutes the computational work.
            \item All instructionsare excecuted here.
            \item components are:
                \begin{enumerate}
                    \item Control unit 
                    \item ALU
                    \item Registers
                \end{enumerate}
        \end{itemize}

    \item RAM: 
        \begin{itemize}
            \item Random Access Memory 
            \item Stores the data of a program while it's running.
            \item A programs needs to have data to operate, this is $\neq$ to the harddrive.
            \item RAM is another hardware component and is much faster. 
        \end{itemize}
    
    \item HardDrive: 
        \begin{itemize}
            \item Stroes files that contains programs source code, it stores that data even when the computer is turned off.
            \item When a computer is turned off there is nothing in RAM but there is DATA in the harddrive.
            \item Used to store program resources.
        \end{itemize}
    
    \item Operating system:
        \begin{itemize}
            \item Makes a computer usable and easier.
            \item It's a program that runs everytime you use.
            \item Handles input, output, computer resources, storage, etc.
            \item Windows, Unix, Android, etc.
        \end{itemize}
    
    \item Fetch / Excecute cycle: 
        \begin{itemize}
            \item Fetch instructions from memory and sending it to the CPU to be excecuted.
            \item Gigahertz says the amount of times the CPU fetches and excecutes a program.
        \end{itemize}

    \item Higher level programming language:  
        \begin{itemize}
            \item Oposite of assembly language 
            \item C is higher level programming language.
            \item You don't understand the instructions sent to the CPU, but in C you don't have to wory about anything like this.
            \item A \textbf{compiler} interpretes the code writen in the higher level programming language and upon compiling it will translate your statements in that programming language in to machine code tha will then be sent to the CPU.
            \item It's esentially a translation between C to machine code.
            \item The compiler will also check wether your program has the correct syntax, it won't compile if there are errors.
        \end{itemize}
\end{itemize}

%%%%%%%%%%%%%%%%%%%%%%%%%%%%%%%%%%%%%%%%%%%%%%%%%%%%%%%%%%%%%%%%%%%%%%%%%%%%%%%%%%%%%%%%%%%%%%%%
\subsection{Writting a program} 
\begin{enumerate}
    \item Define the program objetives.
        \begin{itemize}
            \item Understand the requirements.
            \item See what you want to do.
        \end{itemize}
    \item Design the program
        \begin{itemize}
            \item Decide how to program will meet the requirements.
            \item What will be the UI.
            \item Organization in your program.
        \end{itemize}
    
    \item Write the code: 
        \begin{itemize}
            \item Start the implementation.
            \item Use an IDE.
        \end{itemize}
    
    \item Compile: 
        \begin{itemize}
            \item Translate the code to CPU.
        \end{itemize}
    
    \item Run the program: 
        \begin{itemize}
            \item Execute.
        \end{itemize}
    
    \item Test and debug: 
        \begin{itemize}
            \item Running $\neq$ properly impliemented.
            \item Debug your errors.
            \item Test by \textbf{SMALL} batches.
        \end{itemize}
    
    \item Maintain and modify the program: 
        \begin{itemize}
            \item Get your program updated and continue dubugging.
            \item Continue fixing and add new features.
            \item Tipically the most expensive step in the process.
            \item Use a methodology that works for you.
        \end{itemize}
\end{enumerate}

\subsection{Observations}
\begin{itemize}
    \item Planning before coding is easier and permits more flexibility.
    \item You always want to plan eficiently.
    \item \textbf{Divide and conquer} for debugging sake.
\end{itemize}
