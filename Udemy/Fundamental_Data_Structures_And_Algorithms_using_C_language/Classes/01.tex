\section{Introduction to the stack}
\begin{itemize}
    \item The stack is a linear data structure where the elements could be added or deleted only at one open end called the top of the stack. 
    \item The elements follow ``Last in first out'', typically called LIFO.
    \item Insertion into the stack is called ``push'' operation and deletion from the stack is called the ``pop'' operation.
        \begin{itemize}
            \item If you need to add an element to the stack, you need to call the push operation.
            \item If you need to delete an element of the stack, you need to call the pop operation.
        \end{itemize}
    
    \item If the stack is full it is said to be in ``Overflow'' state, and push is rejected if stack is overflowed.
    \item If the stack is empty it is said to be in ``Underflow'' state, and pop is rejected if the stack is underflowed. 
\end{itemize}

\subsection{Example of the stack and the push, pop method}
\begin{center}
    \begin{tabular}{|c | c | c | c | c | c | c |}
        \hline 
        \begin{tabular}{ c }
            3 \\
            2 \\
            1 \\
            0 \\
        \end{tabular}
        \begin{tabular}{ |p{0.5cm}| }
            \hline
                \\
            \hline 
                \\
            \hline 
                \\
            \hline 
                \\
            \hline
        \end{tabular}
        &
        \begin{tabular}{ c }
            3 \\
            2 \\
            1 \\
            0 \\
        \end{tabular}
        \begin{tabular}{ |p{0.5cm}| }
            \hline
                \\
            \hline 
                \\
            \hline 
                \\
            \hline 
                10 \\
            \hline
        \end{tabular}
        &
        \begin{tabular}{ c }
            3 \\
            2 \\
            1 \\
            0 \\
        \end{tabular}
        \begin{tabular}{ |p{0.5cm}| }
            \hline
                \\
            \hline 
                \\
            \hline 
                25 \\
            \hline 
                10 \\
            \hline
        \end{tabular}
        &
        \begin{tabular}{ c }
            3 \\
            2 \\
            1 \\
            0 \\
        \end{tabular}
        \begin{tabular}{ |p{0.5cm}| }
            \hline
                \\
            \hline 
                80\\
            \hline 
                25 \\
            \hline 
                10 \\
            \hline
        \end{tabular}
        & 
        \begin{tabular}{ c }
            3 \\
            2 \\
            1 \\
            0 \\
        \end{tabular}
        \begin{tabular}{ |p{0.5cm}| }
            \hline
                \\
            \hline 
                \\
            \hline 
                25 \\
            \hline 
                10 \\
            \hline
        \end{tabular} 
        & 
        \begin{tabular}{ c }
            3 \\
            2 \\
            1 \\
            0 \\
        \end{tabular}
        \begin{tabular}{ |p{0.5cm}| }
            \hline
                \\
            \hline 
                \\
            \hline 
                \\
            \hline 
                10 \\
            \hline
        \end{tabular}
        &
        \begin{tabular}{ c }
            3 \\
            2 \\
            1 \\
            0 \\
        \end{tabular}
        \begin{tabular}{ |p{0.5cm}| }
            \hline
                \\
            \hline 
                \\
            \hline 
                \\
            \hline 
                 \\
            \hline
        \end{tabular}
        \\[1cm] 
        % Coments
        Original & 
        .push(10) & 
        .push(25) & 
        .push(80) & 
        .pop() & 
        .pop() & 
        .pop()
        \\ 
        \hline
    \end{tabular}
\end{center}


%----------------------------------------------------------------------------------------
\section{Practical examples where stacks are used}
\begin{itemize}
    \item Text editors use a stack when registering new words and undoing one with ctrl+z. In the example below:
        \begin{minted}[autogobble]{java}
            stack.push("First");
            stack.push("Second");
            stack.push("Third");
        \end{minted}
        \begin{lstlisting}
            First 
            Second 
            Third
        \end{lstlisting}
        \begin{minted}[autogobble]{java}
            // ctrl+z in keyboard
            stack.pop();
        \end{minted}
        \begin{lstlisting}
            First 
            Second 
        \end{lstlisting}
        \begin{minted}[autogobble]{java}
            stack.pop();
        \end{minted}
        \begin{lstlisting}
            First 
        \end{lstlisting}
    
    \item The back button in the browser is also a stack implementation. 
    \item In the command line, the terminal saves the commands executed in a stack, the upper arrow key access the commands. The commands are stored in a stack. 
    \item There are plenty of examples. 
\end{itemize}

%----------------------------------------------------------------------------------------
\section{Basic algorithm for stack data structure}
\begin{itemize}
    \item Implementation of stack using 1-D array.
        \begin{minted}[autogobble]{c}
    int stack[5];
    int index = -1;
\end{minted}
\begin{tabular}{|c | c | c | c |}
    \hline 
    \begin{tabular}{ c }
        3 \\
        2 \\
        1 \\
        0 \\
    \end{tabular}
    \begin{tabular}{ |p{0.5cm}| }
        \hline
            \\
        \hline 
            \\
        \hline 
            \\
        \hline 
            \\
        \hline
    \end{tabular}
    &
    \begin{tabular}{ c }
        3 \\
        2 \\
        1 \\
        0 \\
    \end{tabular}
    \begin{tabular}{ |p{0.5cm}| }
        \hline
            \\
        \hline 
            \\
        \hline 
            \\
        \hline 
            10 \\
        \hline
    \end{tabular}
    &
    \begin{tabular}{ c }
        3 \\
        2 \\
        1 \\
        0 \\
    \end{tabular}
    \begin{tabular}{ |p{0.5cm}| }
        \hline
            \\
        \hline 
            \\
        \hline 
            25 \\
        \hline 
            10 \\
        \hline
    \end{tabular}
    &
    \begin{tabular}{ c }
        3 \\
        2 \\
        1 \\
        0 \\
    \end{tabular}
    \begin{tabular}{ |p{0.5cm}| }
        \hline
            \\
        \hline 
            80\\
        \hline 
            25 \\
        \hline 
            10 \\
        \hline
    \end{tabular}
    \\[1cm] 
    % Coments
    \makecell[l]{Stack is \\ empty or \\ underflowed.}
    & 
    \makecell[l]{
        \mintinline{c}{++index;} \\
        \mintinline{c}{stack[index] = 10;} \\
        \mintinline{c}{//index:0}
    }
    & 
    \makecell[l]{
        \mintinline{c}{++index;} \\
        \mintinline{c}{stack[index] = 25;} \\
        \mintinline{c}{//index:1} 
    }
    & 
    \makecell[l]{
        \mintinline{c}{++index;} \\
        \mintinline{c}{stack[index] = 80;} 
    }
    \\ \hline
    \begin{tabular}{ c }
        3 \\
        2 \\
        1 \\
        0 \\
    \end{tabular}
    \begin{tabular}{ |p{0.5cm}| }
        \hline
            \\
        \hline 
            \\
        \hline 
            25 \\
        \hline 
            10 \\
        \hline
    \end{tabular} 
    & 
    \begin{tabular}{ c }
        3 \\
        2 \\
        1 \\
        0 \\
    \end{tabular}
    \begin{tabular}{ |p{0.5cm}| }
        \hline
            \\
        \hline 
            \\
        \hline 
            \\
        \hline 
            10 \\
        \hline
    \end{tabular}
    &
    \begin{tabular}{ c }
        3 \\
        2 \\
        1 \\
        0 \\
    \end{tabular}
    \begin{tabular}{ |p{0.5cm}| }
        \hline
            \\
        \hline 
            \\
        \hline 
            \\
        \hline 
                \\
        \hline
    \end{tabular}
    &
    \\ 
    \makecell[l]{
        \mintinline{c}{stack[index] = 0;} \\
        \mintinline{c}{index--;} 
    }
    & 
    \makecell[l]{
        \mintinline{c}{stack[index] = 0;} \\
        \mintinline{c}{index--;} 
    }
    & 
    \makecell[l]{
        \mintinline{c}{stack[index] = 0;} \\
        \mintinline{c}{index--;} 
    } &
    \\ 
    \hline 
\end{tabular}

    
    \item Push pseudo-code:
        \begin{center}
            \begin{algorithm}[H]
                \SetAlgoLined
                % \DontPrintSemiColon
                initialize(stack)\; 
                \If{stack.top $\neq$  stack.size - 1}{
                    display "STACK OVERFLOW"\; 
                    exit push\; 
                }
                stack.top = stack.top + 1; \Comment{Increment by one}\;
                stack[stack.top] = v; \Comment{Assign element at the top position}\;
                end procedure push\; 
                \caption{Procedure Push (v)}
            \end{algorithm}
        \end{center}
        % \begin{lstlisting}
        %     PROCEDURE PUSH (v)
        %     IF TOP := SIZE - 1 THEN 
        %         DISPLAY "STACK OVERFLOW" 
        %         EXIT PUSH 
        %     END IF
        %     TOP := TOP + 1 // Increment by one.
        %     S[TOP] := v // Assign element at the to position of the stack
        %     END PROCEDURE PUSH 
        % \end{lstlisting}
    
    \item Pop pseudo-code:
        \begin{center}
            \begin{algorithm}[H]
                \SetAlgoLined
                \large
                % \SetAlgoNlRelativeSize{1}
                % \DontPrintSemiColon
                \If{stack.top == -1}{
                    display "STACK OVERFLOW"\; 
                    exit pop\; 
                }
                v = stack[stack.top]\;
                stack.top = stack.top - 1\; 
                return v\; 
                end procedure POP\;
                \caption{Procedure pop ()}
            \end{algorithm}
        \end{center}
        % \begin{lstlisting}
        %     PROCEDURE POP() 
        %     IF S.TOP = -1 THEN 
        %         DISPLAY "STACK UNDERFLOW" 
        %         EXIT POP 
        %     END IF 
        %     v := S[TOP] 
        %     TOP := TOP - 1
        %     RETURN v 
        %     END PROCEDURE
        % \end{lstlisting}
\end{itemize}


%----------------------------------------------------------------------------------------
\section{Implementation of stack - Code along}
\begin{itemize}
    \item After including the standard libraries:
        \begin{minted}[autogobble]{c}
            #include <stdio.h>
            #include <stdlib.h>
        \end{minted}
    \item Lets implement a typedef struct with a fixed size member array called item.
        \begin{minted}[autogobble]{c}
            #define SIZE 10
            typedef struct {
                int item[SIZE];
                int top; 
            } Stack; 
        \end{minted}
    
    \item Now the functions pertinent to the stack: push, pop. 
        \begin{minted}[autogobble]{c}
            // Prototype declarations
            void push(Stack *, int );
            int pop(Stack *);
        \end{minted}
        \begin{minted}[autogobble]{c}
            void push(Stack *sp, int value ){
                if (sp->top == SIZE -1){
                    perror("Stack overflow.\n");
                    exit(-1);
                }
                ++sp->top;
                sp->item[sp->top] = value;
            }
            int pop(Stack *sp){
                if (sp->top == -1){
                    printf("Stack underflow.\n");
                    return -9999; // this will be the error code. 
                }
                --sp->top; // don't mind 
                return sp->item[sp->top + 1];
            }
        \end{minted}
    
    \item For functionality we'll implement some utility functions called init, print\_s:
        \begin{itemize}
            \item init(): initializes the top struct stack member to -1.
            \item print\_s(): prints the used members of the stack array (item).
        \end{itemize}
        \begin{minted}[autogobble]{c}
            // protype declarations
            void init(Stack *);
            void print_s(Stack *, const char[]);
        \end{minted}
        \begin{minted}[autogobble]{c}
            void init(Stack *sp){
                sp->top = -1;
            }
            void print_s(Stack *sp, const char str[]){
                printf("%s:\n",str);
                for (int i = 0; i < sp->top + 1; i++) {
                    printf("\t%d\n",sp->item[i]);
                }
            }
        \end{minted}
    
    \item Now we'll implement the main function and call every function that we have declared thus far.
        \begin{minted}[autogobble]{c}
            int main() {
                Stack s1 = {}, s2 = {}; // initialize everything to 0
                init(&s1);
                init(&s2);
                // push to s1
                push(&s1,100);
                push(&s1,200);
                push(&s1,356);
                push(&s1,120);
                // push to s2
                push(&s2,189);
                push(&s2,167);
                push(&s2,156);
                push(&s2,789);

                print_s(&s1,"s1"); // print used elements in s1.item

                // pop operation
                printf("deleted from s1: %d\n\n",pop(&s1));
                printf("deleted from s1: %d\n\n",pop(&s1));
                printf("deleted from s1: %d\n\n",pop(&s1));

                print_s(&s1,"s1"); // print used elements in s1.item

                printf("---------------------------\n");
                
                print_s(&s2,"s2"); // print used elements in s2.item

                // pop operation 
                printf("deleted from s2: %d\n\n",pop(&s2));
                printf("deleted from s2: %d\n\n",pop(&s2));
                printf("deleted from s2: %d\n\n",pop(&s2));
                
                print_s(&s2,"s2"); // print used elements in s1.item
                return 0;
            }
        \end{minted}
    
    \item The results are as follows:
        \begin{minted}[autogobble]{c}
            /* Output: 
            s1:
                100
                200
                356
                120
            deleted from s1: 120

            deleted from s1: 356

            deleted from s1: 200

            s1:
                100
            ---------------------------
            s2:
                189
                167
                156
                789
            deleted from s2: 789

            deleted from s2: 156

            deleted from s2: 167

            s2:
                189

            */
        \end{minted}
\end{itemize}


%----------------------------------------------------------------------------------------
\section{Making a menu for the stack}
\begin{itemize}
    \item To add a menu to the stack implementation previously demonstrated, apart from the libraries already included, include the stdbool.h header file:
        \begin{minted}[autogobble]{c}
            #include <stdio.h>
            #include <stdlib.h>
            #include <stdbool.h>
        \end{minted}
    
    \item The struct Stack definition remains the same, as well as the the functions: print\_s, init, pop, push.
    \item The main function gets a menu added.
        \begin{itemize}
            \item We'll add a switch statement for menu purposes.
        \end{itemize}
        \begin{minted}[autogobble]{c}
            int main() {
                Stack s1 = {}; 
                init(&s1);

                printf("1. Push\n2. Pop\n3. Print Stack\n4. Exit\n\n");
                int choice, value; 

                while (true){
                    printf("Enter choice:");
                    scanf("%d",&choice);
                    switch (choice) {
                    case 1:
                        printf("Enter integer value: ");
                        scanf("%d",&value);
                        push(&s1,value);
                        printf("\n");
                        break;
                    case 2:
                        value = pop(&s1);
                        if (value != -9999){
                            printf("Poped data: %d\n", value);
                        }
                        break; 
                    case 3:
                        print_s(&s1,"s1");
                        break; 
                    case 4: 
                        exit(0);
                    default:
                        printf("Invalid choice.\n");
                        break;
                    }
                }
                return 0;
            }
        \end{minted}
    
    \item An exemplary result is:
        \begin{minted}[autogobble]{c}
            /* Output: 
            1. Push
            2. Pop
            3. Print Stack
            4. Exit

            Enter choice:1
            Enter integer value: 123131

            Enter choice:1
            Enter integer value: 3423

            Enter choice:1
            Enter integer value: 234234

            Enter choice:1
            Enter integer value: 23425

            Enter choice:1
            Enter integer value: 890

            Enter choice:3
            s1:
                [0]: 123131
                [1]: 3423
                [2]: 234234
                [3]: 23425
                [4]: 890
            Enter choice:2
            Poped data: 890
            Enter choice:2
            Poped data: 23425
            Enter choice:2
            Poped data: 234234
            Enter choice:2
            Poped data: 3423
            Enter choice:3
            s1:
                    [0]: 123131
            Enter choice:4

            */
        \end{minted}
\end{itemize}


%----------------------------------------------------------------------------------------
\section{Dynamic memory in the usage of our stack, stack evolution}
\begin{itemize}
    \item For this include the libraries:
        \begin{minted}[autogobble]{c}
            #include <stdio.h>
            #include <stdlib.h>
            #include <stdbool.h>
        \end{minted}

    \item The struct Stack member item is a fixed size array, optimally we would like to allocate memory dynamically for this member. Thus, we re-declare the struct Stack member item as a pointer: 
        \begin{minted}[autogobble]{c}
            typedef struct {
                int *item;
                int top; 
                int size; 
            } Stack; 
        \end{minted}
    
    \item Redefine some functions:
        \begin{minted}[autogobble]{c}
            // Prototype declarations
            void push(Stack *, int );
            int pop(Stack *);
            void init(Stack *, int);
            void print_s(Stack *, const char[]);
            void deallocate(Stack *);
        \end{minted}
        \begin{itemize}
            \item The push function is redefined to include the dynamic memory reallocation in case of stack overflow.
            \item The pop function stays the same.
            \item The init function is changed to include memory allocation using malloc() for the member item, and modify the size accordingly.
            \item The print\_s is changed to include printing of members all the way up to the size of memory allocated for the array.
            \item The deallocate function, frees the space dynamically allocated by the malloc function, also 
        \end{itemize}
        \begin{minted}[autogobble]{c}
            void push(Stack *sp, int value ){
                if (sp->top == sp->size -1){
                    printf("Stack overflow.\n");
                    sp->item = (int*)realloc(sp->item,sizeof(sp->size)*2);
                    if (sp->item == NULL)
                        printf("Error in realloc.\n");
                    sp->size *= 2;
                }
                ++sp->top;
                sp->item[sp->top] = value;
            }

            int pop(Stack *sp){
                if (sp->top == -1){
                    printf("Stack underflow.\n");
                    return -9999;
                }
                --sp->top; // don't mind 
                return sp->item[sp->top + 1];
            }

            void init(Stack *sp, int size){
                sp->top = -1;
                sp->item = (int*) malloc(size * sizeof(int));
                if (sp->item == NULL){
                    printf("Error.\n");
                    exit(1);
                }
                sp->size = size;
            }

            void print_s(Stack *sp, const char str[]){
                printf("%s:\n",str);
                for (int i = 0; i < sp->size; i++) {
                    printf("\t[%d]: %d\n",i,sp->item[i]);
                }
            }

            void deallocate(Stack *sp){
                if (sp->item != NULL)
                    free(sp->item);
                sp->top = -1; 
                sp->size = 0;
            }
        \end{minted}
    
    \item The main function changes to be:
        \begin{itemize}
            \item The exit function includes the deallocate function, and the previously redefined push function reallocates the memory every time a stack overflow occurs.
        \end{itemize}
        \begin{minted}[autogobble]{c}
            int main() {
                Stack s1 = {}; 
                init(&s1,3);

                printf("1. Push\n2. Pop\n3. Print Stack\n4. Exit\n\n");
                int choice, value; 

                while (true){
                    printf("Size of the stack is: %d\n",s1.size);
                    printf("Enter choice:");
                    scanf("%d",&choice);
                    printf("\n");
                    switch (choice) {
                    case 1:
                        printf("Enter integer value: ");
                        scanf("%d",&value);
                        push(&s1,value);
                        printf("\n");
                        break;
                    case 2:
                        value = pop(&s1);
                        if (value != -9999){
                            printf("Poped data: %d\n", value);
                        }
                        break; 
                    case 3:
                        print_s(&s1,"s1");
                        printf("\n");
                        break; 
                    case 4: 
                        deallocate(&s1);
                        exit(0);
                        break;
                    default:
                        printf("Invalid choice.\n");
                        break;
                    }
                }
                return 0;
            }
        \end{minted}
    
    \item All of which prompt the following exemplary results:
        \begin{minted}[autogobble]{c}
            /* Output: 
            1. Push
            2. Pop
            3. Print Stack
            4. Exit

            Size of the stack is: 3
            Enter choice:1

            Enter integer value: 45

            Size of the stack is: 3
            Enter choice:1

            Enter integer value: 56

            Size of the stack is: 3
            Enter choice:1

            Enter integer value: 12

            Size of the stack is: 3
            Enter choice:1

            Enter integer value: 23
            Stack overflow.

            Size of the stack is: 6
            Enter choice:1

            Enter integer value: 25

            Size of the stack is: 6
            Enter choice:1

            Enter integer value: 14

            Size of the stack is: 6
            Enter choice:1

            Enter integer value: 36
            Stack overflow.

            Size of the stack is: 12
            Enter choice:1

            Enter integer value: 17

            Size of the stack is: 12
            Enter choice:1

            Enter integer value: 18

            Size of the stack is: 12
            Enter choice:1

            Enter integer value: 19

            Size of the stack is: 12
            Enter choice:1

            Enter integer value: 20

            Size of the stack is: 12
            Enter choice:3

            s1:
                [0]: 45
                [1]: 56
                [2]: 12
                [3]: 23
                [4]: 25
                [5]: 14
                [6]: 36
                [7]: 17
                [8]: 18
                [9]: 19
                [10]: 20
                [11]: 0

            Size of the stack is: 12
            Enter choice:2

            Poped data: 20
            Size of the stack is: 12
            Enter choice:2

            Poped data: 19
            Size of the stack is: 12
            Enter choice:2

            Poped data: 18
            Size of the stack is: 12
            Enter choice:2

            Poped data: 17
            Size of the stack is: 12
            Enter choice:3
            */
        \end{minted}
\end{itemize}

%----------------------------------------------------------------------------------------
\section{Stack in action - Decimal to binary conversion}
\subsection{Decimal to binary conversion}
\begin{itemize}
    \item Let's say we want to convert 10 to a binary representation. 
        \begin{align*}
            10  \rightarrow 10+\frac{0}{2}:\; & \p{10 \pmod{2}} =  0 \;\text{ LSB }\\ 
            10/2\rightarrow 5+\frac{0}{2}:\;  & \p{5 \pmod{2}} = 1 \\  
            5/2 \rightarrow 2+\frac{1}{2}:\;  & \p{2  \pmod{2}} = 0 \\ 
            2/2 \rightarrow 1+\frac{0}{2}:\;  & \p{1 \pmod{2}} = 1 \;\text{ MSB }\\
        \end{align*}
\end{itemize}

\subsection{Let's implement the function printBinary}
\begin{itemize}
    \item Taking in to account the process observed above. 
        \begin{minted}[autogobble]{c}
            // function prototype
            void printBinary(unsigned int);
        \end{minted}
        \begin{minted}[autogobble]{c}
            void printBinary(unsigned int num){
                Stack s; 
                const int base = 2; 
                const int num_ = num;
                init(&s,10);
                int rem;

                while (num > 0){
                    rem = num % base; 
                    push(&s,rem);
                    num /= base; 
                }
                printf("Binary equivalent of %d is: \n", num_);
                while (s.top != -1){
                    printf("%d", pop(&s));
                }
                deallocate(&s);
            }
        \end{minted}
    
    \item Defining a test main function we have:
        \begin{minted}[autogobble]{c}
            int main() {
                printBinary(10);
                return 0;
            }
        \end{minted}
    
    \item Of which results are:
        \begin{minted}[autogobble]{c}
            /* Output: 
            Binary equivalent of 10 is:
            1010
            */
        \end{minted}
\end{itemize}


%----------------------------------------------------------------------------------------
\section{Stack in action: Reversing the content of a text file}
\begin{itemize}
    \item Data structure needed: stack of characters.
    \item Algorithm:
        \begin{enumerate}
            \item Go on reading characters from source file until End-Of-File is reached.
            \item PUSH each character read into the stack.
            \item When done, POP characters from the stack and write them into the destination file until stack is underflow.
        \end{enumerate}
\end{itemize}

\inputcode{c}{\code/invert_file_cont_stack.c}


