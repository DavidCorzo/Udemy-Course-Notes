\begin{itemize}
    \item When a program needs to store the instructions of it's program and teh data that it act's upon while your computer is executing that program.
    \item This information is stored in RAM.
    \item Harddrives store presistent data, RAM is eliminated when the computer is turned off.
    \item Think of RAM as an ordered sequence of boxes:
        \begin{itemize}
            \item The bow is full when it represents a 1 or the bow is empty when it represents 0.
            \item Eaxh bow represents one binary digit called a bit.
            \item Bits in memory are grouped into sets of eight called a byte.
            \item Each byte has a label, this is a number, this is referenced with this label.
            \item Each byte has a unique reference.
        \end{itemize}
    
    \item A program is only going to be powerful if it can manipulate data.
    \item So we need to understand the different data types you can use.
    \item Constants are types of data that do not change and retain their values throughout the life of the program.
    \item Variables are types of data that may change or be assigned values as the program runs, these values can change, this is the oposite of the constant, variables and constants are stored in memory and referenced using these hexadecimal adresses.
    \item Variable names reference these hexadecimal adresses so that you don't have to memorize them.
    \item You can access these variable by their name.
\end{itemize}
%%%%%%%%%%%%%%%%%%%%%%%%%%%%%%%%%%%%%%%%%%%%%%%%%%%%%%%%%%%%%%%%%%%%%%%%%%%%%%%%%%%%%%%%%%%%%%%%
\subsection{Naming variables}
\begin{itemize}
    \item In C all the names must begin a letter or underscore and what ever follows them.
    \item Valid:
        \begin{Verbatim}[breaklines=true, breakanywhere=true]
            David; myFlag; j5x7; _anotherVariable
        \end{Verbatim}
    
    \item Invalid:
        \begin{Verbatim}[breaklines=true, breakanywhere=true]
            temp$value // has a dollar sign
            my Flag // has a space 
            3jason // starts with number
            int //reserved word
        \end{Verbatim}
    
    \item Use meaningfull names, this is a documentation process as well.
\end{itemize}
%%%%%%%%%%%%%%%%%%%%%%%%%%%%%%%%%%%%%%%%%%%%%%%%%%%%%%%%%%%%%%%%%%%%%%%%%%%%%%%%%%%%%%%%%%%%%%%%
\subsection{Data types}
\begin{itemize}
    \item The computer needs a way to identify what dat type it's processing.
    \item Do this by preceding them by the data type for example ``int myvar''; this also corrsponds to how much momory you need for the type being declared.
    \item Primitive data types , in C everything is a primitive data type, there are no objects in C.
\end{itemize}
%%%%%%%%%%%%%%%%%%%%%%%%%%%%%%%%%%%%%%%%%%%%%%%%%%%%%%%%%%%%%%%%%%%%%%%%%%%%%%%%%%%%%%%%%%%%%%%%
\subsection{Declaring variables}
\begin{itemize}
    \item When you declare the variable you must do this using the format of first the data type (int, float, double, char, etc) and then your variable name.
        \begin{itemize}
            \item \emph{type-specifier variable-name;}
        \end{itemize}
    \item You can declare multiple varibles with comas. 
    \item Variable declaration is just saying what your program has and will use. 
    \item Example:
        \begin{Verbatim}[breaklines=true, breakanywhere=true]
            int x;
            int x,y,z;
        \end{Verbatim}
        C will allocate the memory for x, y and z; but they are null.
\end{itemize}
%%%%%%%%%%%%%%%%%%%%%%%%%%%%%%%%%%%%%%%%%%%%%%%%%%%%%%%%%%%%%%%%%%%%%%%%%%%%%%%%%%%%%%%%%%%%%%%%
\subsection{Initializing variables}
\begin{itemize}
    \item After declaration of the variables you assign a value to that variable:
        \begin{Verbatim}[breaklines=true, breakanywhere=true]
            int x = 0;
            int y = 9, r = 5;
        \end{Verbatim}
\end{itemize}
%%%%%%%%%%%%%%%%%%%%%%%%%%%%%%%%%%%%%%%%%%%%%%%%%%%%%%%%%%%%%%%%%%%%%%%%%%%%%%%%%%%%%%%%%%%%%%%%
\subsection{Example}
\begin{Verbatim}[breaklines=true, breakanywhere=true]
#include <stdio.h>
#include <stdlib.h>

int main()
{
    int david = 90; // declaration and initialization
    
    david = 8; // modification a that variable
    
    return 0;
}

\end{Verbatim}

