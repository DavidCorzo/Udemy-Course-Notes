\section{The congnitive foundation introduction}
\begin{itemize}
    \item Contract example: 
        \begin{itemize}
            \item How to comunicate the same concept.
            \item How to show the idea to convey vasts amounts of knoledge. 
        \end{itemize}
    
    \item Why is this so intuitive? Cognitive research proves these maps are intuitive.
        \begin{itemize}
            \item Knowledge acquisition 
            \item Visual communication 
            \item Creative thinking 
        \end{itemize}
\end{itemize}


%%%%%%%%%%%%%%%%%%%%%%%%%%%%%%%%%%%%%%%%%%%%%%%%%%%%%%%%%%%%%%%%%%%%%%%%%%%%%%%%%%%%%%%%%%
\section{Three memory systems}
\begin{itemize}
    \item The brain has three memory sistem:
        \begin{enumerate}
            \item Sensory information storage: thw subdivisions, we hold information for 0.5 sec, whether we do something with that information determines if this information passes to the working memory.
                \begin{itemize}
                    \item Iconic memory 
                    \item Echoic memory
                \end{itemize}
            
            \item Working memory: Central excecutive, can stored for 10 sec.
            \item Long term memory: three subdivisions, memory can be stored for years, weeks, months, etc.
                \begin{itemize}
                    \item Semantic memory
                    \item Procedure memory 
                    \item Episodic memory 
                \end{itemize}
        \end{enumerate}
\end{itemize}


%%%%%%%%%%%%%%%%%%%%%%%%%%%%%%%%%%%%%%%%%%%%%%%%%%%%%%%%%%%%%%%%%%%%%%%%%%%%%%%%%%%%%%%%%%
\section{Semantic memory}
\begin{itemize}
    \item It organizes knowledge in networks.
    \item Example: 
        \begin{itemize}
            \item Iconic memory, holds for 0.5 sec 
            \item Working memory, process 10 seconds 
            
            \item Long term memory: 
            \begin{itemize}
                \item Semantic memory, encoded in diferent ways, in images or words or other ways. These nodes become conected ith previous memory, this is how it is remembered in the long term.
                \item Episodic memory: holds specific situations I have been through, the semantic memory you ``don't know why you know it you just know it, this is because of the links it does to n already existing memory network'' 
                \item Procedure memory: how you remember procedures, you cant probably say how you do it because it's procedural.
            \end{itemize}
        \end{itemize}
    
    \item How to encode knowledge? there is a process it's not just encoded by the information we see.
    \item Example: 
        \begin{itemize}
            \item Agroup given complex material.
            \item Directly after, thir understanding was tested. 
            \item Sometime later they were probed again.
            \item The ones who got the meaning from the beggining had much better memory when probed again.
            \item The ones who understood this from the very beggining recalled it better.
        \end{itemize}
        \begin{itemize}[label=\#]
            \item This means that weather we store semantic memory:
                \begin{itemize}
                    \item Shallow: poor resistance, mechanical rehearsal, methonds of regurgitations.
                    \item Intermediate: better resistance, Attention to sound and appearance, details method.
                    \item Deep processing: best resistance, focus on semantic meaning, this is a process that links new links from the existing memory to the new.
                \end{itemize}
        \end{itemize}
    
    \item This means that in order for knoledge to stick we need to create new links and to the new memory nodes.  
    \item The best resistance method, deep processing, allows us to fetch with great efectiveness knoledge from the long term memory to the working memory.
\end{itemize}


%%%%%%%%%%%%%%%%%%%%%%%%%%%%%%%%%%%%%%%%%%%%%%%%%%%%%%%%%%%%%%%%%%%%%%%%%%%%%%%%%%%%%%%%%%
\section{Brains like visualizations}
\begin{itemize}
    \item Brains are likely to recall with resistance knoledge when they are shown elements that are integrated, remember th example of the doll siting on a chair holding a flag, rather than the three objects shown separately.
\end{itemize}


%%%%%%%%%%%%%%%%%%%%%%%%%%%%%%%%%%%%%%%%%%%%%%%%%%%%%%%%%%%%%%%%%%%%%%%%%%%%%%%%%%%%%%%%%%
\section{How does creative thinking work}
\begin{itemize}
    \item Where does creativity it come from?
    \begin{itemize}
        \item Understad the problems structure
        \item Activate spreading in the knowledge network.
        \item The more knowledge - more activation 
        \item If knowledge is richly interconnected then ...
        \item ... Creative leaps might occur.
    \end{itemize}
\end{itemize}


%%%%%%%%%%%%%%%%%%%%%%%%%%%%%%%%%%%%%%%%%%%%%%%%%%%%%%%%%%%%%%%%%%%%%%%%%%%%%%%%%%%%%%%%%%
\section{A concept map of the cognitive foundation}
\begin{itemize}
    \item A tool tha t pus focus on meanind in the subject area.
    \item Creates a scaffold for the limited working memory so it does not loose context in teh constant iterations with long term memory.
    \item Utilizes integrated visualization that maps the to+be+learned to already estabished knowledge.
    \item Increases the associative links which increase the possibility for creative ``leaps''
    \item The conceptual map does all these things.
\end{itemize}








%%%%%%%%%%%%%%%%%%%%%%%%%%%%%%%%%%%%%%%%%%%%%%%%%%%%%%%%%%%%%%%%%%%%%%%%%%%%%%%%%%%%%%%%%%
\section{Resources}
\begin{itemize}
    \item Eric Ries: ``The lean startup''
\end{itemize}
