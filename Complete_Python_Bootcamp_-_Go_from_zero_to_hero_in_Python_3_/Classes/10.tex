\section{Advanced Python modules}

\subsection{Collections module}
Count occurances: 
\begin{minted}[autogobble]{python}
    from collections import Counter 
    l = [1,1,1,1,12,2,2,2,2,3,3,3,3,3,4,5,5,5,6]
    Counter(l)
    # output: Counter({3: 5, 1: 4, 2: 4, 5: 3, 12: 1, 4: 1, 6: 1})
    s = "aaaassssssvvvvsssaasassjjjdjjd"
    Counter(s)
    # output: Counter({'s': 12, 'a': 7, 'j': 5, 'v': 4, 'd': 2})
    s = "Hello this is a sentence containing lots of hellos and hello statments hello hello hello"
    Counter(s.split())
    # output: Counter({'hello': 4, 'Hello': 1, 'this': 1, 'is': 1, 'a': 1, 'sentence': 1, 'containing': 1, 'lots': 1, 'of': 1, 'hellos': 1, 'and': 1, 'statments': 1})
    c = Counter(s.split())
    c.most_common(2) # Two most common
    # output: [('hello', 4), ('Hello', 1)]
\end{minted}

\begin{center}
    \begin{tabular}{ |p{8cm}|p{8cm}| }
        \hline
            \verb|sum(c.values())| & total of all counts \\
            \verb|c.clear()| & reset all counts \\ 
            \verb|list(c)| & list unique elements \\ 
            \verb|set(c)| & convert to a set \\ 
            \verb|dict(c)| & convert to dict \\ 
            \verb|c.items()| & convert to a list of (elem,cnt) pairs \\ 
            \verb|Counter(dict(list_of_pairs))| & convert from a list of (elem,cnt) to pairs \\ 
            \verb|c.most_common()[:-n-1:-]| & n least comon elements \\ 
            \verb|c += Counter()| & remove zero and negative counts \\ 
        \hline
    \end{tabular}
\end{center}



%----------------------------------------------------------------------------------------
\section{defaultdict}
\begin{itemize}
    \item defaultdict will never raise a KeyError, any key that isn't found returns the value of the default factory. 
\end{itemize}
\begin{minted}[autogobble]{python}
    from collections import defaultdict
    d = defaultdict(object)
    d['one'] 
    # output: <object object at 0x0000019728D8E1B0>
    d = defaultdict(lambda: 0) # lambda that returns 0
    d['one'] 
    # output: 0
    d['two'] = 2
    d 
    # output: defaultdict(<function <lambda> at 0x0000019728DDC1E0>, {'one': 0, 'two': 2})
\end{minted}


%----------------------------------------------------------------------------------------
\section{OrderedDict}
\begin{itemize}
    \item This data structure considers the order in which elements were added to a list. 
\end{itemize}
\begin{minted}[autogobble]{python}
    d = {}
    d['a'] = 1
    d['b'] = 2
    d['c'] = 3
    d['d'] = 4
    d['e'] = 5
    d 
    # output: {'a': 1, 'b': 2, 'c': 3, 'd': 4, 'e': 5}

    for k,v in d.items():
        print(k,v)
    # output: a 1
    # output: c 3
    # output: e 5
    # output: b 2
    # output: d 4

    from collections import OrderedDict
    d = OrderedDict()
    d['a'] = 1
    d['b'] = 2
    d['c'] = 3
    d['d'] = 4
    d['e'] = 5
    d 
    # output: a 1
    # output: b 2
    # output: c 3
    # output: d 4
    # output: e 5

    d1 = {}
    d1['a'] = 1
    d1['b'] = 2

    d2 = {}
    d2['a'] = 2
    d2['b'] = 1

    print(d1 == d2)
    # output: True

    d1 = OrderedDict()
    d1['a'] = 1
    d1['b'] = 2

    d2 = OrderedDict()
    d2['b'] = 2
    d2['a'] = 1

    print(d1 == d2)
    # output: False 
\end{minted}


%----------------------------------------------------------------------------------------
\section{namedtuple}
\begin{itemize}
    \item This allows to create a class on one line. 
\end{itemize}
\begin{minted}[autogobble]{python}
    from collections import namedtuple
    Dog = namedtuple('Dog','age breed name')
    sam = Dog(age=2,breed='Lab',name='Sammy')
    sam.age
    # output: 2
    sam[0]
    # output: 2

    Cat = namedtuple('Cat','fur claws name')
    c = Cat(fur='Fuzzy',claws=False,name='Kitty')
    c.name
    # output: 'Kitty'
    c[2]
    # output: 'Kitty'
\end{minted}


%----------------------------------------------------------------------------------------
\section{Datetime}
\begin{minted}[autogobble]{python}
    import datetime 
    t = datetime.time(5,25,1)
    print(t)
    t.hour
    # output: 5
    t.minute
    # output: 25
    t.second
    # output: 1
    print(datetime.time.min)
    # output: 00:00:00
    print(datetime.time.max)
    # output: 23:59:59.999999
    print(datetime.time.resolution)
    # output: 0:00:00.000001

    today = datetime.date.today() 
    print(today)
    # output: 2020-05-25
    today.timetuple() 
    # output: time.struct_time(tm_year=2020, tm_mon=5, tm_mday=25, tm_hour=0, tm_min=0, tm_sec=0, tm_wday=0, tm_yday=146, tm_isdst=-1)
    today.year
    # output: 2020
    today.month
    # output: 4
    today.day
    # output: 25
    print(datetime.date.min)
    # output: 0001-01-01
    print(datetime.date.max)
    # output: 9999-12-31
    print(datetime.date.resolution)
    # output: 1 day, 0:00:00

    d1 = datetime.date(2015,3,11)
    print(d1)
    # output: 2015-03-11
    d2 = d1.replace(year=1900)
    print(d2) 
    # output: datetime.date(1900, 3, 11)
    d1-d2 
    # output: datetime.timedelta(days=42003)
\end{minted}

%----------------------------------------------------------------------------------------
\section{Python debugger}
\begin{itemize}
    \item python debugger is called pdb 
    \item The pbd module allows for an interactive debugging environment after the excecution of \verb|pbd.set_trace()|
\end{itemize}
\begin{minted}[autogobble]{python}
    import pdb 
    x = [1,3,4]
    y = 2 
    z = 3 
    result = y + z 
    print(result)

    pdb.set_trace()

    result2 = y + x 
    print(result2)
    # output: import pdb 
    x = [1,3,4]
    y = 2 
    z = 3 
    result = y + z 
    print(result)

    pdb.set_trace()

    result2 = y + x 
    print(result2)
    # output: -> result2 = y + x
    # output: (Pdb) y
    # output: 2     
    # output: (Pdb) x
    # output: [1, 3, 4]
    # output: (Pdb) y+x
    # output: *** TypeError: unsupported operand type(s) for +: 'int' # output: and 'list'
    # output: (Pdb) x+z
    # output: *** TypeError: can only concatenate list (not "int") to # output: list
    # output: (Pdb) y+x
    # output: *** TypeError: unsupported operand type(s) for +: 'int' # output: and 'list'
    # output: (Pdb) y+z
    # output: 5
    # output: (Pdb) q
\end{minted}

%----------------------------------------------------------------------------------------
\section{Timing your code}
\begin{itemize}
    \item timeit 
\end{itemize}
\begin{minted}[autogobble]{python}
    import timeit 
    timeit.timeit('"-".join(str(n) for n in range(100))',number=1_000)
    # output: 0.018360799999999955
    timeit.timeit('"-".join([str(n) for n in range(100)])',number=10_000)
    # output: 0.16509299999999993
    timeit.timeit("-".join(map(str,range(100))),number=10_000)
    # output: 0.00012020000000001474
\end{minted}

%----------------------------------------------------------------------------------------
\section{Regular expressions}
\begin{minted}[autogobble]{python}
    import re 
    patterns = ["term1","term2"]
    text = 'This is a string with term1, but not the other term'
    re.search('hello','hello world!')
    # output: <re.Match object; span=(0, 5), match='hello'>
    match = re.search(patterns[0],text)
    type(match)
    # output: <class 're.Match'>
    match.start()
    # output: 22
    match.end()
    # output: 27
    split_term = '@'
    phrase = 'What is your email, is it hello@gmail.com'
    re.split(split_term,phrase)
    # output: ['What is your email, is it hello', 'gmail.com']
    'hello world'.split()
    # output: ['hello', 'world']
    re.findall(pattern="match",string="Here is one match, here is another match")
    # output: ['match', 'match']
\end{minted}

\begin{itemize}
    \item Repetition syntax: 
        \begin{minted}[autogobble]{python}
            def multi_re_find(patterns,phrase):
                for pattern in patterns: 
                    print(f"Searching the phrase using the re check: {pattern}")
                    print(re.findall(pattern,phrase))

            test_phrase = 'sdsd..sssddd...sdddsddd...dsds...dsssss...sdddd'
            test_patterns = [
                'sd*', # s followed by zero or more d's
                'sd+', # s followed by one or more d's
                'sd?', # s followed by zero or one d's
                'sd{3}', # s followed by three d's 
                'sd{2,3}' # s followed by two to three d's
            ]
            multi_re_find(test_patterns,test_phrase)

            # output: Searching the phrase using the re check: sd*
            # output: ['sd', 'sd', 's', 's', 'sddd', 'sddd', 'sddd', 'sd', 's',  's', 's', 's', 's', 's', 'sdddd']
            # output: Searching the phrase using the re check: sd+
            # output: ['sd', 'sd', 'sddd', 'sddd', 'sddd', 'sd', 'sdddd']
            # output: Searching the phrase using the re check: sd?
            # output: ['sd', 'sd', 's', 's', 'sd', 'sd', 'sd', 'sd', 's', 's',  's', 's', 's', 's', 'sd']
            # output: Searching the phrase using the re check: sd{3}
            # output: ['sddd', 'sddd', 'sddd', 'sddd']
            # output: Searching the phrase using the re check: sd{2,3}
            # output: ['sddd', 'sddd', 'sddd', 'sddd']
        \end{minted}
    
    \item Character sets: 
        \begin{minted}[autogobble]{python}
            test_phrase = 'sdsd..sssddd...sdddsddd...dsds...dsssss...sdddd'
            test_patterns = [
                '[sd]', # either s or d
                's[sd]+' # s followed by one or more s or d
            ]
            multi_re_find(test_patterns,test_phrase)
            # output: Searching the phrase using the re check: [sd]
            # output: ['s', 'd', 's', 'd', 's', 's', 's', 'd', 'd', 'd', 's', 'd', 'd', 'd', 's', 'd', 'd', 'd', 'd', 's', 'd', 's', 'd', 's', 's', 's', 's', 's', 's', 'd', 'd', 'd', 'd']
            # output: Searching the phrase using the re check: s[sd]+
            # output: ['sdsd', 'sssddd', 'sdddsddd', 'sds', 'sssss', 'sdddd']
        \end{minted}
    
    \item Exclusion: 
        \begin{minted}[autogobble]{python}
            test_phrase = 'This is a string! But it has punctuation. How can we remove it?'
            re.findall('[^!.? ]+',test_phrase)
            # output: ['This', 'is', 'a', 'string', 'But', 'it', 'has', 'punctuation', 'How', 'can', 'we', 'remove', 'it']
        \end{minted}
    
    \item Character ranges: 
        \begin{minted}[autogobble]{python}
            test_phrase = 'This is an example sentence. Lets see if we can find some letters.'
            test_patterns = [
                '[a-z]+', # sequences of lower case letters
                '[A-Z]+', # sequences of upper case letters
                '[a-zA-Z]+', # sequences of lower or upper case letters 
                '[A-Z][a-z]+' # one upper case letter followed by lower case letters.
            ]
            multi_re_find(test_patterns,test_phrase)
            # output: Searching the phrase using the re check: [a-z]+
            # output: ['his', 'is', 'an', 'example', 'sentence', 'ets', 'see', 'if', 'we', 'can', 'find', 'some', 'letters']
            # output: Searching the phrase using the re check: [A-Z]+
            # output: ['T', 'L']
            # output: Searching the phrase using the re check: [a-zA-Z]+
            # output: ['This', 'is', 'an', 'example', 'sentence', 'Lets', 'see', 'if', 'we', 'can', 'find', 'some', 'letters']
            # output: Searching the phrase using the re check: [A-Z][a-z]+
            # output: ['This', 'Lets']
        \end{minted}
    
    \item Escape Codes: 
        \begin{center}
            \begin{tabular}{ |p{5cm}|p{8cm}| }
                \hline
                    \verb|\d| & a digit \\
                    \verb|\D| & a non-digit \\
                    \verb|\s| & whitespace (tab,space,newline,etc.) \\
                    \verb|\S| & non-whitespace \\
                    \verb|\w| & alphanumeric \\
                    \verb|\W| & non-alphanumeric \\
                \hline
            \end{tabular}
        \end{center}
        \begin{minted}[autogobble]{python}
            test_phrase = 'This is a string with some numbers 1233 and a symbol #hashtag'
            test_patterns = [
                r'\d+', # sequense of digits
                r'\D+', # sequense of non-digits
                r'\s+', # sequense of whitespace
                r'\S+', # sequense of characters
                r'\w+', # alphanumeric characters
                r'\W+' # non-alphanumeric
            ]
            multi_re_find(test_patterns,test_phrase)
            # output: Searching the phrase using the re check: \d+
            # output: ['1233']
            # output: Searching the phrase using the re check: \D+
            # output: ['This is a string with some numbers ', ' and a symbol #hashtag']
            # output: Searching the phrase using the re check: \s+
            # output: [' ', ' ', ' ', ' ', ' ', ' ', ' ', ' ', ' ', ' ', ' ']
            # output: Searching the phrase using the re check: \S+
            # output: ['This', 'is', 'a', 'string', 'with', 'some', 'numbers', '1233', 'and', 'a', 'symbol', '#hashtag']
            # output: Searching the phrase using the re check: \w+
            # output: ['This', 'is', 'a', 'string', 'with', 'some', 'numbers', '1233', 'and', 'a', 'symbol', 'hashtag']
            # output: Searching the phrase using the re check: \W+
            # output: [' ', ' ', ' ', ' ', ' ', ' ', ' ', ' ', ' ', ' ', ' #']
        \end{minted}
\end{itemize}


%----------------------------------------------------------------------------------------
\section{StringIO}
\begin{itemize}
    \item Allows us to treat a string as a file. 
\end{itemize}
\begin{minted}[autogobble]{python}
    import StringIO
    message = 'This is just a normal string'
    f = StringIO.StringIO(message)
    f.read()
    f.write(' Second')
    f.seek(0)
\end{minted}
