    
\begin{itemize}
    \item Insertion at REAR.
        \begin{itemize}
            \item Inserting at rear and deleting from rear are exactly the same as the push and pop operations we had in the stack.
            \item Assume:
                \begin{itemize}
                    \item REAR = -1
                    \item FRONT = 0
                \end{itemize}
        \end{itemize}
        \begin{center}
            \begin{tabular}{ |p{6cm}|p{6cm}| }

                \hline
                \vspace{0.1cm}
                \begin{tabular}{ cc }
                    {}&{4} \\ {}&{3} \\ {}&{2} \\ {}&{1} \\ {F$\rightarrow$}&{0} \\ {}&{}\\
                \end{tabular}
                \begin{tabular}{|p{0.75cm}|}
                    \hline {} \\ \hline {} \\ \hline {} \\ \hline {} \\ \hline {} \\ \hline \multicolumn{1}{c}{} \\
                \end{tabular}
                \begin{tabular}{ c }
                    {} \\ {} \\ {} \\ {} \\ {} \\ {$\leftarrow$R=-1} \\ 
                \end{tabular}
                {
                    \begin{itemize}
                        \item The DEQueue is empty and the rear is in -1 and front is in 0.
                    \end{itemize}
                }
                & 
                \vspace{0.1cm}
                \begin{tabular}{ cc }
                    {}&{4} \\ {}&{3} \\ {}&{2} \\ {}&{1} \\ {F$\rightarrow$}&{0} \\ {}&{}\\
                \end{tabular}
                \begin{tabular}{|p{0.75cm}|}
                    \hline {} \\ \hline {} \\ \hline {} \\ \hline {} \\ \hline {12} \\ \hline \multicolumn{1}{c}{} \\
                \end{tabular}
                \begin{tabular}{ c }
                    {} \\ {} \\ {} \\ {} \\ {$\leftarrow$R=0} \\ {} \\ 
                \end{tabular}
                {
                    \begin{minted}[autogobble]{c}
                        DEQueue.insertAtRear(12);
                    \end{minted}
                }

                \\ \hline
                \vspace{0.1cm}
                \begin{tabular}{ cc }
                    {}&{4} \\ {}&{3} \\ {}&{2} \\ {}&{1} \\ {F$\rightarrow$}&{0} \\ {}&{}\\
                \end{tabular}
                \begin{tabular}{|p{0.75cm}|}
                    \hline {} \\ \hline {} \\ \hline {} \\ \hline {} \\ \hline {12} \\ \hline \multicolumn{1}{c}{} \\
                \end{tabular}
                \begin{tabular}{ c }
                    {} \\ {} \\ {} \\ {} \\ {$\leftarrow$R=0} \\ {} \\ 
                \end{tabular}

                & 
                \vspace{0.1cm}
                \begin{tabular}{ cc }
                    {}&{4} \\ {}&{3} \\ {}&{2} \\ {}&{1} \\ {F$\rightarrow$}&{0} \\ {}&{}\\
                \end{tabular}
                \begin{tabular}{|p{0.75cm}|}
                    \hline {} \\ \hline {} \\ \hline {} \\ \hline {14} \\ \hline {12} \\ \hline \multicolumn{1}{c}{} \\
                \end{tabular}
                \begin{tabular}{ c }
                    {} \\ {} \\ {} \\ {$\leftarrow$R=1} \\ {} \\ {} \\ 
                \end{tabular}
                {
                    \begin{minted}[autogobble]{c}
                        DEQueue.insertAtRear(14);
                    \end{minted}
                }       
                         
                \\ \hline

                \vspace{0.1cm}
                \begin{tabular}{ cc }
                    {}&{4} \\ {}&{3} \\ {}&{2} \\ {}&{1} \\ {F$\rightarrow$}&{0} \\ {}&{}\\
                \end{tabular}
                \begin{tabular}{|p{0.75cm}|}
                    \hline {} \\ \hline {} \\ \hline {15} \\ \hline {14} \\ \hline {12} \\ \hline \multicolumn{1}{c}{} \\
                \end{tabular}
                \begin{tabular}{ c }
                    {} \\ {} \\ {$\leftarrow$R=2} \\ {} \\ {} \\ {} \\ 
                \end{tabular}
                {
                    \begin{minted}[autogobble]{c}
                        DEQueue.insertAtRear(15);
                    \end{minted}
                }
                & 
                \vspace{0.1cm}
                \begin{tabular}{ cc }
                    {}&{4} \\ {}&{3} \\ {}&{2} \\ {}&{1} \\ {F$\rightarrow$}&{0} \\ {}&{}\\
                \end{tabular}
                \begin{tabular}{|p{0.75cm}|}
                    \hline {} \\ \hline {} \\ \hline {} \\ \hline {14} \\ \hline {12} \\ \hline \multicolumn{1}{c}{} \\
                \end{tabular}
                \begin{tabular}{ c }
                    {} \\ {} \\ {} \\ {$\leftarrow$R=1} \\ {} \\ {} \\ 
                \end{tabular}
                {
                    \begin{minted}[autogobble]{c}
                        DEQueue.deleteAtRear();
                    \end{minted}
                }
                \\ \hline
                
            \end{tabular}
        \end{center}
        \begin{itemize}
            \item The overflow condition for the insert at REAR is going to be when R=4.
        \end{itemize}
    
    \item Insertion at FRONT: when we have inserted with rear elements, and we have a situation in which the front is at index 0 and the rear is in index 2, when the rear is higher in terms of index than the front the insertion at FRONT will not be possible, this is the overflow condition.
    \item Perform deletions in a double ended queue, using the deletion from the FRONT.
        \begin{center}
            \begin{tabular}{ |p{6cm}|p{6cm}| }

                \hline
                \vspace{0.1cm}
                \begin{tabular}{ cc }
                    {}&{4} \\ {}&{3} \\ {}&{2} \\ {F$\rightarrow$}&{1} \\ {}&{0} \\ {}&{}\\
                \end{tabular}
                \begin{tabular}{|p{0.75cm}|}
                    \hline {} \\ \hline {} \\ \hline {} \\ \hline {14} \\ \hline {} \\ \hline \multicolumn{1}{c}{} \\
                \end{tabular}
                \begin{tabular}{ c }
                    {} \\ {} \\ {} \\ {$\leftarrow$R=1} \\ {} \\ {} \\ 
                \end{tabular}
                {
                    \begin{minted}[autogobble]{c}
                        DEQueue.deleteFromFront();
                        // deletes 12 and increments front by one
                    \end{minted}
                }
                & 
                \vspace{0.1cm}
                \begin{tabular}{ cc }
                    {}&{4} \\ {}&{3} \\ {F$\rightarrow$}&{2} \\ {}&{1} \\ {}&{0} \\ {}&{}\\
                \end{tabular}
                \begin{tabular}{|p{0.75cm}|}
                    \hline {} \\ \hline {} \\ \hline {} \\ \hline {} \\ \hline {} \\ \hline \multicolumn{1}{c}{} \\
                \end{tabular}
                \begin{tabular}{ c }
                    {} \\ {} \\ {} \\ {$\leftarrow$R=1} \\ {} \\ {} \\ 
                \end{tabular}
                {
                    \begin{minted}[autogobble]{c}
                        DEQueue.deleteFromFront();
                        // deletes 14 and increments front by one
                    \end{minted}
                }
                \\ \hline
            \end{tabular}
            \begin{itemize}
                \item Note that the queue is empty, we can not perform any more deletion.
                \item The underflow state happens when the front is greater than the rear. 
            \end{itemize}
        \end{center}
    
    \item Now lets try an insertion from the front.
        \begin{center}
            \begin{tabular}{ |p{6cm}|p{6cm}| }

                \hline
                \vspace{0.1cm}
                \begin{tabular}{ cc }
                    {}&{4} \\ {}&{3} \\ {}&{2} \\ {F$\rightarrow$}&{1} \\ {}&{0} \\ {}&{}\\
                \end{tabular}
                \begin{tabular}{|p{0.75cm}|}
                    \hline {} \\ \hline {} \\ \hline {} \\ \hline {10} \\ \hline {} \\ \hline \multicolumn{1}{c}{} \\
                \end{tabular}
                \begin{tabular}{ c }
                    {} \\ {} \\ {} \\ {$\leftarrow$R=1} \\ {} \\ {} \\ 
                \end{tabular}
                {
                    \begin{minted}[autogobble]{c}
                        DEQueue.insertAtFront(10);
                        // decrese the front by one and adds 10
                    \end{minted}
                }
                & 
                \vspace{0.1cm}
                \begin{tabular}{ cc }
                    {}&{4} \\ {}&{3} \\ {}&{2} \\ {}&{1} \\ {F$\rightarrow$}&{0} \\ {}&{}\\
                \end{tabular}
                \begin{tabular}{|p{0.75cm}|}
                    \hline {} \\ \hline {} \\ \hline {} \\ \hline {10} \\ \hline {20} \\ \hline \multicolumn{1}{c}{} \\
                \end{tabular}
                \begin{tabular}{ c }
                    {} \\ {} \\ {} \\ {$\leftarrow$R=1} \\ {} \\ {} \\ 
                \end{tabular}
                {
                    \begin{minted}[autogobble]{c}
                        DEQueue.insertAtFront(20);
                        // decrese the front by one and adds 20
                    \end{minted}
                }
                \\ \hline
                {
                    \begin{itemize}
                        \item Let's say we want to insert in the rear, this will be the procedure.
                    \end{itemize}
                }
                \vspace{0.1cm}
                \begin{tabular}{ cc }
                    {}&{4} \\ {}&{3} \\ {}&{2} \\ {}&{1} \\ {F$\rightarrow$}&{0} \\ {}&{}\\
                \end{tabular}
                \begin{tabular}{|p{0.75cm}|}
                    \hline {} \\ \hline {} \\ \hline {30} \\ \hline {10} \\ \hline {20} \\ \hline \multicolumn{1}{c}{} \\
                \end{tabular}
                \begin{tabular}{ c }
                    {} \\ {} \\ {$\leftarrow$R=2} \\ {} \\ {} \\ {} \\ 
                \end{tabular}
                {
                    \begin{minted}[autogobble]{c}
                        DEQueue.insertAtRear(30);
                        // increases rear by one and adds 30
                    \end{minted}
                }
                &
                \\ \hline
            \end{tabular}
        \end{center}
\end{itemize}
