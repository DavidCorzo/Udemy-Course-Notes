\section{Fundamentals of a program}
\begin{itemize}
    \item Computers are told instructions in the form of algorithms, when a computer program is executed the CPU computes the instructions of the algorithm. 
\end{itemize}

\subsection{Terminology}
\begin{itemize}
    \item CPU: central processing unit, instructions are executed here. 
    \item RAM: random access memory, stores data while it is running; RAM != hard drive.
    \item Hard drive: permanent storage, stores files, program source code, even while the computer is turned off.
    \item  Operating system: program that controls all the operations of the computer, it knows how to interact with the hardware, manages computer resources, executes programs. Windows, Unix, android are all OS.
    \item fetch / execute cycle (life of a CPU): fetches an instruction from memory (using registers) and executes, this is a loop. Gigahertz are the unit of measurement of which this process referenced, means CPU will fetch and execute one billion times a second. 
\end{itemize}

\subsection{Higher level programming language}
\begin{itemize}
    \item A high level language is the opposite of an assembly language.
    \item C is a higher level language. 
    \item A compiler translates higher level code in to machine code. Compilers also checks for syntax rules. 
\end{itemize}

\subsection{Writing a program}
\begin{itemize}[label=$\downarrow$]
   \item 1. Define a program objectives: understand the requirements.
   \item 2. Design the program: how are you going to do it.
   \item 3. Write the code: using the programming language write it. 
   \item 4. Compile 
   \item 5. Run the program: executed 
   \item 6. Test and debug the program: just because it's running it doesn't mean it works.
   \item 7. Maintain and modify the program: sometimes the most expensive step. 
\end{itemize}
\vspace{1cm}
\begin{itemize}
    \item The larger the program the larger the planning. 
    \item Divide and conquer: small steps and constantly test you code. 
\end{itemize}



%----------------------------------------------------------------------------------------
\section{Overview}
\begin{itemize}
    \item C is a general purpose imperative computer programming languages that supports structures programming. 
    \item Uses statements that change a program's state. 
    \item C is the preferred language for producing word processing, spreadsheet programs and compilers; microprocessors, DVD's, cameras all use C. 
    \item C++ is a subset of C, if you learn C you also learn C++.
    \item Provides low-level access to memory, has low level capacities. 
    \item Was invented by Dennis Ritchie of Bell Laboratories, he was working on the design of the UNIX operating system. 
    \item Main goal is to be a useful language, not having to write 10 lines to add two numbers.
    \item C evolved from a previous programming language named B, in B all data items occupied one ``word'' in memory, it was a ``typeless'' language. 
    \item C is available for most computers. C is also hardware independent. 
    \item 1970 evolution of C to ``traditional C'', versions of C where: C90,C89,C99 standardization is important. The current standard is C11.
        \begin{itemize}
            \item C89: most C code is based on C89.
            \item C99: refines and expands the capabilities of C. 
        \end{itemize}
\end{itemize}


%----------------------------------------------------------------------------------------
\section{Language features}
\begin{itemize}
    \item Efficiency and portability: C is efficient because of it's compact and fast customizable. 
    \item C is the leader on portability. 
    \item C was used to create FORTRAN, Perl, Python, Pascal, LISP, Logo and Basic. 
    \item C programs have been used for solving physics and engineering problems and even for animating effects for movies. 
    \item C is flexible: flexibility can cause more bugs and others. 
    \item C fulfills the needs of programmers: 
        \begin{itemize}
            \item Gives you access to hardware. 
            \item Enables you to manipulate individual bits in memory. 
        \end{itemize}
    
    \item C contains a large selection of operators. 
    \item C gives you more freedom, but it also puts more responsibility on you. 
    \item C implementations have a large library of useful C functions. 
    \item Programs can be manipulated at a bit level. 
\end{itemize}
\textbf{Disadvantages}:
\begin{itemize}
    \item Flexibility and freedom also added responsibility. 
\end{itemize}


%----------------------------------------------------------------------------------------
\section{Creating a C program}
\begin{itemize}
    \item Editing: 
        \begin{itemize}
            \item writing the code.
        \end{itemize}
    \item Compiling:
        \begin{itemize}
            \item converts your source code into machine language. 
            \item Two phases: preprocessing phase and the compilation stage. 
            \item Compiler examines each program statement and checks for syntax and semantic errors. 
            \item The compiler will take each statement of the program and translate it into assembly language, then translates the assembly language statements in to machine code. 
            \item The output from the compiler is known as object code and it is stored with file extensions .o or .obj called object files.
            \item Compiling: \verb|gcc -c myprog.c| if you omit the \verb|-c| flag the program will automatically link as well. 
        \end{itemize}
    \item Linking: is just getting all the dependencies in place so that one program is turned into one executable. 
        \begin{itemize}
            \item A linker combines the object modules generated by the compiler with the additional libraries needed by the program to create the whole executable.
            \item Linking errors occur when in this linking phase a .o dependency is not found. 
        \end{itemize}
    \item Executing:
        \begin{itemize}
            \item Results of the program are displayed on a window called console. 
        \end{itemize}
\end{itemize}


