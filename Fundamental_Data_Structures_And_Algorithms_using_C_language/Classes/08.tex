\section{Introduction}
\begin{itemize}
    \item In a circular linked lists we have pretty much the same situation as a singly linked list, the difference is that we have only one pointer to the list, the tail, and the tail.next does not point to NULL, instead it points to the first element of the list. 
    \item We only keep track of the tail, considering the tail.next is the first node of the list.
    \item There is only one terminal node, no head, the tail will tell us everything we need to know. 
    \item The best use for a circular linked list is to build a circular queue. 
\end{itemize}

\subsection{Circular linked list visualization}
A populated circular linked list.
\begin{figure}[H]
    \centering
    \includegraphics[
        % width=\textwidth, height={\textheight-1.2cm}, keepaspectratio
    ]{\figs/cll} 
\end{figure}
When there is only one node, the list looks like this.
\begin{figure}[H]
    \centering
    \includegraphics[
        % width=\textwidth, height={\textheight-1.2cm}, keepaspectratio
    ]{\figs/cllone} 
\end{figure}
\begin{itemize}
    \item When the list is empty the first node inserted will have a .next pointing to itself.
\end{itemize}

\section{Operation for inserting a node to circular linked list}
\begin{figure}[H]
    \centering
    \includegraphics[
    width=\textwidth, height={\textheight-1.2cm}, keepaspectratio
                    ]{\figs/insertnodecll}
\end{figure}

\section{Operation for finding a target node in circular linked list}


\section{Operation for deleting a node in circular linked list}


\section{Operation for printing nodes in circular linked list}  

