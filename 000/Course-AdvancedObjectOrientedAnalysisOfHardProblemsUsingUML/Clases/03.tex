\section{Section overview}
\begin{itemize}
    \item From brute facts to social facts.
        \begin{center}
        \begin{tabular}{ | p{5cm} | p{5cm} | p{5cm} | }
            \hline
                Exists independently to us & Existis since we use and treat as such & Exist since we have jointly declared the to do so     \\
            \hline
                \begin{itemize}
                    \item Bacteria 
                    \item Sun 
                    \item Tree 
                    \item Mountain 
                \end{itemize}
                & 
                \begin{itemize}
                    \item Hammer 
                    \item Table 
                    \item Computer Screwdriver
                \end{itemize}
                & \begin{itemize}
                    \item University
                    \item Birthday party 
                    \item Loan 
                    \item Insurance
                \end{itemize}
                \\ 
                \hline
        \end{tabular}
        \end{center}
    
    \item Example dollar bill:
        \begin{itemize}
            \item Declaration: this note is legal tender to all debts public and private.
        \end{itemize}
    
    \item An assertion exits by it's self, a declarations exists from an assertion, from an accepted declarations we create social facts. (confusing)
    
    \item We are creating evergrowing sistems of concepts, this is how we create the reality concept.    

    \item Concepts triangle.
\end{itemize}



%%%%%%%%%%%%%%%%%%%%%%%%%%%%%%%%%%%%%%%%%%%%%%%%%%%%%%%%%%%%%%%%%%%%%%%%%%%%%%%%%%%%%%%%%%
\section{What happens when we visualize concepts together?}
\begin{itemize}
    \item It's useful for knowledge transmition.
    \item Knowledge is usually heterogenic.
    \item Some people have the power to emit declarations, others don't. Some people have lacking powers to assert their beliefs.
    \item If the normal person says war then it's just an asertion, if the president says that then it is war declaration.
    \item Visualizing concepts makes it easier to:
        \begin{enumerate}
            \item Transfer knowledge 
            \item Spot knowledge differences 
            \item Acquire knowledge 
            \item Makes it clear which declarations are made
            \item Visualizing concepts makes it easier to develop complex systems of any kind
            \item Facilitates creative thinking 
        \end{enumerate}
\end{itemize}



%%%%%%%%%%%%%%%%%%%%%%%%%%%%%%%%%%%%%%%%%%%%%%%%%%%%%%%%%%%%%%%%%%%%%%%%%%%%%%%%%%%%%%%%%%
\section{Philosophical foundations overview analysis}
\begin{itemize}
    \item This is an ongoing debate. 
    \item Section structure:
        \begin{itemize}
            \item Ogden's triangle 
            \item We know objects, we have concepts in our mind, then we use language to refer to both the object and the concept.
        \end{itemize}
\end{itemize}


%%%%%%%%%%%%%%%%%%%%%%%%%%%%%%%%%%%%%%%%%%%%%%%%%%%%%%%%%%%%%%%%%%%%%%%%%%%%%%%%%%%%%%%%%%
\section{Four thories about concepts - First corner of the triangle}
\begin{itemize}
    \item Why is the concept ``concept'' so hard to define:
        \begin{itemize}
            \item Concepts are the constituents of thoughts.
        \end{itemize}
    
    \item Four theories:
        \begin{itemize}
            \item The clasical theory 
            \item The prototype theory 
            \item The theory theory 
            \item The conceptual atomism theory 
        \end{itemize}
\end{itemize}



%%%%%%%%%%%%%%%%%%%%%%%%%%%%%%%%%%%%%%%%%%%%%%%%%%%%%%%%%%%%%%%%%%%%%%%%%%%%%%%%%%%%%%%%%%
\section{The clasical theory of concepts}
\begin{itemize}
    \item It's been around for 2,000 to 2,500 years.
    \item A concept C has definitional structure in that it is composed of:
        \begin{itemize}
            \item Simpler concepts that express 
            \item Necesary and 
            \item Sufficient conditions for 
        \end{itemize}
    
    \item The man, unmarried, bachelor concept.
        \begin{itemize}
            \item Bachelor: unmarried man 
            \item Man and unmaried is a bachelor.
        \end{itemize}
    
    \item Categorization: A psycological process where a concept is matched to an item if,
        \begin{itemize}
            \item Each and every one of the concept's definitional constituents 
            \item Aplies to the target 
        \end{itemize}
    
    \item Example: set of people, some man some women, some are married, the not maried and man are bachelors.
\end{itemize}

%----------------------------------------------------------------------------------------
\subsection{Problems of the conceptual view:}
\begin{itemize}
    \item Plato's problem: 
        \begin{enumerate}
            \item Even the simples concepts are hard to define clearly. 
            \item Easy to spot counter examples.
        \end{enumerate}
        \begin{itemize}
            \item Paint: to paint or not to paint, defined ``x covers y with paint'', does this provide sufficient condition for something falling under a concept? 
            \item First try: ``A paint factory (x) explodes and covers spectators (y) with paint.'' Has x painted y?
            \item Lets add X needs to be an agent and Y needs to be a surface.
            \item Second try: ``you x accidentally kicks paint over your (y) shoes''
            \item Lets add x needs to paint intentionally.
            \item Third try: ``Michaelangelo wasn't painting the celing of the sistine chapel, ha was painting a picture on the ceiling'', the painting unintentonally painted the ceiling.
            \item Let's add that the primary intention should be to paint it.
            \item Fourth try: ``Michaelangelo does have a primary intention to dip the tip of the paintbrush in the pain when painting the picture''
            \item Has Michaelangelo painted the paint brush?
        \end{itemize}
    
    \item These are counter examples.
    \item The tipicality effect: We rank items with respect to how ``tipical they are'' as members of a category. This is problematic, considering this  concept undefines the clasical theory definition.
    \item The clasical theory definition is binary.
\end{itemize}



%%%%%%%%%%%%%%%%%%%%%%%%%%%%%%%%%%%%%%%%%%%%%%%%%%%%%%%%%%%%%%%%%%%%%%%%%%%%%%%%%%%%%%%%%%
\section{Prototype theory of concepts}
\begin{itemize}
    \item The prototype thoery states that a concept C doesn't have a definitional structure but has probabilistic structure in that:
        \begin{enumerate}
            \item Something falls under C 
            \item Just in case it satisfies a 
            \item Suficient number of properties 
            \item Encoded by C's costituents.
        \end{enumerate}
\end{itemize}

\subsection{Witchestein}        
\begin{itemize}
    \item Wittgenstein: 
        \begin{itemize}
            \item Example: Poker, solitare, fishing, chess
            \item There are all diferent but can be considered a family if objects game.
            \item They don't have the same atributes.
            \item Family resemblence.
        \end{itemize}
    
    \item ``Fuzzy'' concepts:
        \begin{itemize}
            \item You cannot define a concept more clearly than it appears to us in the world, ``for imagine having to sketch a shaply defined picture corresponding to a blurre one''.
        \end{itemize}
    
        
    \item ``If you are surprised that one can now something and not be able to say it, you are perhaps thinking of a case like the first.''
\end{itemize}


%----------------------------------------------------------------------------------------
\subsection{Problems of the prototype}
\begin{itemize}
    \item Many concepts lack prototypes, there are somethings that can't be cataloged by prototyping because they don't have a clear prototype to identify, example you can prototype fish, birds but ``a pet fish that swallowed their owner in Armenia'' cannot be prototyped.
    \item No compositionality, prototypes are often not functions of the prototypes of their constituent concepts, remember the bachelor, man, unmarried, they can not be applied in the theory of prototyping.
\end{itemize}



%%%%%%%%%%%%%%%%%%%%%%%%%%%%%%%%%%%%%%%%%%%%%%%%%%%%%%%%%%%%%%%%%%%%%%%%%%%%%%%%%%%%%%%%%%
\section{Theory theory}
\begin{itemize}
    \item Concepts stand in:
        \begin{enumerate}
            \item Relation with one another in the same way as 
            \item The terms of a scientific theory and 
            \item Categorization is a process that strongly resembles 
            \item Scientific thorizing
        \end{enumerate}
    
    \item Shape concepts in iterations: 
        \begin{itemize}
            \item Develop abstrat, coherent system or a \emph{theory}
            \item Make predictions and interpret and explain evidence.
            \item Experiment to test the thoery and reevaluate it 
            \item When it's falsified we seek alternatives
            \item Repeat if falsified // 
        \end{itemize}
    
        
    \item The theory is not said as truth but is thought of as the most accurate way to explain up to date.
\end{itemize}


%----------------------------------------------------------------------------------------
\subsection{Problems - Stability}
\begin{itemize}
    \item If concepts are theories that are ever evolving, 
    \item How could ever two persons share the same concept?
    \item Do we know the same concept as yesterday? 
\end{itemize}



%%%%%%%%%%%%%%%%%%%%%%%%%%%%%%%%%%%%%%%%%%%%%%%%%%%%%%%%%%%%%%%%%%%%%%%%%%%%%%%%%%%%%%%%%%
\section{The conccptual atomism theory}
\begin{itemize}
    \item A casual relation between 
    \item The concept and it's 
    \item Instance determines it's reference 
    \item A causal relation between the instance determined it's reference.
    \item Concepts are primitive and have no structure.
\end{itemize}

\subsection{2,500 years of phylosophy}
\begin{itemize}
    \item It's not setled, the conceptual structure is unsettles.
    \item Solution?:
        \begin{itemize}
            \item No ``one true'' structure of concepts 
            \item Diferent structure for diferent explanatory functions
        \end{itemize}
\end{itemize}

\subsection{A pluralism about concepts}
\begin{itemize}
    \item Concepts might have multiple ``paralel'' structure operation in our minds.
    \item This is called the Dual Theory where we differentiate between:
        \begin{itemize}
            \item Identification procedure 
            \item Core component 
        \end{itemize}
    
    \item Identification procedure:
        \begin{itemize}
            \item Quick categorizartion 
            \item Based on prototype theory 
        \end{itemize}
    
    \item Core component: 
        \begin{itemize}
            \item Used when resources are not limited, a combination of clasical and thoery and atomism theory.
            \item There is more knowledge.
        \end{itemize}
    
    \item Concept of concepts are not universally defined, in artifitial inteligence this is a problem, this is a disagreement because sometimes we must model the human mind and we are not in agreement as to how a human mind handles concepts.
\end{itemize}

%%%%%%%%%%%%%%%%%%%%%%%%%%%%%%%%%%%%%%%%%%%%%%%%%%%%%%%%%%%%%%%%%%%%%%%%%%%%%%%%%%%%%%%%%%
\section{Resources}
\begin{itemize}
    \item Wittgenstein heritage 
    \item Concept coreseries Eric Ergo
    \item Gregory All Murphy Concepts
\end{itemize}
