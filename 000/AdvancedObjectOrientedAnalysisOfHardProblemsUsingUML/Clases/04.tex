\section{Intentionality}
\begin{itemize}
    \item The ``aboutness'' of the mind 
    \item Intentionality is the way that the mind is directed at or about the world.
    \item An intentionality state consists of:
        \begin{enumerate}
            \item A propositional content, p
            \item A psychological mode, S
        \end{enumerate}
        it is noted as S(p)
    
    \item Propositional content: 
        \begin{itemize}
            \item Whatever that-clauses contribute to whatever the intentionality is about.
            \item Example:
                \begin{itemize}
                    \item Ralph believes that Jack Sparrow is alive.
                    \item Ralph hopes that Jack Sparrow is alive.
                    \item Ralph desires that Jack Sparrow is alive.
                \end{itemize}
            
            \item This will formally be noted as Ralph: 
                \begin{itemize}
                    \item The first one: Believe(is alive)
                \end{itemize}
        \end{itemize}
    
    \item Psychological mode:
        \begin{itemize}
            \item You can keep the propostional content constant while varying the mode.
            \item There are generic modes.
            \item Some people think tha beliefs and desires can form all teh other modes..
            \item Remember the logical operators and the double negatives.
        \end{itemize}
        \item Conditions of satisfaction:
        \begin{itemize}
            \item The propositional content also determines what will count as it's condition of satisfaction:
                \begin{itemize}
                    \item What makes the belief true? 
                    \item What makes the intentions and desires fulfilled?
                \end{itemize}
            
            \item Propositional content dosnt's have to be text.
            \item Condition of satisfaction is a necesity fulfulled from doing a task.
        \end{itemize}
    
    \item Direction of fit:
        \begin{itemize}
            \item (Water) If I desire water (CoS) when water gets inserted into my mouth that is the condition of satisfaction (Mind)
        \end{itemize}
\end{itemize}



%%%%%%%%%%%%%%%%%%%%%%%%%%%%%%%%%%%%%%%%%%%%%%%%%%%%%%%%%%%%%%%%%%%%%%%%%%%%%%%%%%%%%%%%%%
\section{Sumarry lecture}
\begin{itemize}
    \item We have concepts about objects in propositional content as part of intentional states in the mind.
    \item Concepts, objects are instances of objects.
\end{itemize}



%%%%%%%%%%%%%%%%%%%%%%%%%%%%%%%%%%%%%%%%%%%%%%%%%%%%%%%%%%%%%%%%%%%%%%%%%%%%%%%%%%%%%%%%%%
\section{The network of intentionality and the background}
\begin{itemize}
    \item Intentional states come in a network.
    \item Example: Intend to drive to work, but you can't do that unless you believe you have a job, you have a license, you have teh capacity... 
    \item There form a network not individual units.
    \item Background:   
        \begin{itemize}
            \item Wittgenstein: You woudn't even understand an action with out a background.
            \item We have a hurly-burly that are going on at the same time and only this can give something meaning, this is called background.
            \item If im playing the piano I don't think of how I place fingers but just playing the chords.
        \end{itemize}
    
    \item The ``background'':
        \begin{itemize}
            \item The network is not infinite 
            \item If you follow out the threads and you will find non-intentional states.
            \item You don't belive that your finges can grip the steering wheel you just take that for granted.
            \item This is a very impotant role when we conceptualize anything.
        \end{itemize}
\end{itemize}



%%%%%%%%%%%%%%%%%%%%%%%%%%%%%%%%%%%%%%%%%%%%%%%%%%%%%%%%%%%%%%%%%%%%%%%%%%%%%%%%%%%%%%%%%%
\section{Collective intentionality}
\begin{itemize}
    \item Emerges when we have intentional states about other people's intentionality states.
    \item In coordination with others we create the possibility of collective intentionality.
    \item It emerges as a result of:
        \begin{enumerate}
            \item Multilie people 
            \item Having intentional states andmutual intentional beliefs about 
            \item each other's intentional states.
        \end{enumerate}
    
    \item They have an intenional state of each other's intentional state.
    \item Multiplier effect: 
        \begin{itemize}
            \item Systems exists in out collective intentionality. There are no objective independent systems.
            \item They only exist because I think they exist, you think they exist and I think you think they exist and you think I think they exist.
            \item This is recursive, reciprocity in concepts of one individual with respect to the other.
            \item This is the way we form institutional facts, such examples of these facts is money, money just has value in our minds. 
        \end{itemize}
\end{itemize}


%%%%%%%%%%%%%%%%%%%%%%%%%%%%%%%%%%%%%%%%%%%%%%%%%%%%%%%%%%%%%%%%%%%%%%%%%%%%%%%%%%%%%%%%%%
\section{Sumarry}
\begin{itemize}
    \item We use our concepts about objects and facts in the world as part of the propositional content in intentional states in our minds.
    \item These intentional states come in a network that takes certain background abilities for granted.
\end{itemize}



%%%%%%%%%%%%%%%%%%%%%%%%%%%%%%%%%%%%%%%%%%%%%%%%%%%%%%%%%%%%%%%%%%%%%%%%%%%%%%%%%%%%%%%%%%
\section{Language overview - second corner}
\begin{itemize}
    \item Language: ``The systematic creation maintenance and use of systems of symbols, which dynamically reference concepts and assemble according to structured patterns to communicate meaning''
    \item It's something that is systematic, there are rules, languages also evolves, it's a systems of symbols as well.
    \item Components of languages:
        \begin{enumerate}
            \item Phonology: how words and sentences are pronounced.
            \item Syntax: how words are arranged in sentences.
            \item Semantics: meaning of words and morphems.
            \item Pragmatics: sets general constraints on the use of language.
        \end{enumerate}
    
    \item What do we do with a language? 
        \begin{itemize}
            \item We use language to communicate our intentional states and, as a subset of that, to inform about the world.
        \end{itemize}
\end{itemize}



%%%%%%%%%%%%%%%%%%%%%%%%%%%%%%%%%%%%%%%%%%%%%%%%%%%%%%%%%%%%%%%%%%%%%%%%%%%%%%%%%%%%%%%%%%
\section{Terms and propositions}
\begin{itemize}
    \item Proposition in phylosophy of language:  
        \begin{enumerate}
            \item The ``content'' or ``meaning'' of a meaningful declarative sentence.
            \item The pattern of symbols, marks, or sounds that make up a meaningful declarative sentence.
        \end{enumerate}

    \item A propositions is: 
        \begin{itemize}
            \item A pattern of symbols, a proposition is the pattern of symbols, marks or sounds that make up a meaningful declarative sentence.
        \end{itemize}
    
    \item Symbols (os sign or term etc):
        \begin{itemize}
            \item Link of relationship between the signified and the signifier:
                \begin{itemize}
                    \item The signified is entity in the world or concept in the mind.
                    \item Carl(sign referencing a real person) is a head master(sign refering to a concept ``headmaster'' in the mind).
                \end{itemize}
        \end{itemize}
    
    \item Terms refer to concepts:
        \begin{itemize}
            \item Concepts do not have ``names''
            \item We refer to concepts using signs or terms.
            \item Think of a concept, the concept in the mind doesn't have a name, name the concept agreement or name the concept contract? 
            \item It's a misunderstaning.
            \item This is actually called synonyms and homonyms.
            \item Homonyms and synonyms are common sources for misunderstanding.
        \end{itemize}
\end{itemize}



%%%%%%%%%%%%%%%%%%%%%%%%%%%%%%%%%%%%%%%%%%%%%%%%%%%%%%%%%%%%%%%%%%%%%%%%%%%%%%%%%%%%%%%%%%
\section{Generativity and compositionality}
\begin{itemize}
    \item Features of propositions, propositions have compositionality.
        \begin{itemize}
            \item Compositionality is the framework of rules in which morphemes can be composed, it applies to both syntax and semantics.
            \item The lion ate the apple, or, the apple ate the lion; both have meaning to us.
        \end{itemize}
    
    \item Generativity: is the feature of a language in which it:
        \begin{itemize}
            \item Through recursive syntactical operations (subordinate clauses and conjuntions)
            \item You could produce an infinite number of sentences.
            \item Example: you can append to a sentence more and more compositionality and generativity infinetly.
        \end{itemize}
\end{itemize}

%----------------------------------------------------------------------------------------
\section{Sentence meaning is not enough}
\begin{itemize}
    \item Signs are composed in propositions according to syntactical rules and creates: Sentence meaning.
        \begin{itemize}
            \item Example: can you go to the desk overthere? r// yes i can, I would just use my legs. 
            \item From a sentence meaning it's a correct answer, but clearly the person is asking a favor in the way ``I would like you to go to the desk''
            \item Sentence meaning doesn't transmit intentionality.
            \item This is a computer problem, it's ambiguous to talk to a computer using only sentence meaning, you need to speak literally.
        \end{itemize}
\end{itemize}



%----------------------------------------------------------------------------------------
\section{The king of France is bald}
\begin{itemize}
    \item ``The king of France is bald'' Bertrand Russell.
    \item Natural languages can sometimes be very ambiguous.
    \item If the king of france isn't bald then the negation must be true? this is not always the case.
        \begin{itemize}
            \item \[
              \exists x ( \text{ King of France(x) \&} \forall y(\text{ King of France(y) } \rightarrow y=x) \text{ \& } 
              \text{ Bald }(x)
              )
            \]
            
            \item Or:
                \begin{enumerate}
                    \item There is an $x$ such that $x$ is presently king of France.
                    \item For any $x$ and $y$, if $x$ is presently King of France and $y$ is presently King of France then $x=y$ 
                    \item For every $x$ that is presently King of France, $x$ is bald.
                \end{enumerate}
                \begin{itemize}[label=\#]
                    \item In this case one condition fails, there is no presently a king of France.
                \end{itemize}
        \end{itemize}
\end{itemize}


%----------------------------------------------------------------------------------------
\section{The indeterminancy of translation}
\begin{itemize}
    \item ``The indeterminancy of translation'' Willian Van Oranam Quine.
    \item The rabbit example: ``your out in the bushes, with a native speaker of the language Arunta, a rabbit passes by, he yells 'Gavadai' how could you translate this statement? ''
        \begin{itemize}
            \item Lo food? 
            \item Let's go hunting? 
            \item There will be a storm tonight? 
            \item How do you determine the transation?
            \item This is not about the fact that you don't speak arunta, every child struggles with this, in the bussiness world people speaking the same langueage sometimes are not clear in their comunciations. 
            \item The theory theory ``how do we know that we all have the same concept?''
        \end{itemize}
\end{itemize}



%----------------------------------------------------------------------------------------
\section{Speech acts}
\begin{itemize}
    \item Speech acts are the Uttarances real, intended meaning.
    \item The insufficient sentence meaning, an enourmous amount of background that is presupused when we comunicate. There is always an underlaying background given when we communicate. 
    \newline 
    Example: 
        \begin{itemize}
            \item Let's go out for a drink?
            \item Sorry I can´t, my doctos wond't allow me.
            \item What's the matter with you? 
            \item --------------------------------
            \item Let's go out for a drink?
            \item Sorry,i can't my mother in law won't allow me.
            \item What's the mater with you?
        \end{itemize}
        See the diference? Language is ambigous.
    
    \item We can't spell out all the context, this would make communication imposible, communication is necesarily lacking.
    \item Intentionality's natural extension, intentional states lead to speech acts.
    \item The intentional state ``S'' about a proposition ``p'':
        \[
          \text{ S(p) }
        \]
        Translates to an illocutionary force ``F'' about the same proposition ``p'':
        \[
          \text{ F(p) }
        \]
        Example: ``$\underbrace{\text{ The salt passed to me (p) }}_{\text{ Intentional state }}$ '' $\rightarrow$ ``$\underbrace{\text{ Please hand me the salt! (p) }}_{\text{ Speech act }}$ '' $\rightarrow$ Request (F)
\end{itemize}



%----------------------------------------------------------------------------------------
\section{Meaning through speech acts}
\begin{itemize}
    \item Example: ``$\overbrace{\underbrace{\text{ True is a cat sits on a mat }}_{\text{ Condition of satisfaction }}}^{\text{ Belief (S) }}$'', Intention ``$\underbrace{\text{ This utterance has identical condition of satisfaction as my belief  }}_{\text{ Intention (S) }}$ '' ,the speech act ``$\overbrace{\underbrace{\text{ There is a fluffy mouse catcher on the mat }}_{\text{ (p) }}}^{\text{ Assertion(F) }}$'', given mind to world and a word to fit world.
\end{itemize}



%----------------------------------------------------------------------------------------
\section{Five types of speech acts}
\begin{enumerate}
    \item Asertives:
        \begin{itemize}
            \item Commits a speaker to the truth of the expressed proposition. 
            \item Commits me to the truth given that I said something.
        \end{itemize}
    
    \item Directives:
        \begin{itemize}
            \item Cause the hearer to take a particular action, a request or order.
            \item Example: SALT, NOW! 
        \end{itemize}
    
    \item Commissives:
        \begin{itemize}
            \item That commits a speaker to some future action, a promise.
        \end{itemize}
    
    \item Expressive:
        \begin{itemize}
            \item Used to express the speakers attitudes and emotions towards the proposition.
        \end{itemize}
    
    \item Declarations:
        \begin{itemize}
            \item Change the reality in accordance with the roposition of the declaration, for example declaring someone guilty.
        \end{itemize}
        
\end{enumerate}

They have relation with intentionality:
\begin{itemize}
    \item Asertive $\leftrightarrow $  Belief 
    \item Directive $\leftrightarrow $  Desire 
    \item Commisive $\leftrightarrow $ Intention
    \item Expressive $\leftrightarrow $ Emotive
\end{itemize}


%----------------------------------------------------------------------------------------
\section{The strange thing about declarations}
\begin{itemize}
    \item ``This party starts now'', as Declaration (F): there is no intentional state directly linked to that, I can believe and desire the party has started but it will not start until I perform the declaration.
    \item If you are a judge you can't just judge you have to declare.
    \item Declarations have double direction of fit: Its conditions of satisfaction are fulfilled when:
        \begin{itemize}
            \item The act as such 
            \item Alter the world in the way that 
            \item The world is represented to be altered in propositional content.
        \end{itemize}
        We have double direction of fit.
    
    \item Power of declaration: declarations are one of the most important aspects of language:
        \begin{itemize}
            \item We as humans use this to create our social world.
            \item Declarations are the building blocks of the social world, they are used recursivelly.
            \item Examples: you need to declare it to make it happen.
        \end{itemize}
\end{itemize}



%----------------------------------------------------------------------------------------
\section{Language summary}
\begin{itemize}
    \item We are a reace of humans who have consciousness, intentionality and the capacity of the collective intentionality. We can represent objects and facts that we believe, desire, intent to do, etc. 
    \item  Language has diferent components and on the lowest level, terms are combined into propositions that have sentence meaning.
    \item Language is also the natural extension of the intentionality. Speech acts are utterances real, intended, meaning and is made up os illocunary force and a proposition (statement).
    \item The illocutionary force connects the intentionality with the language usage and corresponding facts and objects in the world.
\end{itemize}


%%%%%%%%%%%%%%%%%%%%%%%%%%%%%%%%%%%%%%%%%%%%%%%%%%%%%%%%%%%%%%%%%%%%%%%%%%%%%%%%%%%%%%%%%%

\section{Metaphysics - introduction}
\begin{itemize}
    \item Metaphysics of the world: objects, properties and relations, facts and truth, functions and social facts.
\end{itemize}


%----------------------------------------------------------------------------------------
\section{Objects}
\begin{itemize}
    \item Objects exists so we need a metaconcept of existance.
    \item Existance - familiar, yet elusive:
        \begin{itemize}
            \item We know how to use the word but have a hard time describing it.
            \item One could be inclined to call only material stuff.
            \item However, that will rule out a lot of the things that we with just a reflection also would call objects. Example: products, agreements, etc. 
            \item A general notion that tends to hold for most situations is ``anything that may be presented to the mind; object of thought''.
        \end{itemize}
    
    \item Concrete objects:
        \begin{itemize}
            \item They only have spatiotemporal properties, they ocupy space and exist for a period of time. Examples: mountains, trees, etc.
            \item Not only spatio-temporal: do human-created material objects exists in the same way as natural objects? Example: upside down woman, upside down table.
            \item Even spaciotemporal objects are always tied to something else.
        \end{itemize}
    
    \item To be or not to be, four different meanings of ``is'', 
        \begin{itemize}
            \item the ``is'' of existance:   
                \begin{itemize}
                    \item ``Socrates is'', $\exists x ( x= \text{ Socrates })$ ''
                \end{itemize}
            
            \item The ``is'' of identity: 
                \begin{itemize}
                    \item ``Hesperus is Phosphorus'' $\text{ Hesperus } = \text{ Phosphorus }$, here we have two words for the same objects.
                \end{itemize}
            
            \item ``The is of predication''
                \begin{itemize}
                    \item ``Socrates is wise'' $\text{ Wise }(\text{ Socrates })$ 
                \end{itemize}
            
            \item ``The ``is'' of general implication'':
                \begin{itemize}
                    \item ``Man is an animal'' $\forall x (\text{ Man }(x) \rightarrow \text{ Animal }(x))$  if $x$ is a man then $x$ is also an animal.
                \end{itemize}
        \end{itemize}
    
    \item Objects of objects: Objects can be composed out of smaller parts, Objects are not always simple, there are objects composed of objects, this is called aggregation. This is called a whole-part relation.
    \item When is an object composed of parts?
        \newline Examples: 
        \begin{itemize}
            \item In contact? Must all parts have to be in contact with each other? is an atom an object? under this definition an atom woudn't be an  object, and that is wrong.
            \item We can say that they must have a force-wise collectively fastened to each other? How much fastened? Example: deck of cards, is it a deck of cards if I spread them around? 
            \item The ``two-woman'': if two people shake and thir fingers get stuck, are they then one object? no, they are still two diferent women.
            \item Maybe there are no composite objects, only mereological simples? indentity crisis? but if there wew only simples, how can human beings with identity exist? we are actually a compositions of a lot of objects.
        \end{itemize}
        A diferent angle: 
        \begin{itemize}
            \item Existence is a. second level feature that applies to concepts, not objects.
            \item That something exists is to say that the concepts have instances.
            \item It's in the concepts, in this way, the existance of composition is dependent on the concepts we have, not a spceific feature of the physical object.
            \item Example: duck or a rabbit? where is the criteria.
        \end{itemize}
\end{itemize}


%----------------------------------------------------------------------------------------
\section{Properties and relations}
\begin{itemize}
    \item Properties: are the 
        \begin{itemize}
            \item attributes or 
            \item qualities or 
            \item features or 
            \item characteristics of 
        \end{itemize}
        things.
    
    \item Phenomena of interest: 
        \begin{itemize}
            \item Properties are typically introduced to help explain or account or phenomena of philosophical interest. 
        \end{itemize}
    
    \item How do we notice properties? when we have a breakdown case: related to phenomenology of Heidegger ``A just perfect hammer for me''
        \begin{itemize}
            \item When the hammer weighs ``just perfect for me '' then we don't hammer and at the same time think ``such a perfect weight for a hammer'' the hammer is just a natural part of the world. I won't think of the hammer if it's perfect for me.
            \item Something stop's us: for some reason we stop and start contemplating the tool, it suddenly for this particular job it's too heavy,
            \item Breakdown case, we contemplate on the hammer and notice that hammers in general could be too light, perfect, too heavy, but they are all for me.
            \item De-worlding: we remove ourselves from the equation and attribute hammers as having weight. If there were no break down case we wouldn't notice the object's properties.
            \item In 99.999\% of the cases we dont notice the properties of an object unless the object doesn't ``do it's job'' or if we are reflecting on the nature of the object.
            \item No breakdown case $\rightarrow$ no property. Example: does money have the propesty weight? in the old times it did, it was a property weight, now there isn't a weight property in today's money.
        \end{itemize}
    
    \item Primary and secondary properties: 
        \begin{itemize}
            \item Primary: properties are objective features of the world: shapes, size, mass, etc. 
            \item Secondary properties are mind-dependent: colors, tests, etc. (intrincic or relative properties)
        \end{itemize}
    
    \item Relations: 
        \begin{itemize}
            \item ``Beind in love is a specific type of property'' love involves multiple types of objects, this is a special property.
            \item Relations are a type of property, relations are properties that exists in multiple things, the property being in love is a two-place relation, there can be multi-place relations as well.
            \item Relations have a single relation: 
                \begin{itemize}
                    \item Romeo $\underbrace{\leftrightarrow }_{\text{ Loves }}$ Juliet
                \end{itemize}
                The objects have roles in the relation or they can be multidimentional.
                \newline Example: 
                \begin{itemize}
                    \item Work well together.
                \end{itemize}
        \end{itemize}
    
    \item Instantiation: ``instantiates'' is a special kind of relation between an object and a propery (concept). 
        \begin{itemize}
            \item Relation between objects and the concept we have in our mind. 
        \end{itemize}
\end{itemize}



%----------------------------------------------------------------------------------------
\section{Facts}
\begin{itemize}
    \item Facts: a situation that the actual world must be in to make a given proposition about the world true: the truth maker.
        \begin{itemize}
            \item At any given situation, the world ob objects with their properties is arranged in a certain way. This we call a fact.
            \item Example: 7 yellow ducks, it is a ``state of affars'' or ``fact''; the proposition ``means'' the fact.
        \end{itemize}
    \item The fact is not just an alone object, the fact obtains if an object excemplifies:
        \begin{enumerate}
            \item at least one property or 
            \item one or mode objects stand in relation 
        \end{enumerate}
        \begin{itemize}
            \item ``A car''is not a fact, ``two pink cars that face each other'' is a fact  they stand in relation and excemplify the properties.
        \end{itemize}
    
    \item The truth-maker and truth bearer: 
        \begin{itemize}
            \item Truth-maker: the fact.
            \item Truth bearer: the actual propoition about the fact. In this relation. 
        \end{itemize}
    
    \item Facts are not true or false: 
        \begin{itemize}
            \item Propositions can be true or false, the truth value is a meta-linguistic property of a proposition. 
            \item A fact in it of itself is not true nor false, or more generically state-of-affairs, can only obtain or fail-to-obtain.
        \end{itemize}
    
    \item Propositions picture simple facts: 
        \begin{itemize}
            \item In the simplest case, the proposition can be thought of as a ``picture'' of the real world fact. The propositional content is the content of the ``picture''.
            \item Counterfactuals are also facts: the negation of some facts. Picture the negation of the fact or the fact with variation, there are endless ways to negate a fact so this is tricky.
        \end{itemize}
    
    \item Facts are any condition that makes a proposition true: 
        \begin{itemize}
            \item Not just a picture.
        \end{itemize}
\end{itemize}



%----------------------------------------------------------------------------------------
\section{Social facts}
\begin{itemize}
    \item All types of objects and properties are involved in facts, not only natural kinds. Example: the money is on the mat, that is a fact.
    \item Objective knowledge: this means that we can have objectively true knowledege about subjective relative phenomena. The statement, proposition can be true, where as tha subjective opinion is isolated. Example: the country is in a resion is a fact based on the subjective fact.
    \item There is no true knowledge at all, because all knowledge is derivated from subjective phenomena, this is called the \textbf{pragmatic truth}.
\end{itemize}


%----------------------------------------------------------------------------------------
\section{Status functions adn institutional facts}
\begin{itemize}
    \item Function: 
        \begin{itemize}
            \item Function is always observer relative, secondary property that is. 
            \item Example: hammer, it's part of a whole, a funtion is defined in a context of others.
            \item Functions are contextualizes explicitly or implicitly in systems.
            \item We assign a function to an object, this function plays a role  in the system.  
        \end{itemize}
    
    \item System has goals: 
        \begin{itemize}
            \item The system strives towars some type of precieved goal, value, purpose or goal, a system. This shoun by the fact that they can fail, or malfunction.
            \item Example: hammer, nails, planks function as a shelter. 
            \item Exists on a background, the whole system axists within an intentional network and preuppose a background. 
            \item Example: I use hammers to drive nails into planks to build a house, which is a shelter. I need shelter because otherwise I would freeze. 
        \end{itemize}
    
    \item No purpose, goal os value, no function: true also for bio-physical function. 
        \begin{itemize}
            \item This is true for all systems.
            \item Biological functions are assigner in respect of some higher imposed value.
            \item The function server a role in the explanation of a theory.
            \item The functions are no intrinsic: We impose them on organs to explein them, if you cahnge the value to destruction from survical, then the organs are malfunctioning, the system always strive to accomplish a higher goal or purpose.
        \end{itemize}
    
    \item Status functions: A special type of agentive function is the status function. We use the declaration speech act to create a function. Example: ``this hereby counts as money in this room'' if everyone accepts this declaration then that piece of paper is money in that room. 
     \newline  We impose functions on objects using declaration of the format:
        \[
          X \; \text{ counts as }\, Y \; \text{ in } \; C
        \]
        
    \item Recursion does the rest for us: 
        \begin{itemize}
            \item This is the way we create social reality.
            \item We declare status functions with the format and then we use recursion to construct the social reality.
        \end{itemize}
    
    \item Steps for status functions:
        \begin{enumerate}
            \item Declare a system 
                \begin{itemize}
                    \item Example: there is hereby a concept called money which the central bank is our state is allowed to issue. Declaration (F).
                \end{itemize}
            \item Recognize and accept: 
                \begin{itemize}
                    \item Everyone accepts the intitution and the money concept.
                \end{itemize}
            
            \item Assign a function to agents: 
                \begin{itemize}
                    \item Who does what.
                \end{itemize}
            
            \item Recognize and accept the declaration of powers: 
                \begin{itemize}
                    \item We know the concept and assign status functions to the concept and object.
                \end{itemize}
            
            \item Declare intitutional facts:
                \begin{itemize}
                    \item We trust and know the social facts of intitutions.
                    \item Recursion
                \end{itemize}
        \end{enumerate}
    
    \item Functions in ``thin air'':
        \begin{itemize}
            \item They don't need for an object at all, these function are called be Searle ``free-standing Y terms'', there is no X in the statement.
            \item You can have a status function without the X.
            \item The normal status function has the power to issuea declaration, example: ``thereby counts as a corporation (declaration F)'' Corporation is Y, the word corporation is Y, a corporation exists from the declaration, and is sometimes independent of X.
            \item We can't do this without language, we can't just think and make it so, a declaration is requiered.
            \item Language is super important.
        \end{itemize}
    
    \item With: 
        \begin{enumerate}
            \item Collective intentionality.
            \item Declare speech acts and
            \item Recursion 
        \end{enumerate}
        We create our social world:
        \begin{itemize}
            \item This realli exists in our collective minds, somethings don't exist out of the social collective mind.
        \end{itemize}
    
        
    \item Things can exist in diferent forms:
        \begin{center}
           \begin{tabular}{ | p{5cm} | p{5cm} | p{5cm} | }
               \hline
                   Exists independently to us & Exists since we use and treat as such & Exists since we have jointly declared them to do so     \\
               \hline
                    \begin{itemize}
                        \item Human
                        \item Being 
                        \item Bacteria 
                        \item Sun 
                        \item Tree 
                        \item Mountain 
                        \item Earth 
                        \item River 
                    \end{itemize}
                    & 
                    \begin{itemize}
                        \item Hammer 
                        \item Table 
                        \item Computer 
                        \item Screwdriver 
                        \item Fork 
                        \item Engine 
                        \item Thermometer
                    \end{itemize}
                    & 
                    \begin{itemize}
                        \item University degree 
                        \item Birthday party 
                        \item President 
                        \item Money 
                        \item Loan 
                        \item Insurance 
                    \end{itemize} 
                    \\ 
               \hline
           \end{tabular}
        \end{center}
        \begin{itemize}
            \item We use language to create social reality.
        \end{itemize}
\end{itemize}


%----------------------------------------------------------------------------------------
\section{The triangle in new light - Summary}
\begin{enumerate}
    \item We take for granted that there are concrete objects and state of affairs that exists.
    \item We have concepts about these objects 
    \item We form intentional states based on the concepts 
    \item We use language signs that represent the objects and concepts.
    \item We make declarations of new concepts and assignments of these.
    \item Those status functions are collectively rectognized and accepted
    \item We apply the new concept and thus we create new abstract objects in the world.
    \item Repeat and repeat.
\end{enumerate}
This is done very implicitly, the objective is to grab those facts and objects and make them explicit.

%----------------------------------------------------------------------------------------
\section{Resources}
\begin{itemize}
    \item Making the social world 
\end{itemize}
