\subsection{Enums}
\begin{itemize}
    \item Enums are data type that allows a programmer to define a variable and specify the valid values that could be stored in that variable.
    \item It allows you to esentially create your own data type.
    \item To use enums you need to give it a name, use the enum keyword, after this provide a name of that enum, after list the values available.
    \item Example: \verb|enum primaryColor {red, yellow, blue};| this can only store the values defined inside the curly braces. 
    \item Variables that are declared enum have to follow the definition standards in the declaration.
    \item You can declare it the same way as a variable, you can declare them and them.
    \item If you specify a value and then call a diferent one we get a compiler error.
\end{itemize}
%%%%%%%%%%%%%%%%%%%%%%%%%%%%%%%%%%%%%%%%%%%%%%%%%%%%%%%%%%%%%%%%%%%%%%%%%%%%%%%%%%%%%%%%%%%%%%%%
\subsection{Enums as ints}
\begin{itemize}
    \item The enums are stored as ints and have to be called that way, this is important.
    \item Access this in the same way you would a list.
    \item Each element is asociated by intigers, you can assign values explicitly that are sociated, for example: \verb|enum direction {up, down, left=10, right};| , you can control left and set it to 10.
\end{itemize}
%%%%%%%%%%%%%%%%%%%%%%%%%%%%%%%%%%%%%%%%%%%%%%%%%%%%%%%%%%%%%%%%%%%%%%%%%%%%%%%%%%%%%%%%%%%%%%%%
\subsection{Char}
\begin{itemize}
    \item Chars represent a single character such as the letter 'a'.
    \item You can declare the char data type, use the char keyword, and assign a value \textbf{inside single quotes ''}.
    \item \textbf{Single quotes} are for chars, \textbf{double quotes} are for strings.
    \item A char can be a number valid in the ascii table, for example \verb|char grade = 65; | will be A in the ascii table.
\end{itemize}
%%%%%%%%%%%%%%%%%%%%%%%%%%%%%%%%%%%%%%%%%%%%%%%%%%%%%%%%%%%%%%%%%%%%%%%%%%%%%%%%%%%%%%%%%%%%%%%%
\subsection{Escape characters}
\begin{itemize}
    \item These characters represent an action.
    \item You can represent this actions to represent a newline for example; for example \verb|char x = '\n';| this will be a new line, it won't print out in the console as \verb|'\n'| it will just print a new line.
    \item See: C Primer Plus, Prata.
        \begin{itemize}
            \item \verb|\a| Alert (ANSI C)
            \item \verb|\b| Backspace
            \item \verb|\f| Form feed
            \item \verb|\n| Newline 
            \item \verb|\r| Carrige return 
            \item \verb|\t| Horizontal tab
            \item \verb|\\| Vertical tab 
            \item \verb|\'| Backslash 
            \item \verb|\"| Single quote
            \item \verb|\?| Question mark
            \item \verb|\0oo| Double qoute 
            \item \verb|\xhh| Hexadeciml value, h is a hexadecimal digit.
        \end{itemize}
\end{itemize}
