\begin{center}
    \begin{itemize}
        \item Let's suppose that we start at an underflow condition.
        \item REAR = SIZE - 1 
        \item FRONT = SIZE - 1
        \item SIZE = 5
    \end{itemize}
    \begin{tabular}{ cc }
        {R$\rightarrow$}&{4} \\ {}&{3} \\ {}&{2} \\ {}&{1} \\ {}&{0} \\
    \end{tabular}
    \begin{tabular}{|p{0.75cm}|}
        \hline {} \\ \hline {} \\ \hline {} \\ \hline {} \\ \hline {} \\ \hline
    \end{tabular}
    \begin{tabular}{ c }
        {$\leftarrow$F} \\ {} \\ {} \\ {} \\ {} \\
    \end{tabular}

    \begin{minted}[autogobble]{c}
        CircularQueue.enqueue(5);
        // Increase the rear: r = (4+1)%5 -> 0
    \end{minted}

    \begin{tabular}{ cc }
        {}&{4} \\ {}&{3} \\ {}&{2} \\ {}&{1} \\ {R$\rightarrow$}&{0} \\
    \end{tabular}
    \begin{tabular}{|p{0.75cm}|}
        \hline {} \\ \hline {} \\ \hline {} \\ \hline {} \\ \hline {5} \\ \hline
    \end{tabular}
    \begin{tabular}{ c }
        {$\leftarrow$F} \\ {} \\ {} \\ {} \\ {} \\
    \end{tabular}

    \begin{minted}[autogobble]{c}
        CirucularQueue.enqueue(10);
        // rear: (5+1)%5 = 1
    \end{minted}

    \begin{tabular}{ cc }
        {}&{4} \\ {}&{3} \\ {}&{2} \\ {R$\rightarrow$}&{1} \\ {}&{0} \\
    \end{tabular}
    \begin{tabular}{|p{0.75cm}|}
        \hline {} \\ \hline {} \\ \hline {} \\ \hline {10} \\ \hline {5} \\ \hline
    \end{tabular}
    \begin{tabular}{ c }
        {$\leftarrow$F} \\ {} \\ {} \\ {} \\ {} \\
    \end{tabular}

    \begin{minted}[autogobble]{c}
        CircularQueue.enqueue(15);
        // rear\alpha (6+1)%5 = 2
    \end{minted}

    \begin{tabular}{ cc }
        {}&{4} \\ {}&{3} \\ {R$\rightarrow$}&{2} \\ {}&{1} \\ {}&{0} \\
    \end{tabular}
    \begin{tabular}{|p{0.75cm}|}
        \hline {} \\ \hline {} \\ \hline {15} \\ \hline {10} \\ \hline {5} \\ \hline
    \end{tabular}
    \begin{tabular}{ c }
        {$\leftarrow$F} \\ {} \\ {} \\ {} \\ {} \\
    \end{tabular}

    \begin{minted}[autogobble]{c}
        CircularQueue.enqueue(20);
        // rear: (7+1)%5 = 3
    \end{minted}

    \begin{tabular}{ cc }
        {}&{4} \\ {R$\rightarrow$}&{3} \\ {}&{2} \\ {}&{1} \\ {}&{0} \\
    \end{tabular}
    \begin{tabular}{|p{0.75cm}|}
        \hline {} \\ \hline {20} \\ \hline {15} \\ \hline {10} \\ \hline {5} \\ \hline
    \end{tabular}
    \begin{tabular}{ c }
        {$\leftarrow$F} \\ {} \\ {} \\ {} \\ {} \\
    \end{tabular}

    \begin{itemize}
        \item An error will occur because in the next insertion will satisfy the underflow condition (rear == front).
    \end{itemize}
    \begin{minted}[autogobble]{c}
        CircularQueue.enqueue(25);
        // rear: (8+1)%5 = 4
    \end{minted}
    \begin{itemize}
        \item This operation will enter the if statement of the queue underflow.
            \begin{minted}[autogobble]{c}
                if (rear == front){
                    printf("Underflow");
                }
            \end{minted}
        
        \item The insertion will never happen. 
        \item We need to dequeue the last value at index 4.
        \begin{minted}[autogobble]{c}
            CircularQueue.dequeue();
            // This will set the front back to 0
            // f = (f+1)%SIZE: (4+1)%5 = 0
        \end{minted}
    \end{itemize}
    \begin{tabular}{ cc }
        {}&{4} \\ {R$\rightarrow$}&{3} \\ {}&{2} \\ {}&{1} \\ {}&{0} \\
    \end{tabular}
    \begin{tabular}{|p{0.75cm}|}
        \hline {} \\ \hline {20} \\ \hline {15} \\ \hline {10} \\ \hline {5} \\ \hline
    \end{tabular}
    \begin{tabular}{ c }
        {} \\ {} \\ {} \\ {} \\ {$\leftarrow$F} \\
    \end{tabular}
    

    \begin{itemize}
        \item Now we can move the rear to the next element without satisfying the underflow condition (front == rear).
        \item The queue is now at an overflow state.
    \end{itemize}
    \begin{minted}[autogobble]{c}
        CircularQueue.enqueue(25);
        // rear: 
    \end{minted}
    \begin{itemize}
        \item You can't enqueue anymore because all are occupied. 
    \end{itemize}
    \begin{tabular}{ cc }
        {R$\rightarrow$}&{4} \\ {}&{3} \\ {}&{2} \\ {}&{1} \\ {}&{0} \\
    \end{tabular}
    \begin{tabular}{|p{0.75cm}|}
        \hline {25} \\ \hline {20} \\ \hline {15} \\ \hline {10} \\ \hline {5} \\ \hline
    \end{tabular}
    \begin{tabular}{ c }
        {} \\ {} \\ {} \\ {} \\ {$\leftarrow$F} \\
    \end{tabular}
\end{center}
