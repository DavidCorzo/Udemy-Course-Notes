\subsection{Overview}
\begin{itemize}
    \item They are used to display data as output sotred in a variable.
    \item They specify the type of data of the variable to be displayed.
    \item Example: \verb|printf("%s", var);|
    \item printf is a function, a function is a block of code, a function can be invoqued and called to do something.
    \item When you print a variable you must specify the data type that will be outputed.
    \item printf takes two arguments, the first one is the text to be displayed, the variable will be formated to the \%s, or \%d for example, the second parameter is the variable.
    \item The percentage symbol is the format specifier, inmedietly after the percent symbol the letters represent an identity for a data type.
    \item Format specifiers:
        \begin{itemize}
            \item char $\rightarrow$ \verb|%c|
            \item \_Bool $\rightarrow$ \verb|%i, %u|
            \item short int $\rightarrow$ \verb|%hi, %hx, %ho|
            \item unsigned short int $\rightarrow$ \verb|%hu, %hx, %ho|
            \item int $\rightarrow$ \verb|%i, %x, %o|
            \item unsigned int $\rightarrow$ \verb|%u, %x, %o|
            \item long int $\rightarrow$ \verb|%li, %lx, %lo|
            \item unsigned long int $\rightarrow$ \verb|%lu, %lx, %lo|
            \item long long int $\rightarrow$ \verb|%lli, %llx, %llo|
            \item unsigned long long int $\rightarrow$ \verb|%llu, %llx, %llo|
            \item float $\rightarrow$ \verb|%f, %e, %g, %a|
            \item double $\rightarrow$ \verb|%f, %e, %g, %a|
            \item long double $\rightarrow$ \verb|%Lf, $Le, %Lg|
            \item strings $\rightarrow$ \verb|%s|
        \end{itemize}
    
    \item Format specifiers can be used more than once if you specify the data type, you can send more arguments, the new arguments ( 3 an so on ) will be more variables to be formated using the format specifiers.
    \item Width specifier, in floating point numbers and intigers values that have lots of decimals you can specify the cuantity of digits it will print out, example: 
        \begin{Verbatim}[breaklines=true, breakanywhere=true]
            int main(){
                float x = 3.999192;
                printf("%.2f", x)\neg // will print 3.9
            }
        \end{Verbatim}
    The width operator starts with a point and then a number that represents how many digits will be printed.
    
    \item This can be used for rounding floating points as well.
    \item 
\end{itemize}
