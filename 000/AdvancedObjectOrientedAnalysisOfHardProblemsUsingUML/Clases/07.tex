\section{Workshop overview}
\begin{itemize}
    \item Producing crystal clear models is a craftmanship. 
    \item Workshop tecniques:
        \begin{enumerate}
            \item Preparing for a workshop 
            \item Storytelling 
            \item Object diagrams examples 
            \item Pattern recognition 
            \item Pattern utilization 
            \item Process and concept modeling 
            \item Diagram composition 
        \end{enumerate}
    
    \item Three phases of the workshop:
        \begin{enumerate}
            \item Preparation 
            \item Facilitation 
            \item Documentation 
        \end{enumerate}
\end{itemize}


%----------------------------------------------------------------------------------------
\section{Workshop preparation}
\begin{itemize}
    \item You are the facilitator, you don't tell what to do, facilitate, you are the facilitator not the expert.
    \item Still, ou must understand what experts are trying to say, you are afterall the person who is going to make the diagrams.
    \item Hence, read and analyze as much you can find on the web sites, process descriptions, marketing material. 
    \item Interview the assignor:
        \begin{itemize}
            \item This is the one who is going to recieve the result.
            \item Why are we doing this? describe the purpose.
            \item What is the scope? state a domain. 
            \item As-is or to-be analysis? map the concepts that we are using, create new way of conceptual analysis for the future? you listen in the as-is and you propose in the to-be.
            \item Who should attend? 
            \item When it should be done? 
            \item Where should we do it? online, phone, in a room? what limitations? 
            \item Which methods should be applied? 
        \end{itemize}
    
    \item Workshop agenda:
        \begin{enumerate}
            \item Icebreaker: break the language breaker. For example, introduce yourself, everyone will follow the format of introduction, even if they know themselves for 15 years they will start to think, this icebreaker can't be stiff.
            \item Introduction slides:
                \begin{itemize}
                    \item Background and purpose.
                    \item Workshop goals.
                    \item Scope.
                    \item Brief introduction to concept.
                    \item Description of the process.
                \end{itemize}
            \item Workshop! 
            \item Wrap up: (have 15 minutes to make a summary etc.)
                \begin{itemize}
                    \item Summary 
                    \item Were the goals achivied 
                    \item Next steps 
                \end{itemize}
            \item Ask them for comments.
        \end{enumerate} 
    
    \item Steps: 
        \begin{enumerate}
            \item Analysis 
            \item Goals 
            \item Organization 
            \item Execution 
            \item Evaluation 
        \end{enumerate}
    
    \item You have diferent roles during the workshop:
        \begin{itemize}
            \item Domain experts: they are the persons who have the terminology and the specialized knowledge.
            \item Facilitator: You have the idea of concepts and everything. 
        \end{itemize}
    
    \item Do not give up until you as the facilitator, understands!
        \begin{itemize}
            \item A fuzzed speech is often a fuzzed thought.
            \item It is you that shoul deliver a coherent model, you is the one that is going to deliver this coherent model. 
            \item However, make sure that they not trying to describe how a clarinet sounds. If you get stuck, leave it and get back. Maybe there are prerequisites not yet known to you.
        \end{itemize}
\end{itemize}


%----------------------------------------------------------------------------------------
\section{Storytelling}
\begin{itemize}
    \item It's about telling stories, its a powerful way of knowledge aquisition. 
    \item It's a powerful way of organizing information. 
    \item The magical words are: ``let me tell you a story''
        \begin{enumerate}
            \item People start listening.
            \item A new entity is created: the story. 
            \item Listeners get emotionally involved.
            \item Create more persistent knowledge. 
            \item Foster the shared understanding. 
            \item Activation of prior knowledge. 
            \item Foster collective creativity.
        \end{enumerate}
    
    \item Conceptual modeling using Storytelling:
        \begin{itemize}
            \item The facilitator is not the story teller. 
            \item ``Today you the participant is going to prepare us a story''
            \item However, it's not just any stories. I am interested in what happens when a payment is sent invoice is recieved? 
            \item Then listen very carefully and model everything you hear.
            \item ``Sure, let me tell you a story... Payments are recieved to out bank accounts twice per day. Every payment has a reference text which we use to figure out which invoice the payment related to.''
            \item So a payment for an invoice is made by a payee and includes a payment reference. The is recieved at a bank account.  
        \end{itemize}
    
    \item Remember:
        \begin{itemize}
            \item You are not the expert, you should be asking the questions. 
            \item Make parables form other similar situations. 
            \item This is one way of trying to make inferences for checking if the same conceptualizations are being done. 
        \end{itemize}
    
    \item Question current stated model: 
        \begin{itemize}
            \item Make specific question of the explicit model you have created, they may've only given you one scenario out of five right? 
            \item Raise ``stupid'' questions: play the role of the donkey. 
                \begin{itemize}
                    \item ``I might be a bit stupid... but can anyone here tell me exactly what a train is''
                \end{itemize}
            
            \item State implications from the current model:
                \begin{itemize}
                    \item Can you verify this statement= 
                    \item ``All product prices must be overridden and connceted to an agreement price. '' 
                \end{itemize}
            
            \item Involve yourself as an actor in the actual story.
                \begin{itemize}
                    \item Let's say i'm the \placeholder{role in the story}, what would happen?  
                    \item Ok \placeholder{this} will happen.  
                \end{itemize}
        \end{itemize}
    
    \item The process:
        \begin{itemize}
            \item Start by: which stroies to use.
            \item Ask the participants to start telling stories. 
            \item Tell the stories. 
            \item Model the concepts. Glify a workshop tool.  
            \item Pause, associate.
            \item Raise control questions 
            \item Role playing 
            \item Derive and implicate 
            \item Recite where you stopped and start again. 
            \item You do this until you are done, do it in iterations. 
        \end{itemize}
\end{itemize}


%----------------------------------------------------------------------------------------
