\section{NOT NULL, UNIQUE}
\begin{itemize}
    \item \mintinline{sql}{NOT NULL} keyword is used to describe an attribute in a table which cannot be null, it has to have a value.
    \item \mintinline{sql}{UNIQUE} keyword is used to say that that attribute needs to be unique in that column or attribute:
        \begin{minted}[autogobble]{sql}
            USE giraffe;
            CREATE TABLE student (
                student_id INT,
                name VARCHAR(20) NOT NULL, -- This attribute cannot be null.
                major VARCHAR(20) UNIQUE, -- Whatever values this attribute hlds must be unique among all others.
                PRIMARY KEY(student_id)
            );

            INSERT INTO student VALUES(1, 'Jack', 'Biology'); -- OK
            INSERT INTO student VALUES(2, 'Kate', 'Sociology'); -- OK
            INSERT INTO student VALUES(3, NULL, 'Chemistry'); -- Error Code: 1048. Column 'name' cannot be null.
            INSERT INTO student VALUES(4, 'Jack', 'Biology'); -- Error Code: 1062. Duplicate entry 'Biology' for key 'student.major'
        \end{minted}
    
    \item A primary key can be thought of as an attribute that is \mintinline{sql}{NOT NULL} and \mintinline{sql}{UNIQUE}. 
    \item You can use the \mintinline{sql}{DEFAULT} to define default values to attributes in the situation no in which no value is provided:
        \begin{minted}[autogobble]{sql}
            CREATE TABLE student (
                student_id INT,
                name VARCHAR(20) ,
                major VARCHAR(20) DEFAULT 'undecided',
                PRIMARY KEY(student_id)
            );

            INSERT INTO student(student_id, name) VALUES(1, 'Jack'); -- 1, Jack, undecided
        \end{minted}
    \item You can use the \mintinline{sql}{AUTOINCREMENT} keyword to make a primary automatically increment when an object is added:
        \begin{minted}[autogobble]{sql}
            CREATE TABLE student (
                student_id INT AUTO_INCREMENT,
                name VARCHAR(20) ,
                major VARCHAR(20),
                PRIMARY KEY(student_id)
            );

            INSERT INTO student(name, major) VALUES('Jack', 'Biology'); -- 1,  Jack, Biology
            INSERT INTO student(name, major) VALUES('Kate', 'Sociology'); -- 2, Kate, Sociology
        \end{minted}
        \begin{itemize}
            \item Notice that we obtained the primary keys of 1 and 2 despite not telling SQL explicitly.
        \end{itemize}
\end{itemize}
