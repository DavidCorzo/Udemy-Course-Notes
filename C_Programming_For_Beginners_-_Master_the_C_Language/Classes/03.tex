\section{Comments}
\begin{itemize}
    \item Useful for documenting a program. 
    \item Useful for reminging. 
    \item Comments are ignored by the compiler. 
\end{itemize}

\subsection{Syntax}
\begin{itemize}
    \item Multiline comments: /*Comment*/
    \item Single line comments: // Comment 
\end{itemize}

\subsection{Use of comments}
\begin{itemize}
    \item Use comments in moderation. 
    \item Self documenting comments by using meaningful names for variables and  other elements of the code. 
\end{itemize}


%----------------------------------------------------------------------------------------
\section{The proprocessor}
\begin{itemize}
    \item It's part of the compiling phase. 
    \item This is done before compiling any sourcode. 
\end{itemize}

\subsection{Syntax}
\begin{itemize}
    \item Preprocessor statements are preceded by a \# sign. It must not include a space, it's the first non-space character on the line. 
    \item The \#include is a preprocessor directive. 
\end{itemize}

\subsection{Defining your own preprocessors}
\begin{itemize}
    \item For creating constants: \#define 
    \item For building your own library files: \#include 
    \item Make more powerful programs with the conditional \#ifdef, \#endif, \#else and \#ifndef 
\end{itemize}


%----------------------------------------------------------------------------------------
\section{The \#include statement}
\begin{itemize}
    \item It is a preprocessor directive. 
    \item The program won't work without it. 
    \item Tells the compiler to include the contents of the header file stdio.h.
\end{itemize}

\subsection{Header files}
\begin{itemize}
    \item Header files define information about some functions that are provided by that file. 
    \item stdio.h is standard library header and provides functionality for displaying output. 
    \item It for example contains the information for the compiler to know what printf means.
    \item stdio stands for standard input / output.
    \item Header files are case-sensitive.
    \item Two ways to include header files:  
        \begin{itemize}
            \item The \mintinline{c}{<stdio.h>} syntax is usually used for built-in functions.
            \item The \mintinline{c}{"something.h"} syntax is usually for user defined header files.
        \end{itemize}
    
    \item The \mintinline{c}{#define} and \mintinline{c}{#ifndef} protect against multiple inclusions of the header file for efficiency.
    \item The header files can have: \mintinline{c}{#define} directives, structure declarations, typedef statements, function prototypes, constants, etc.
    \item Executable code goes in a .c file not in a header file. 
    \item Header files are needed to create linking during compilation. 
\end{itemize}


%----------------------------------------------------------------------------------------
\section{Displaying output}
\begin{itemize}
    \item \mintinline{c}{printf()} is a standard library function. 
    \item Used to display program results, print values of variables, results of computations, used for debugging. 
\end{itemize}

%----------------------------------------------------------------------------------------
\section{Reading input from the terminal}
\begin{itemize}
    \item Used to ask a user for data. 
    \item \mintinline{c}{scanf()} is used for console reading, also to read data from files. 
    \item stdin standard input stream. 
    \item You must have a format specifier. 
\end{itemize}
\subsection{scanf()}
\begin{itemize}
    \item String followed by list of arguments. 
    \item \mintinline{c}{scanf()} function uses pointers.
\end{itemize}
Three rules for scanf:
\begin{enumerate}
    \item returns the number of items that it successfully reads.
    \item If you use \mintinline{c}{scanf()} to read a value for one of the basic variable types we've discussed, precede the variable name with a \& this is a pointer.
    \item If you use \mintinline{c}{scanf()} to read a string into a character array, don't use an \& this is a pointer.
\end{enumerate}
scanf takes two arguments: 
\begin{enumerate}
    \item The string with format specifiers.
    \item The \& to the variable. 
\end{enumerate}
\subsection{Peculiarities of the scanf}
\begin{itemize}
    \item The scanf() function uses white space (newlines, tabs and spaces) to decide how to divide the input into separate fields. 
    \item scanf() is the inverse of printf(), converts integers, floating-points, characters and C strings to text that is to be displayed onscreen
    \item When scanf() is executed the program is paused and you will be required to enter data.
    \item scanf() expects input in the same format as you provided \mintinline{c}{%s} and \mintinline{c}{%d}, you have to proide valid input like string then integer for \mintinline{c}{%s} \mintinline{c}{%d}.
    \item scanf() is problematic when you call it again and again do to the fact that it only reads up until a space and not a return, thus it gets messed up.
\end{itemize}
\subsection{Example}
\inputcode{\lang}{\code/scanf_func.c}
