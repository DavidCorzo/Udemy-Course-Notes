\begin{itemize}
    \item C suports diferent types of data.
    \item The basic data types in C are the:
        \begin{itemize}
            \item int 
            \item float 
            \item double 
            \item char 
            \item \_Bool
        \end{itemize}
    
    \item The diference between these types are the amount of memory that is going to be allocated to store them; this is important for the compiler and for the programmer.
    \item Difereces:
        \begin{itemize}
            \item The amount of memory.
            \item Machine dependent for memory allocation. A windows machine can allocate more or less memory depending on the machine.
        \end{itemize}
\end{itemize}
%%%%%%%%%%%%%%%%%%%%%%%%%%%%%%%%%%%%%%%%%%%%%%%%%%%%%%%%%%%%%%%%%%%%%%%%%%%%%%%%%%%%%%%%%%%%%%%%
\subsection{Int}
\begin{itemize}
    \item Data type to store integral values only.
    \item You can use negative values, it can store all the values of $\mathbb{Z}$.
    \item You can assign a hexadecimal number, to do this use \verb|int something = 0xFFEF0D;|.
    \item Rules:
        \begin{itemize}
            \item No embedded spaces are permitted between the digits.
            \item Big numbers can't be written with comma separators.
        \end{itemize}
\end{itemize}
%%%%%%%%%%%%%%%%%%%%%%%%%%%%%%%%%%%%%%%%%%%%%%%%%%%%%%%%%%%%%%%%%%%%%%%%%%%%%%%%%%%%%%%%%%%%%%%%
\subsection{Float}
\begin{itemize}
    \item Data type for numbers with decimal places.
    \item Floating-point such as 3.14
    \item You can represent scientific notation in C, an example is \verb|1.7e4| one point seven times 10 e to the power of four.
\end{itemize}
%%%%%%%%%%%%%%%%%%%%%%%%%%%%%%%%%%%%%%%%%%%%%%%%%%%%%%%%%%%%%%%%%%%%%%%%%%%%%%%%%%%%%%%%%%%%%%%%
\subsection{Double}
\begin{itemize}
    \item Very similar to the float, but can store a bigger float.
    \item A double allocates two times the memory allocated for the float.
    \item All floating point constant are taken as doubles by the compiler.
    \item To make somehting a float append f at the end ot the decimal number, 14.5f.
\end{itemize}
%%%%%%%%%%%%%%%%%%%%%%%%%%%%%%%%%%%%%%%%%%%%%%%%%%%%%%%%%%%%%%%%%%%%%%%%%%%%%%%%%%%%%%%%%%%%%%%%
\subsection{\_Bool}
\begin{itemize}
    \item Data types to store the values true or false, 0 or 1, it's for binary choices.
    \item Will store a true or false value.
    \item 0 is used to indicate a false value, 1 indicates a true value.
    \item They are stored as 0 or 1 not as true or false.
    \item With C89 compiler you can use bool prefix for this data type.
\end{itemize}
%%%%%%%%%%%%%%%%%%%%%%%%%%%%%%%%%%%%%%%%%%%%%%%%%%%%%%%%%%%%%%%%%%%%%%%%%%%%%%%%%%%%%%%%%%%%%%%%
\subsection{Other data types}
\begin{itemize}
    \item There are other data types that allows you to have a more specified type of int for example, use:
        \begin{itemize}
            \item short
            \item long
            \item unsigned
        \end{itemize}
    
    \item This is an extra adjetive that allows you to be more eficient.
    \item Short int uses less memory than an int.
    \item Long use more memory than an int.
    \item A double is a type of long float.
    \item You can use long long adjectives.
    \item You can use them like this:
        \begin{Verbatim}[breaklines=true, breakanywhere=true]
            long int = 98L; 
        \end{Verbatim}
    
    \item If you dont want an int to store a negative number you can use the unsigned adjective, this is in theory, some compilers don't enforce it. Basically to extend the int to be $\mathbb{Z}^{+}$.
    \item You can combine these such as short int, signed short, long unsigned.

\end{itemize}
