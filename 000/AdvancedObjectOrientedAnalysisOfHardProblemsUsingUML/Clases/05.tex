\section{Lessons Learned}
\begin{itemize}
    \item Let's do some analysis: we now have the tools to understand what is going on in communication between people.
    \item Let's use them to make sense situations where communication ``waste'' is produced. 
    \item And see how the waste can be eliminated using concept modelling.
    \item This is lean thinking, which persues the elimination of ``muda'' or waste.
    \item How to use the tools now learned to eliminate waste.
    \item Case studies:
        \begin{itemize}
            \item The meaning of breakdown 
            \item The slow knowledge 
            \item The language legacy 
            \item The rigid IT system
            \item The lack of creative leaps
        \end{itemize}
\end{itemize}


%----------------------------------------------------------------------------------------
\section{Case studies: Meaning breakdown}
\begin{itemize}
    \item Let's say:
        \begin{itemize}
            \item A group of people are discusing a problem, thus an agrement is created.
            \item Do they think that they agree or do they actually agree? 
        \end{itemize}
    
    \item They are using the same references to concepts but who knows if they all have the same concept.
    \item Do they think they agree? 
        \begin{itemize}
            \item Taking in to account the diferences in concepts, in plain sight nothing indicates disagreement, but some can have diferent concepts of strategy or action plan.
        \end{itemize}
    \item So:
        \begin{itemize}
            \item How many agreements will the customer need to sign? 
            \item What happens if we just broker a service and ``we'' are not the supplier of the service? 
            \item How many agreement templates need to be developed? 
            \item This is the meaning breakdown, this produces waste, they sit down and try to explain their concepts for hours and hours, and never reach anything of an agreement because they all have diferent conceptualizations of the same terms they are using.
        \end{itemize}
\end{itemize}


%----------------------------------------------------------------------------------------
\section{Slow knowledge acquisition}
\begin{itemize}
    \item Let's say:
        \begin{itemize}
            \item People are discussing solutions in a bussiness.
            \item Someone sugests to increase some product sales.
            \item He is answered yes I will do that.
            \item The conceptualizations are diferent, though the conceptualizartion is very simple, they are diferent.
            \item Something goes wrong, then waste comes, then the agreement of the persons are diferent conceptualizations.
        \end{itemize}
    \item The conceptual map gives us easier agreement and more precise conceptualization of solution:
        \begin{itemize}
            \item You need to consider all the premises.
            \item Then connect all the ideas.
            \item Define the types, the relations and the properties of each step of conception in the conceptual map.
            \item What happens if this? what happens if that? evaluate.
            \item Adverbs, they are very ambigous, how we describe actions can invite easy mistake.
        \end{itemize}
    
    \item So many situations where you need to gain knowledge very fast and modeling and visualizing can enable us to do so.
\end{itemize}


%----------------------------------------------------------------------------------------
\section{The language legacy}
\begin{itemize}
    \item Language gets stuck in conversation, it's stuck.
    \item Let's say:
        \begin{itemize}
            \item Implicitly declared set of status function concepts.
            \item Persons that declared the status functions are long gone.
            \item For strategic reasons \placeholder{this happens} 
            \item In early conceptualization of a concept the terms where changed, so they are refering to diferent objects with diferent properties and concepts and instances with the same term.
            \item Until you see that the object is another you're stuck.
            \item The keep referring to and trating this crucial objecto as another term.
            \item This creates waste, lots of waste.
            \item Example of customer as provider.
        \end{itemize}
    
    \item New status functions implied in the strategy must be declared explicitly by the agent with the power and collectively accepted by the organization.
        \begin{itemize}
            \item Agrement 1: customer agrement, make the declaration explicitly.
            \item Agrement 2: broker agrement, who is the broker, what is his role.
        \end{itemize}
    
    \item You can get stuck in an old language and how that sometimes can prevent you from implementing a new business strategy.
\end{itemize}


%----------------------------------------------------------------------------------------
\section{The rigid information systems}
\begin{itemize}
    \item Group of people that are sketching out the idea of a new project:
        \begin{itemize}
            \item The model of the project is encoded in the IT system and database, the service is becoming very popular, the larger it grows the harder it becomes to change its structure.
            \item Years after... they want to implement new features to the model, now implementing this would take too much work, adding this ``modifiability'' from the beginning would have been a small thing, adding it now results in costs that are exhorbitant.
            \item This is waste! 
        \end{itemize}
\end{itemize}



%----------------------------------------------------------------------------------------
\section{The creative leaps catalyst}
\begin{itemize}
    \item This is a sucess story: 
        \begin{itemize}
            \item Starting position - lack of a common language (and concepts) 
            \item Study of other domains and gain new knowledge.
            \item Let's say that we study lean, and we connect the patterns that we study to my challenge, then a framework is formed.
            \item Not treating this creative process the same way is waste.
        \end{itemize}
\end{itemize}


%----------------------------------------------------------------------------------------
\section{Early customer acceptance language}
\begin{itemize}
    \item Once you create status functions you need to make sure your organization does understand your model.
    \item Declaring the status functions is not enough to make their instances ``alive'' in the collective intentionality... 
    \item This declared set of status functions needs to be ``rooted'', incorporated and accepted in the user's existing intentional network and background... 
    \item Once you have learned the language you will be able to communicate better.
    \item Teach your language to real consumers as early as posible! 
    \item You will neew to adjust your status functions for best optimal fit with existing user's network and background to eliminate waste.
    \item Pick you language and test it on real customers, try to fit the in to the collective intentionality if they don't understand it you need rework it and make it clearer. This will eliminate waste. 
\end{itemize}


%----------------------------------------------------------------------------------------
\section{Final conclusions}
\begin{itemize}
    \item Concept of concept: we create our social reality when creating function concepts, declaring instances of them and getting people to recognize these as existing. This is done using language. 
    \item In the same way as a hammer can be too heavy, a malfunction conceptual system os status functions could prevent growth, development and lead to disasters. 
    \item When we communicate we often leave out the network and background to a level that we think is common with the people that we communicate. This is natural, if we said everything in our collective intentionality it would take ages to explain ``I'm taking a walk'', we tend however to leave out a little bit too much. Try to be too detailed in explaining the collective intentional network to understand your domain. Say a littlebit more about the network you want to create.
    \item This process is so deeply part of what is to be human that we often neglect it's logical mechanisms and lack the tools to fiz the problems that arise frome it.
    \item What to do then? Eliminate communication waste,
        \begin{itemize}
            \item Pay attention to implicit differences.
            \item Visualise conceptual structures. Doing it visualy helps the brain understand.
            \item Reconsider the shared legacy of concepts when game shifts occur.
            \item Make the language flaxible for future changes. Make you construct flexible. 
            \item Find conceptual similarities between diferent domains. This jump starts the creative leap process.
            \item Get early ``acceptance'' from all users of the language, get feedback in small iterations.
            \item Refine the language in iterations to let it ground itself in the collective network of intentionality. People have diferent conceptualizations about the same world, with tools such as conceptual maps it's easier to see te diferences.
        \end{itemize}
    \item Conceptual maps help a lot to eliminate communication waste.
\end{itemize}





%----------------------------------------------------------------------------------------
\section{Resources}
\begin{itemize}
    \item The lean startup, Eric Ries
\end{itemize}
