\section{Overview}
\begin{itemize}
    \item Statements are executed in the order they appear. 
    \item Control flow statements will break this order based on decision-making. 
    \item Code can be repeated; branching statements can execute only certain code, these are used inside loops; decision-making.  
\end{itemize}
\subsection{Decision-making}
\begin{itemize}
    \item Asks a question inside, the answer will respond. 
    \item If the condition is true a statement is executed, if not then the code isn't executed.  
\end{itemize}
If statements: 
\begin{center}
    \begin{tabular}{ |p{5cm}|p{7cm}| }
        \hline
            Statement & Description \\
        \hline \hline
            if statement & If statement consists of a boolean expression, followed by on or more statements. \\ 
        \hline
            if...else statement & if statements can be followed by else statements which executes when the boolean expression of the if statement is false \\ 
        \hline
            nested if statements & multiple conditions are evaluated, if statement inside a statement. \\ 
        \hline
    \end{tabular}
\end{center}
\subsection{Repeating code}
\begin{itemize}
    \item Loop statements allows us to repeat code. 
    \item When execution leaves a scope, all automatic objects that were created in that scope are destroyed. 
\end{itemize}
Loops   
\begin{center}
    \begin{tabular}{ |p{5cm}|p{7cm}| }
        \hline
            Loop type & Description \\
        \hline
            while & repeats until condition \\ 
            for & repeats a number of times \\ 
            do...while & while loop but with the exception that it tests the condition at the end of the loop body \\ 
            nested loops & using one inside another \\ 
        \hline
    \end{tabular}
\end{center}


%----------------------------------------------------------------------------------------
\section{If statements}
\subsection{if else}
\begin{itemize}
    \item Syntax is: \mintinline{c}{if (expression) { /*program statement */};}
    \item For compound statements you need to surround the program statement with curly braces, if only one statement is used it's okay to use it without the curly braces. 
\end{itemize}
If with an else: 
\begin{itemize}
    \item An addition to the if block, you can add an else clause. 
    \item Syntax is: \mintinline{c}{if (expression) statement; else statement;} 
\end{itemize}
\subsubsection{Example}
\inputcode{\lang}{\code/if_else_statements.c}

\subsection{else if}
\begin{itemize}
    \item To ask multiple questions. 
    \item Syntax is: 
        \begin{minted}[autogobble]{c}
            if (expression) /* program statement */; 
            else if (expression) /* program statement */; 
            else /* program statement */; 
        \end{minted}
\end{itemize}
\subsubsection{Example}
\inputcode{\lang}{\code/else_if_statement.c}

\subsection{nested if-else statement}
\begin{itemize}
    \item If or if else statement inside another. 
    \item Syntax is: 
        \begin{minted}[autogobble]{c}
            if (expression) {
                if (expression) {
                    statements;
                }
            }
        \end{minted}
\end{itemize}

\subsection{The conditional operator (ternary operator)}
\begin{itemize}
    \item This is equivalent to an if else statement. 
    \item Syntax is: \placeholder{\mintinline{c}{condition}}? \mintinline{c}{expression1: expression2;}|
    \item First operand is placed before the ?, the second after the ? and before the : and the third one after the : 
    \item This is a short hand notation for if else statements, there are no compound statements here, only simple. 
    \item Example: 
        \begin{minted}[autogobble]{c}
            x = y > 7 ? 25 : 50; 
        \end{minted}
    result in x being set to 25 if y is greater than 7 or to 50 otherwise. 
\end{itemize}
\subsubsection{Example}
\inputcode{\lang}{\code/ternary_operator.c}


%----------------------------------------------------------------------------------------
\section{Switch statement}
\begin{itemize}
    \item Serve the same purpose as an if, the difference is that is more organized and made for multiple alternatives rather than using else if multiple times. 
    \item else ifs can be prone to errors. 
    \item Switch are more efficient than else if. 
    \item Syntax is: 
        \begin{minted}[autogobble]{c}
            switch (expression) {
                case val_1: 
                    program statement; break;
                case val_2: 
                    program statement; break;
                case val_n: 
                    program statement; break;
                default: 
                    program statement; break;
            }
        \end{minted}
    
    \item The expression in the switch argument is the thing to check for. Cases are like if (val). Default means else.
    \item Cases must be constants or constant expression. 
    \item  When more than one statement is included they don't have to have curly braces surrounding the statement. 
    \item The break statement indicates the termination of a particular case and terminates the switch statement. It jumps out. If the expression is not mutually exclusive (more than one case will be true), don't put the break statements, otherwise put them, keep in mind that there must be one case that terminates the switch statement. This can cause bugs. 
    \item If none of the cases are true the default block is executed. 
\end{itemize}
\subsection{Example}
\inputcode{\lang}{\code/switch.c}
\inputcode{\lang}{\code/another_switch.c}
%----------------------------------------------------------------------------------------
\subsection{goto statement}
\begin{itemize}
    \item Used for jumping to a specific line of code.
    \item Has two parts, the goto and the label name.
    \item Label is named following the same convention used in naming a variable. 
    \item Example: goto part2; part2 is a label and it has to be labeled for the compiler to know what line you mean. 
    \item You can jump all around in your code.
\end{itemize}
\subsubsection{Example}
\inputcode{\lang}{\code/goto.c}

%----------------------------------------------------------------------------------------
\section{Challenge calculate week pay}
\inputcode{\lang}{\code/challenge_week_pay.c}


%----------------------------------------------------------------------------------------
\section{For loops }
\begin{itemize}
    \item Allows us to repeat code, this is a counter controlled loop because the number of iterations are predefined. 
    \item Sentinel loops execute an undefined number of times until a certain condition is met. 
    \item For simple statements you can omit the braces.
    \item Below C99 you must declare the counter variable outside the for loop.
    \item Syntax is: \mintinline{c}{for (starting condition; continuation condition; action per iteration) {statements;}}
    \item Example:  
        \begin{minted}[autogobble]{c}
            for (int i, j = 2; i <= 5; ++i, j += 2) 
                printf("%5d",i*j);
        \end{minted}
    \item If the action per iteration is not put, the for loop becomes an infinite loop. 
    \item Something like: \mintinline{c}{for (;;){statements;}}
\end{itemize}
\subsection{Example}
\inputcode{\lang}{\code/infinite_loops.c}


%----------------------------------------------------------------------------------------
\section{While loop, do while}
\subsection{While loops}
\begin{itemize}
    \item Mechanism to execute a set of statements as long as a condition is met.
    \item Syntax is: \mintinline{c}{while (expression) {statements;}}
    \item You don't need to put curly braces when you only have one statement. 
    \item If the loop is true the loop has to have a breaking mechanism so that it doesn't become an infinite loop.
    \item While loops are called pretest loop, it executes if a condition pre-evaluated results in true.
\end{itemize}
\subsection{do-while loops}
\begin{itemize}
    \item In the do while loop, you execute the code at least once, while loops execute code if the condition is true, do-while loops are going to execute the code at least once, never zero, regardless of the condition being true or false. 
    \item After the first iteration it will become a while loop. 
    \item The condition is at the bottom. This is called a post-test loop.
    \item Syntax is: \mintinline{c}{do{statement;}while(condition)}
    \item Example: 
        \begin{minted}[autogobble]{c}
            do { 
                // prompt for password; 
                // read user input; 
            } while (/*input not equal to password*/);
        \end{minted}
\end{itemize}
\subsubsection{Example}
\inputcode{\lang}{\code/do_while.c}

\subsection{Which loop to use}
\begin{itemize}
    \item If you want to execute it at least once unconditionally, use the do-while. 
    \item if you want to execute it while some condition is true then use a regular while loop. 
    \item Its a matter of taste, what you can do in a for loop you can do in a while loop. 
    \item \mintinline{c}{for(;test;)} is the same as \mintinline{c}{while (test)}.
\end{itemize}

\subsection{For loop and while loop equivalents}
For equivalent in a while loop. 
\begin{minted}[autogobble]{c}
    initalize; 
    while (test){
        body; 
        update;
    }
\end{minted}
Same as: 
\begin{minted}[autogobble]{c}
    for (initialize; test; update){ 
        update;
    }
\end{minted}

\begin{itemize}
    \item For loops are appropriate when the loop involves initializing and updating a variable. 
    \item A while loop is better when the conditions are otherwise. 
    \item Use the while loop for logic controlled loops and the for loop for counter controlled loops. 
\end{itemize}
%----------------------------------------------------------------------------------------
\section{Nested loops and loop control - break and continue}
\subsubsection{Nested loops}
\begin{itemize}
    \item Nested loops are loops inside loops. 
    \item You can have a while loop inside a for loop and vice versa, etc. 
\end{itemize}
\subsubsection{Continue statements}
\begin{itemize}
    \item It will skip that iteration, if a condition is met you can skip the iteration. 
\end{itemize}
\subsubsection{Example}
\inputcode{\lang}{\code/continue.c}
PS you can iterate through enums. 

\subsection{Break statement}
\begin{itemize}
    \item Breaks are used to jump out of loops, they are a way to stop execution of the current loop. 
    \item If you are breaking out of a nested loop, the break will only affect the innermost loop containing the break.
    \item Used to break out of the loop if a condition is met. 
    \item Switch statements also use the break keyword, they do the same things. 
\end{itemize}
\subsubsection{Example}
\inputcode{\lang}{\code/break.c}

%----------------------------------------------------------------------------------------
\section{Challenge guess the number}
\inputcode{\lang}{\code/challenge_guess_num.c}

