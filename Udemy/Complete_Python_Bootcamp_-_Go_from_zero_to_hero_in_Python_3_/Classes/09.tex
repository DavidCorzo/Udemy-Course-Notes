\section{Generators}
\begin{itemize}
    \item Allows algorithms to only do what is necessary at a time.  
    \item They don't actually return a value and then exit, it will complete the task periodically.
    \item It's more memory efficient.
        \begin{minted}[autogobble]{python}
            def create_cubes(n):
                """ 
                This function will create an entire list in memory before doing anything else. 
                """
                result = []
                for x in range(n):
                    result.append(x**3)
                return result 

            for x in create_cubes(10):
                print(x)
            # output: 0
            # output: 1
            # output: 8
            # output: 27
            # output: 64
            # output: 125
            # output: 216
            # output: 343
            # output: 512
            # output: 729

            def create_cubes(n):
                """
                This function will produce the cubes according to when they are needed. 
                """
                for x in range(n):
                    yield x**3
            for i in create_cubes(10):
                print(x)
            # output: 0
            # output: 1
            # output: 8
            # output: 27
            # output: 64
            # output: 125
            # output: 216
            # output: 343
            # output: 512
            # output: 729

            def gen_fibon(n):
                """ Memory ineficient fib sequence """
                a,b = 1,1
                output = []
                for i in range(n):
                    output.append(a) 
                    a,b = b,a+b 
                return output
            for num in gen_fibon(10):
                print(num)
            # output: 1
            # output: 1
            # output: 2
            # output: 3
            # output: 5
            # output: 8
            # output: 13
            # output: 21
            # output: 34
            # output: 55

            def gen_fibon(n):
                """ Memory eficient fib sequence """
                a,b = 1,1
                for i in range(n):
                    yield a 
                    a,b = b,a+b 
            for num in gen_fibon(10):
                print(num)
            # output: 1
            # output: 1
            # output: 2
            # output: 3
            # output: 5
            # output: 8
            # output: 13
            # output: 21
            # output: 34
            # output: 55
        \end{minted}

    \item The range function is a generator. 

    \item next(\placeholder{iterable}): returns the next iteration of the generator object.   
        \begin{minted}[autogobble]{python}
            def simple_gen():
                for x in range(3):
                    yield x 
            for num in simple_gen():
                print(num)
            # output: 0
            # output: 1
            # output: 2

            g = simple_gen()
            g
            # output: <generator object simple_gen at 0x000001BB5D22D840>

            print(next(g))
            # output: 0
            print(next(g))
            # output: 1
            print(next(g))
            # output: 2
            print(next(g))
            # Traceback (most recent call last):
            #   File "<stdin>", line 1, in <module>
            # StopIteration

            s = "hello"
            for letter in s: 
                print(letter)
            next(s)
            # Traceback (most recent call last):
            #   File "<stdin>", line 1, in <module>
            # TypeError: 'str' object is not an iterator
        \end{minted}
    \item iter(\placeholder{iterable}): is a function that turns an iterable object in to a generator. 
        \begin{minted}[autogobble]{python}
            s_iter = iter(s) 
            next(s_iter)
            # output: 'h'
            next(s_iter)
            # output: 'e'
            next(s_iter)
            # output: 'l'
        \end{minted}
\end{itemize}

\subsection{Generator comprehension}
You can use comprehension to make generators just like list comprehension just with the diferent syntax, just change the \verb|[]| to \verb|()|.
\begin{minted}[autogobble]{python}
    square = (i**2 for i in range(1,11))
    print(square)
    # output: <generator object <genexpr> at 0x0000014A1427D840>
    for x in square:
        print(x)
    # output: 1
    # output: 4
    # output: 9
    # output: 16
    # output: 25
    # output: 36
    # output: 49
    # output: 64
    # output: 81
    # output: 100
\end{minted}
