\section{Understanding the problem}
\begin{itemize}
    \item To establish a precedence in mathematical operations, we use brackets, the program must check if the program's input has well-formed parenthesis and braces:
        \begin{align*}
            [\{(a+b)\}/k] \text{ Correct } \\ 
            )a+b( \text{ Incorrect } \\ 
            \{(a+b)/c] \text{ Incorrect } \\ 
        \end{align*}
    
    \item For checking this we need a stack.
    \item The stack comes in at: whenever you are getting an opening bracket in an expression while scanning the expression from left to right, push the bracket into the stack, and whenever you are pushing the closing bracket, check whether the top of the stack contains the opening match for this closing bracket or not. 
        \begin{itemize}
            \item In the expression $\{(a+b)/c]$ the $\{$ will be closed first.
        \end{itemize}
    
    \item What we'll do while scanning the expression, if we encounter an opening bracket we'll pushing it to the stack, whenever we encounter a closing bracket, we'll check if the top of the stack contains the opening bracket or not, if the top of the stack contains the opening match for this particular closing bracket, we'll pop the stack, and then continue scanning doing the same thing if we encounter another closing or opening bracket. 
    \item If the expression encounters a closing bracket and the stack is empty the that is an error. 
\end{itemize}

\subsection{Example}
\begin{itemize}
    \item Given a string s=$[2*3\{(3-1)/(5+7)\}-4]$ 
\end{itemize}
\begin{figure}[H]
    \begin{bytefield}{20}
        \bitheader{0-19} \\
        \bitbox{1}{[} &\bitbox{1}{2} &\bitbox{1}{*} &\bitbox{1}{3} &\bitbox{1}{\{} &\bitbox{1}{(} &\bitbox{1}{3} &\bitbox{1}{-} &\bitbox{1}{1} & \bitbox{1}{)} & \bitbox{1}{/} & \bitbox{1}{(} & \bitbox{1}{5} & \bitbox{1}{+} & \bitbox{1}{7} & \bitbox{1}{)} & \bitbox{1}{\}} & \bitbox{1}{-} & \bitbox{1}{4} & \bitbox{1}{]} 
    \end{bytefield}
\end{figure}

\begin{minted}[autogobble]{c}
    Stack sp;
\end{minted}

\begin{itemize}
    \item When scanning position 1, we push the opening brace to the stack.
        \mint{c}{push(&sp,s[0]);}
    
    \item Again when position 4, we push:
        \mint{c}{push(&sp,s[4]);}
    
    \item Again at position 5:
        \mint{c}{push(&sp,s[5]);}
    
    \item The stack looks like this.
        \begin{tabular}{|c|c|}
            2 & ( \\ 
            1 & \{ \\ 
            0 & [ \\ 
        \end{tabular}
    
    \item When encountering the closing parenthesis at position 9 we check if the top of the stack is an opening parenthesis. It indeed is, thus we pop the parenthesis from the stack.
        \mint{c}{pop(&sp); // pops (}
    
    \item Encountering position 11, we push the brack to the stack. 
        \mint{c}{push(&sp,s[11]);}
    
    \item Encountering position 15, we check the top of the stack for an opening parenthesis, then we pop.
        \mint{c}{pop(&sp); // pops (}
    
    \item Encountering closing \} on position 16, we check if the top is \{, then pop if it is.
        \mint{c}|pop(&sp); // pops {|
    
    \item Encountering position 19, we check the top for [ then we pop if it is that. 
    \item The stack is in underflow now. 
\end{itemize}

\subsection{Errors}
\begin{itemize}
    \item In the above example no errors exits, an error will occur when a closing bracket does not match an opening bracket in the stack, or when a closing bracket is encountered with the stack being empty.
    \item Given a string s=$3*[(8+9)/9-2\}/5$  
        \begin{figure}[H]
            \begin{bytefield}{15}
                \bitheader{0-14} \\
                \bitbox{1}{3} & \bitbox{1}{*} &\bitbox{1}{[} &\bitbox{1}{(} &\bitbox{1}{8} &\bitbox{1}{+} &\bitbox{1}{9} &\bitbox{1}{)} &\bitbox{1}{/} &\bitbox{1}{9} &\bitbox{1}{-} &\bitbox{1}{2} &\bitbox{1}{\}} &\bitbox{1}{/} &\bitbox{1}{5} 
            \end{bytefield}
        \end{figure}
    
    \item The position 12 will cause an error because a closing brace is not matched by an opening brace at the top of the stack.
\end{itemize}


%----------------------------------------------------------------------------------------
\section{Developing the algorithm for bracket checking}
Pseudo-code:
\begin{lstlisting}
    error = false
    while Not end of the expression:
        next_char = next character of input expression; 
        if next_char == '(' OR next_char == '{' or next_char == '[' then: 
            push(STACK,next_char)
        else if next_char == ')' OR next_char == '}' OR next_char == ']' then:
            if isEmpty(STACK) then: 
                error = TRUE 
                break while 
            else if isOpenningMatch(stacktop(STACK), next_char) then: 
                pop(STACK)
            else
                error = TRUE 
                break while 
            end if
        end if 
    end while

    if !error and !isEmpty(STACK) then:
        error = TRUE 
    end if 

    if error then: 
        print "The input expression does not contain well formed brackets."
    else: 
        print "The input expression is well formed."
    end if 
\end{lstlisting}
\begin{itemize}
    \item Without the stack this had not been able to be implemented as easily as it did.  
\end{itemize}


%----------------------------------------------------------------------------------------
\section{Implementation of parenthesis checking program}
\inputcode{c}{\code/braces_checking_w_stack.c}

