\section{Advanced Numbers}
\subsection{Hexadecimal numbers}
\begin{minted}[autogobble]{python}
    hex(12)
    # output: '0xc'
    hex(512)
    # output: '0x200'
\end{minted}

\subsection{Binary numbers}
\begin{minted}[autogobble]{python}
    bin(1234)
    # output: '0b10011010010'
    bin(128)
    # output: '0b10000000'
    bin(512)
    # output: '0b1000000000'
\end{minted}

\subsection{Pow function }
\begin{minted}[autogobble]{python}
    pow(2,4) # With three args the third one is mod. 
    # output: 16
\end{minted}


\subsection{Abs function}
\begin{minted}[autogobble]{python}
    abs(-2)
    # output: 2
\end{minted}


\subsection{Round function}
\begin{minted}[autogobble]{python}
    round(3)
    # output: 3.0
    round(3.9)
    # output: 4
    round(3.141592,2)
    # output: 3.14
\end{minted}


%----------------------------------------------------------------------------------------
\section{Advanced strings}
\begin{minted}[autogobble]{python}
    s = 'hello world'
    s.capitalize()
    # output: 'Hello world'
    s.upper()
    # output: 'HELLO WORLD'
    s.lower()
    # output: 'hello world'
    s.count('o')
    # output: 2
    s.find('o')
    # output: 4
    s.center(20,'z')
    # output: 'zzzzhello worldzzzzz'
    'hello\thi'.expandtabs()
    # output: 'hello    hi'
    s.isalnum()
    # output: True
    s.isalpha() 
    # output: True 
    s.islower()
    # output: True
    s.isspace()
    # output: False 
    s.istitle()
    # output: False 
    s.isupper()
    # output: False
    s.endswith('o')
    # output: True
    s.split('e')
    # output: ['h','llo']
    s = 'hihihiiiiihihii'
    s.partition('i') # splits only the first instance
    # output: ('h', 'i', 'hihiiiiihihii')
\end{minted}


%----------------------------------------------------------------------------------------
\section{Advanced sets}
\begin{itemize}
    \item \placeholder{set1}.add(\placeholder{element}): adds elements to a set, if they are not already in there. 
        \begin{minted}[autogobble]{python}
            s = set()
            s.add(1)
            s.add(2)
            s
            # output: {1,2}
            s.add(2)
            s 
            # output: {1,2}
        \end{minted}
        
    \item \placeholder{set1}.clear(): clears the set of all its elements. 
        \begin{minted}[autogobble]{python}
            s.clear()
        \end{minted}
        
    \item \placeholder{set1}.copy(): copies all elements in set to another set. 
        \begin{minted}[autogobble]{python}
            s = {1,2,3}
            sc = s.copy()
            sc
            s.add(4)
            # output: {1,2,3}
        \end{minted}
        
    \item \placeholder{set1}.difference(\placeholder{set2}): returns a set containing the differences between set1 and set2.
        \begin{minted}[autogobble]{python}
            s.difference(sc)
            # output: {4}
        \end{minted}
        
    \item \placeholder{set1}.difference\_update(\placeholder{set2}): it's the same as the .difference() only that the returning set will be assigned to set1.
        \begin{minted}[autogobble]{python}
            s1 = {1,2,3}
            s2 = {1,4,5}
            s1.difference_update(s2) # does the same as diference() just that the return from the diference() is applied to s1
            s1 
            # output: {2,3}
        \end{minted}
        
    \item \placeholder{set1}.discard(\placeholder{element}): eliminates an element from the set.
        \begin{minted}[autogobble]{python}
            s
            # output: {1,2,3,4}
            s.discard(2)
            # output: {1,3,4}
        \end{minted}
        
    \item \placeholder{set1}.intersection(\placeholder{set2}): returns the elements that are common in both sets, or the intersection of the two sets. 
        \begin{minted}[autogobble]{python}
            s1 = {1,2,3}
            s2 = {1,2,4}
            s1.intersection(s2) # elements that are common to both of the sets
            # output: {1, 2}
        \end{minted}
        
    \item \placeholder{set1}.isdisjoint(\placeholder{set2}): returns True if the sets are disjoint, which means that the sets don't have anything in common. 
        \begin{minted}[autogobble]{python}
            s1 = {1,2}
            s2 = {1,2,4}
            s3 = {5}
            s1.isdisjoint(s2) # returns True if they don't have anything in common. 
            # output: False
        \end{minted}
        
    \item \placeholder{set1}.issubset(\placeholder{set2}): returns True if set2 is a super set of set1, or set1 is contained in set2.
        \begin{minted}[autogobble]{python}
            s1.issubset(s2) # {1,2} is a subset of {1,2,4} returns true
            # output: True
        \end{minted}
        
    \item \placeholder{set1}.issuperset(\placeholder{set2}): the complete inverse of .issubset, here if set2 is a super set of set1 returns true.
        \begin{minted}[autogobble]{python}
            s2.issuperset(s1) # is the inverse of the issubset()
            # output: True
        \end{minted}
        
    \item \placeholder{set1}.symmetric\_difference(\placeholder{set2}): returns the elements that only show up in one of the sets.
        \begin{minted}[autogobble]{python}
            s1.symmetric_difference(s2) # the elements that are only in one of the sets.
            # output: {4}
        \end{minted}
        
    \item \placeholder{set1}.union(\placeholder{set2}): returns the union or the elements that are in either set, all of the ven diagram. 
        \begin{minted}[autogobble]{python}
            s1.union(s2) # elements that are in either set. 
            # output: {1,2,4}
        \end{minted}
        
    \item \placeholder{set1}.update(\placeholder{set2}): it's the same as .union() but the returning set is stored in s1.
        \begin{minted}[autogobble]{python}
            s1.update(s2) # it's the same as the union, only that the result is stored in s1. 
        \end{minted}
\end{itemize}   


%----------------------------------------------------------------------------------------
\section{Advanced Dictionaries}
Dictionary comprehension: 
\begin{minted}[autogobble]{python}
    d = {x:x**2 for x in range(10)}
    d
    # output: {0: 0, 1: 1, 2: 4, 3: 9, 4: 16, 5: 25, 6: 36, 7: 49, 8: 64, 9: 81}
    d = {k:v**2 for k,v in zip(['a','b'],range(10))}
    d 
    # output: {'a': 0, 'b': 1}
\end{minted}

%----------------------------------------------------------------------------------------
\section{Advanced lists}
\begin{minted}[autogobble]{python}
    x = [1,2,3]
    x.extend([4,5])
    x 
    # output: [1, 2, 3, 4, 5]

    x.index(2) # you will get an error if the index isn't in list
    # output: 3
    x.insert(2,'inserted')
    x 
    # output: [1, 2, 'inserted', 3, 4, 5]

    x.remove('inserted') # removes the forst occurance of a value
    x
    # output: [1, 2, 3, 4, 5]

    x.reverse() 
    x 
    # output: [5, 4, 3, 2, 1]

    x.sort()
    x
    # output: [1, 2, 3, 4, 5]
\end{minted}
