\section{Polish \& Reverse polish notations}
\subsection{Polish notation}
\begin{itemize}
    \item In order to evaluate expressions. For example $(a+b)*c$. 
        \begin{itemize}
            \item We must first understand precedence, which means, the parenthesis are done first because it has a higher precedence, what ever is inside the parenthesis will be done first and then multiplied by $c$.
        \end{itemize}
    
    \item There are a handful of notations, the above example $(a+b)*c$ is called the infix notation. 
        \begin{itemize}
            \item There are other notations such as polish and reverse polish notation. 
        \end{itemize}
    \item In order to convert $a+b$  notation to polish notation:
        \begin{center}
            \begin{figure}[H]
                \begin{tikzpicture}[auto]
                    \node [] (1) at (0,0) {$a+b$}; 
                    \node [anchor=west] (2) at (2,0.75) {$+ab$ \quad prefix/polish}; 
                    \node [anchor=west] (3) at (2,-0.75) {$ab+$ \quad post-fix/reverse polish}; 

                    \path [line] (1) -- (2);
                    \path [line] (1) -- (3);
                \end{tikzpicture}
            \end{figure}
        \end{center}
    
    \item For evaluating infix expressions complexity tends to rise, since you need to keep details like what is being evaluated and so on, this complexity diminishes with the prefix and post fix notations. This is why we must aspire to make a converter of infix to prefix or post fix notation. 
        \begin{itemize}
            \item This is beneficial and reduces complexity. 
        \end{itemize}
    
    \item Example: given the infix expression $(a+b)*c$ find the equivalent in terms of a prefix expression.
        \[
          \overbrace{(a+b)}^{+ab}*c = \overbrace{+ab}^{r}*c = r * c = *rc = *+abc
        \]
        \begin{itemize}
            \item $*+abc$ this is going to be read from right to left, the first operator is going to be $+$ this allows precedence and allows for no use of braces. 
        \end{itemize}
\end{itemize}

\subsection{Reverse polish notation}
\begin{itemize}
    \item In order to convert from infix to polish notation using the example $(a+b)*c$: 
        \[
            \overbrace{(a+b)}^{ab+}*c = \overbrace{ab+}^{r}*c = r*c = rc* = ab+c*
        \]
    
    \item In the reverse polish notations reading it from left to right the first operation will be $+$. 
\end{itemize}


%----------------------------------------------------------------------------------------
\section{Understanding precedence of operators, conversion idea - infix to prefix/post-fix}
\begin{itemize}
    \item Let \$ be a substitute for the character \verb|^|, to represent exponential operations. 
    \item Remember the precedence of operators go as follows:
        \begin{center}
            \begin{tabular}{ |c|c| }
                \hline
                $\uparrow$  & () \\
                \hline
                $\uparrow$  & *, / \\
                \hline
                $\uparrow$  & +, - \\
                \hline
            \end{tabular}
        \end{center}

    
    \item Given the following expression: $a+b\$(c-d)k\$p$ convert to polish notation. 
        \begin{multline*}
            a+b\$(c-d)k\$p = a*b\$-cd/k\$p = a*\$-cd/k\$p = a*\overbrace{\$b-cd}^{r_1}/\overbrace{\$kp}^{r_2} \\ 
            = a*r_1/r_2 = *ar_1/r_2 = *ar_1/r_2 = /*ar_1r_2 = */a\$b-cd\$kp
        \end{multline*}
        \begin{itemize}
            \item The equivalent expression in prefix notation reading from right to left $*/a\$b-cd\$kp$
        \end{itemize}
    
    \item Given the same expression convert it to post-fix.
        \[
            a+b\$(c-d)k\$p = a*b\$cd-/k\$p = a*bcd-\$/kp\$ = abcd-\$*/kp\$ = abcd-\$*kp\$/
        \]
\end{itemize}


%----------------------------------------------------------------------------------------
\section{How to evaluate polish or reverse polish notation}
\begin{itemize}
    \item Iterating a string \verb|"24+32-*"| written in reverse polish notation. Program to implement has to iterate and will for this example only work for single digit numbers.
    \item Reverse polish notation, remember, is read from left to right.
        \begin{center}
            \begin{bytefield}{7}
                \bitheader{0-6} \\ 
                \bitbox{1}{2} & 
                \bitbox{1}{4} & 
                \bitbox{1}{+} & 
                \bitbox{1}{3} & 
                \bitbox{1}{2} & 
                \bitbox{1}{-} & 
                \bitbox{1}{*} 
            \end{bytefield}
        \end{center}
    
    \item Position 0 and 1 are operands and must be pushed to the stack. 
    \item Position 2 is, however, an operator, poping the stack twice will allow us to perform the operation of adding the operands, the result of both is pushed to the stack. 
    \item Position 3 and 4 are operands, the operator on position 5 pops the stack twice, subtracts the operands and pushes the result to the stack. 
    \item The position 6 operand pops the stack twice and performs the operation, pushing the value to the stack, poping the stack one time will reveal the final result. 
    \item For the polish notation start at the end and do the same thing. 
\end{itemize}


%----------------------------------------------------------------------------------------
\section{Algorithm for evaluating post-fix expression}
\begin{algorithm}[H]
    \SetAlgoLined
    \large
    initialize(stack)\;
    \While{not end of the post-fix string}{
        next\_current = gets the next token from the string\;
        \uIf{next\_token is an operand}{
            push(stack, next\_token)\;
        } \uElseIf{next\_token is an operator}{
            operand1 = pop(stack)\; 
            operand2 = pop(stack)\; 
            result = operate(operand2, operand1, next\_token)\; 
            push(stack, result)\; 
        }
    }
    print pop(stack)\; 
    end procedure interpret post-fix expressions\;
\caption{Interpret post-fix expressions pseudo-code}
\end{algorithm}

%----------------------------------------------------------------------------------------
\section{Implementing evaluation of post-fix expressions with the C programming language}
\inputcode{c}{\code/postfix_eval.c}



%----------------------------------------------------------------------------------------
\section{Understanding the precedence function}
\begin{itemize}
    \item The precedence function takes two arguments, the first argument is the stack top, the second is the token, the function returns true if the first argument (stack top) has equal or higher precedence than the token. 
    \item The function will return false if the token has higher precedence than the second.
    \item The following rules apply here: (op being any operator encountered)
        \begin{itemize}
            \item Any operator existing in the stack compared with the \verb|(| character will return false, meaning the \verb|(| has higher precedence than any other operator.
                \begin{minted}[autogobble]{c}
                    precedence(stack_top,'('); // false
                \end{minted}
            
            \item If the stack top is \verb|(| character, any operator stored in token has higher precedence, thus the function will return false.
                \begin{minted}[autogobble]{c}
                    precedence('(',token); // false
                \end{minted}
            
            \item If the precedence function has a token or second argument equal to the character \verb|)|, this means which ever character stored in the stack top has a higher precedence than closing parenthesis, except the case when the stack top is \verb|(| opening parenthesis.
                \begin{minted}[autogobble]{c}
                    precedence(stack_top,'('); // true
                    // except: stack_top = '('
                \end{minted}
            
            \item If the stack top is \verb|(| opening parenthesis, and the token is \verb|)| closing parenthesis, this is false.
                \begin{minted}[autogobble]{c}
                    precedence('(',')'); // false
                \end{minted}
        \end{itemize}
    
    \item Remember the order of operations or precedence is as follows in mathematics:
        \begin{enumerate}
            \item Anything within parenthesis.
            \item Exponentiation and root extraction. 
            \item Multiplication and division.
            \item Addition and subtraction.
        \end{enumerate}
    
    \item For all intents and purposes for this program if the precedence is the same the precedence function will return true as if one were greater than the other. The following examples apply:
        \begin{center}
    \begin{tabular}{ |l|l| }
        \hline
            False scenarios & True scenarios \\ 
        \hline
            \mintinline{c}{precedence('+','/') = false;} & \mintinline{c}{precedence('+','+') = true;}  \\ 
        \hline
            \mintinline{c}{precedence('+','$') = false;} & \mintinline{c}{precedence('+','-') = true;}  \\
        \hline
            \mintinline{c}{precedence('+','(') = false;} & \mintinline{c}{precedence('+',')') = true;}  \\
        \hline
            \mintinline{c}{precedence('-','*') = false;} & \mintinline{c}{precedence('-','+') = true;}  \\
        \hline
            \mintinline{c}{precedence('-','/') = false;} & \mintinline{c}{precedence('-','-') = true;}  \\
        \hline
            \mintinline{c}{precedence('-','$') = false;} & \mintinline{c}{precedence('-',')') = true;}  \\
        \hline
            \mintinline{c}{precedence('-','(') = false;} & \mintinline{c}{precedence('*','+') = true;}  \\
        \hline
            \mintinline{c}{precedence('*','$') = false;} & \mintinline{c}{precedence('*','-') = true;}  \\
        \hline
            \mintinline{c}{precedence('*','(') = false;} & \mintinline{c}{precedence('*','*') = true;}  \\
        \hline
            \mintinline{c}{precedence('/','$') = false;} & \mintinline{c}{precedence('*','/') = true;}  \\
        \hline
            \mintinline{c}{precedence('/','(') = false;} & \mintinline{c}{precedence('*',')') = true;}  \\
        \hline
            \mintinline{c}{precedence('$','(') = false;} & \mintinline{c}{precedence('/','+') = true;}  \\
        \hline
            \mintinline{c}{precedence('(','+') = false;} & \mintinline{c}{precedence('/','-') = true;}  \\
        \hline
            \mintinline{c}{precedence('(','-') = false;} & \mintinline{c}{precedence('/','*') = true;}  \\
        \hline
            \mintinline{c}{precedence('(','*') = false;} & \mintinline{c}{precedence('/','/') = true;}  \\
        \hline
            \mintinline{c}{precedence('(','/') = false;} & \mintinline{c}{precedence('/',')') = true;}  \\
        \hline
            \mintinline{c}{precedence('(','$') = false;} & \mintinline{c}{precedence('$','+') = true;}  \\
        \hline
            \mintinline{c}{precedence('(','(') = false;} & \mintinline{c}{precedence('$','-') = true;}  \\
        \hline
            \mintinline{c}{precedence('(',')') = false;} & \mintinline{c}{precedence('$','*') = true;}  \\
        \hline
            \mintinline{c}{precedence(')','+') = false;} & \mintinline{c}{precedence('$','/') = true;}  \\
        \hline
            \mintinline{c}{precedence(')','-') = false;} & \mintinline{c}{precedence('$','$') = true;}  \\
        \hline
            \mintinline{c}{precedence(')','*') = false;} & \mintinline{c}{precedence('$',')') = true;}  \\
        \hline
            \mintinline{c}{precedence(')','/') = false;} & \mintinline{c}{precedence(')',')') = true;}  \\
        \hline
            \mintinline{c}{precedence(')','$') = false;} &  \\
        \hline
            \mintinline{c}{precedence(')','(') = false;} &  \\
        \hline
    \end{tabular}
\end{center}

\end{itemize}
%----------------------------------------------------------------------------------------
\section{Explaining how it works}
\begin{itemize}
    \item Given the expression $m+(a+((b-c)*(d+k)))\$(x+y)*p$; its worth noting that this expression can be interpreted as follows:
        \begin{align*}
            =\;&m+\overbrace{(a+((b-c)*(d+k)))}^{r}\$\overbrace{(x+y)}^{w}*p \\    
            =\;&m+r^w*p \\ 
        \end{align*}
    
    \item Let's index the string:
        \begin{center}
            \begin{tabular}{ c }
                \mintinline{c}{infix_string =} \\ \\ \\ \\ 
            \end{tabular}
            \begin{bytefield}{27}
                \bitheader{0-26} \\ 
                \bitboxes{1}{{m} {+} {(} {a} {+} {(} {(} {b} {-} {c} {)} {*} {(} {d} {+} {k} {)} {)} {)} {\$} {(} {x} {+} {y} {)} {*} {p} } \\
            \end{bytefield}
        \end{center}
    
    \item We'll next iterate throughout the entire length of the string, using a loop and a counter variable.
\end{itemize}

\subsection{Each index in the example}
\begin{center}
    \begin{longtable}{ |p{6cm}|p{11cm}| }
        \hline
        \hline
            \multicolumn{2}{|l|}{
                \begin{minipage}{\linewidth}
                {
                    \begin{center}
                        \begin{bytefield}{27}
                            \\
                            \large 
                            \bitheader{0-26} \\ 
                            \bitboxes{1}{{m} {+} {(} {a} {+} {(} {(} {b} {-} {c} {)} {*} {(} {d} {+} {k} {)} {)} {)} {\$} {(} {x} {+} {y} {)} {*} {p} } \\
                            \bitboxes[]{1}{{$\uparrow$} {} {} {} {} {} {} {} {} {} {} {} {} {} {} {} {} {} {} {} {} {} {} {} {} {} {} }
                        \end{bytefield}
                    \end{center}
                }
                \end{minipage}
            } \\
        
        %----------------------------------------------------------------------------------------
        \hline
        \hline
            {
                \begin{minted}[autogobble]{c}
                    infix_string[i=0] = 'm';
                \end{minted}
            }
            \begin{itemize}
                \item \mintinline{c}{'m'} is an operand, thus we just  append it to the post-fix string. 
            \end{itemize}
            &  
                \begin{itemize}
                    \item The stack looks like this: 
                        {
                        \begin{center}
                            \begin{tabular}{ c }
                                \mintinline{c}{OperandStack =} \\ \\
                            \end{tabular}
                            \begin{bytefield}{10}
                                \bitheader{0-9} \\
                                \bitboxes{1}{ {} {} {} {} {} {} {} {} {} {}}
                            \end{bytefield}
                        \end{center}
                        }
                    
                    \item The post-fix string looks like this: 
                        {
                            \begin{minted}[autogobble]{c}
                                postfix = "m";
                            \end{minted}
                        }
                \end{itemize}
            \\
            
        %----------------------------------------------------------------------------------------
        \hline
        \hline
            \multicolumn{2}{|l|}{
                \begin{minipage}{\linewidth}
                {
                    \begin{center}
                        \begin{bytefield}{27}
                            \\
                            \large 
                            \bitheader{0-26} \\ 
                            \bitboxes{1}{{m} {+} {(} {a} {+} {(} {(} {b} {-} {c} {)} {*} {(} {d} {+} {k} {)} {)} {)} {\$} {(} {x} {+} {y} {)} {*} {p} } \\
                            \bitboxes[]{1}{{} {$\uparrow$} {} {} {} {} {} {} {} {} {} {} {} {} {} {} {} {} {} {} {} {} {} {} {} {} {} }
                        \end{bytefield}
                    \end{center}
                }
                \end{minipage}
            } \\
        
        %----------------------------------------------------------------------------------------
        
        \hline
        \hline
            {
                \begin{minted}[autogobble]{c}
                    infix_string[i=1] = '+'; 
                \end{minted}
            }
            \begin{itemize}
                \item \mintinline{c}{'+'} is an operator, since the stack is empty then we just append it to the stack without checking precedence.  
            \end{itemize}
            &  
                \begin{itemize}
                    \item The stack looks like this: 
                        {
                        \begin{center}
                            \begin{tabular}{ c }
                                \mintinline{c}{OperandStack =} \\ \\
                            \end{tabular}
                            \begin{bytefield}{10}
                                \bitheader{0-9} \\
                                \bitboxes{1}{ {+} {} {} {} {} {} {} {} {} {}}
                            \end{bytefield}
                        \end{center}
                        }
                    
                    \item The post-fix string looks like this: 
                        {
                            \begin{minted}[autogobble]{c}
                                postfix = "m";
                            \end{minted}
                        }
                \end{itemize}
            \\
            
        %----------------------------------------------------------------------------------------
        \hline
        \hline
            \multicolumn{2}{|l|}{
                \begin{minipage}{\linewidth}
                {
                    \begin{center}
                        \begin{bytefield}{27}
                            \\
                            \large 
                            \bitheader{0-26} \\ 
                            \bitboxes{1}{{m} {+} {(} {a} {+} {(} {(} {b} {-} {c} {)} {*} {(} {d} {+} {k} {)} {)} {)} {\$} {(} {x} {+} {y} {)} {*} {p} } \\
                            \bitboxes[]{1}{{} {} {$\uparrow$} {} {} {} {} {} {} {} {} {} {} {} {} {} {} {} {} {} {} {} {} {} {} {} {} }
                        \end{bytefield}
                    \end{center}
                }
                \end{minipage}
            } \\
        
        %----------------------------------------------------------------------------------------
        \hline
        \hline 
            {
                \begin{minted}[autogobble]{c}
                    infix_string[i=2] = '(';
                \end{minted}
            }
            \begin{itemize}
                \item \mintinline{c}{'('} is an operator; the stack is not empty so we need to check precedence, taking in to account that the stack top is \verb|'+'| and the token is \verb|'('| if we check the precedence of this using the table above we can observe that the \verb|(| character .
                \item We push it to the stack if the precedence function evaluates to false.
            \end{itemize}
            {
                \begin{minted}[autogobble]{c}
                    precedence('+','('); // -> false
                \end{minted}
            }
            &  
                \begin{itemize}
                    \item The stack looks like this: 
                        {
                        \begin{center}
                            \begin{tabular}{ c }
                                \mintinline{c}{OperandStack =} \\ \\
                            \end{tabular}
                            \begin{bytefield}{10}
                                \bitheader{0-9} \\
                                \bitboxes{1}{ {+} {(} {} {} {} {} {} {} {} {}}
                            \end{bytefield}
                        \end{center}
                        }
                    
                    \item The post-fix string looks like this: 
                        {
                            \begin{minted}[autogobble]{c}
                                postfix = "m";
                            \end{minted}
                        }
                \end{itemize}
            \\
        
        %----------------------------------------------------------------------------------------
        \hline
        \hline
            \multicolumn{2}{|l|}{
                \begin{minipage}{\linewidth}
                {
                    \begin{center}
                        \begin{bytefield}{27}
                            \\
                            \large 
                            \bitheader{0-26} \\ 
                            \bitboxes{1}{{m} {+} {(} {a} {+} {(} {(} {b} {-} {c} {)} {*} {(} {d} {+} {k} {)} {)} {)} {\$} {(} {x} {+} {y} {)} {*} {p} } \\
                            \bitboxes[]{1}{{} {} {} {$\uparrow$} {} {} {} {} {} {} {} {} {} {} {} {} {} {} {} {} {} {} {} {} {} {} {} }
                        \end{bytefield}    \\ 
                    \end{center}
                }
                \end{minipage}
            } \\
        
        %----------------------------------------------------------------------------------------
        \hline
        {
                \begin{minted}[autogobble]{c}
                    infix_string[i=3] = 'a'
                \end{minted}
            }
            \begin{itemize}
                \item \mintinline{c}{'a'} is an operand; this we just append to the post-fix string.
            \end{itemize}
            &  
                \begin{itemize}
                    \item The stack looks like this: 
                        {
                        \begin{center}
                            \begin{tabular}{ c }
                                \mintinline{c}{OperandStack =} \\ \\
                            \end{tabular}
                            \begin{bytefield}{10}
                                \bitheader{0-9} \\
                                \bitboxes{1}{ {+} {(} {+} {} {} {} {} {} {} {}}
                            \end{bytefield}
                        \end{center}
                        }
                    
                    \item The post-fix string looks like this: 
                        {
                            \begin{minted}[autogobble]{c}
                                postfix = "ma";
                            \end{minted}
                        }
                \end{itemize}
            \\
        
        %----------------------------------------------------------------------------------------
        \hline
        \hline
            \multicolumn{2}{|l|}{
                \begin{minipage}{\linewidth}
                {
                    \begin{center}
                        \begin{bytefield}{27}
                            \\
                            \large 
                            \bitheader{0-26} \\ 
                            \bitboxes{1}{{m} {+} {(} {a} {+} {(} {(} {b} {-} {c} {)} {*} {(} {d} {+} {k} {)} {)} {)} {\$} {(} {x} {+} {y} {)} {*} {p} } \\
                            \bitboxes[]{1}{{} {} {} {} {$\uparrow$} {} {} {} {} {} {} {} {} {} {} {} {} {} {} {} {} {} {} {} {} {} {} }
                        \end{bytefield}
                    \end{center}
                }
                \end{minipage}
            } \\
        %----------------------------------------------------------------------------------------
        \hline
        \hline        
        {
                \begin{minted}[autogobble]{c}
                    infix_string[i=4] = '+'
                \end{minted}
            }
            \begin{itemize}
                \item \mintinline{c}{'+'} is an operator, the stack is not empty, since this is the case the precedence needs to be evaluated.
            \end{itemize}
            {
                \begin{minted}[autogobble]{c}
                    precedence('(','+'); // -> false
                \end{minted}
            }
            &  
                \begin{itemize}
                    \item The stack looks like this: 
                        {
                        \begin{center}
                            \begin{tabular}{ c }
                                \mintinline{c}{OperandStack =} \\ \\
                            \end{tabular}
                            \begin{bytefield}{10}
                                \bitheader{0-9} \\
                                \bitboxes{1}{ {+} {(} {+} {} {} {} {} {} {} {}}
                            \end{bytefield}
                        \end{center}
                        }
                    
                    \item The post-fix string looks like this: 
                        {
                            \begin{minted}[autogobble]{c}
                                postfix = "ma";
                            \end{minted}
                        }
                \end{itemize}
            \\
        %----------------------------------------------------------------------------------------
        \hline
        \hline
            \multicolumn{2}{|l|}{
                \begin{minipage}{\linewidth}
                {
                    \begin{center}
                        \begin{bytefield}{27}
                            \\
                            \large 
                            \bitheader{0-26} \\ 
                            \bitboxes{1}{{m} {+} {(} {a} {+} {(} {(} {b} {-} {c} {)} {*} {(} {d} {+} {k} {)} {)} {)} {\$} {(} {x} {+} {y} {)} {*} {p} } \\
                            \bitboxes[]{1}{{} {} {} {} {} {$\uparrow$} {} {} {} {} {} {} {} {} {} {} {} {} {} {} {} {} {} {} {} {} {} }
                        \end{bytefield}
                    \end{center}
                }
                \end{minipage}
            } \\
        
        %----------------------------------------------------------------------------------------
        \hline
        \hline
        {
            \begin{minted}[autogobble]{c}
                infix_string[i=5] = '(';
            \end{minted}
        }
        \begin{itemize}
            \item The parenthesis is an operator, we compare it using the precedence function, and we note that the function returns false because \verb|+| has less precedence than \verb|(|.
        \end{itemize}
        {
            \begin{minted}[autogobble]{c}
                precedence('+','('); // -> false
            \end{minted}
        }
        & 
        \begin{itemize}
            \item The stack looks like this: 
                {
                \begin{center}
                    \begin{tabular}{ c }
                        \mintinline{c}{OperandStack =} \\ \\
                    \end{tabular}
                    \begin{bytefield}{10}
                        \bitheader{0-9} \\
                        \bitboxes{1}{ {+} {(} {+} {(} {} {} {} {} {} {}}
                    \end{bytefield}
                \end{center}
                }
            
            \item The post-fix string looks like this: 
                {
                    \begin{minted}[autogobble]{c}
                        postfix = "ma";
                    \end{minted}
                }
        \end{itemize}
        \\
        
        %----------------------------------------------------------------------------------------
        \hline
        \hline
            \multicolumn{2}{|l|}{
                \begin{minipage}{\linewidth}
                {
                    \begin{center}
                        \begin{bytefield}{27}
                            \\
                            \large 
                            \bitheader{0-26} \\ 
                            \bitboxes{1}{{m} {+} {(} {a} {+} {(} {(} {b} {-} {c} {)} {*} {(} {d} {+} {k} {)} {)} {)} {\$} {(} {x} {+} {y} {)} {*} {p} } \\
                            \bitboxes[]{1}{{} {} {} {} {} {} {$\uparrow$} {} {} {} {} {} {} {} {} {} {} {} {} {} {} {} {} {} {} {} {} }
                        \end{bytefield}
                    \end{center}
                }
                \end{minipage}
            } \\
        
        %----------------------------------------------------------------------------------------
        \hline
        \hline
            {
                \begin{minted}[autogobble]{c}
                    infix_string[i=6] = '(';
                \end{minted}
            }
            \begin{itemize}
                \item 
            \end{itemize}
            {
                \begin{minted}[autogobble]{c}
                    precedence('(','('); // -> false
                \end{minted}
            }
            &  
            \begin{itemize}
                \item The stack looks like this: 
                    {
                    \begin{center}
                        \begin{tabular}{ c }
                            \mintinline{c}{OperandStack =} \\ \\
                        \end{tabular}
                        \begin{bytefield}{10}
                            \bitheader{0-9} \\
                            \bitboxes{1}{ {+} {(} {+} {(} {(} {} {} {} {} {}}
                        \end{bytefield}
                    \end{center}
                    }
                
                \item The post-fix string looks like this: 
                    {
                        \begin{minted}[autogobble]{c}
                            postfix = "ma";
                        \end{minted}
                    }
            \end{itemize}
            \\
        
        %----------------------------------------------------------------------------------------
        \hline
        \hline
            \multicolumn{2}{|l|}{
                \begin{minipage}{\linewidth}
                {
                    \begin{center}
                        \begin{bytefield}{27}
                            \\
                            \large 
                            \bitheader{0-26} \\ 
                            \bitboxes{1}{{m} {+} {(} {a} {+} {(} {(} {b} {-} {c} {)} {*} {(} {d} {+} {k} {)} {)} {)} {\$} {(} {x} {+} {y} {)} {*} {p} } \\
                            \bitboxes[]{1}{{} {} {} {} {} {} {} {$\uparrow$} {} {} {} {} {} {} {} {} {} {} {} {} {} {} {} {} {} {} {} }
                        \end{bytefield}
                    \end{center}
                }
                \end{minipage}
            } \\
        
        %----------------------------------------------------------------------------------------
        \hline
        \hline
            {
                \begin{minted}[autogobble]{c}
                    infix_string[i=7] = 'b';
                \end{minted}
            }
            \begin{itemize}
                \item \mintinline{c}{'b'} is an operand, so we just append it to the post-fix string.
            \end{itemize}
            &  
            \begin{itemize}
                \item The stack looks like this: 
                    {
                    \begin{center}
                        \begin{tabular}{ c }
                            \mintinline{c}{OperandStack =} \\ \\
                        \end{tabular}
                        \begin{bytefield}{10}
                            \bitheader{0-9} \\
                            \bitboxes{1}{ {+} {(} {+} {(} {(} {} {} {} {} {}}
                        \end{bytefield}
                    \end{center}
                    }
                
                \item The post-fix string looks like this: 
                    {
                        \begin{minted}[autogobble]{c}
                            postfix = "mab";
                        \end{minted}
                    }
            \end{itemize}
            \\
        
        %----------------------------------------------------------------------------------------
        \hline
        \hline
            \multicolumn{2}{|l|}{
                \begin{minipage}{\linewidth}
                {
                    \begin{center}
                        \begin{bytefield}{27}
                            \\
                            \large 
                            \bitheader{0-26} \\ 
                            \bitboxes{1}{{m} {+} {(} {a} {+} {(} {(} {b} {-} {c} {)} {*} {(} {d} {+} {k} {)} {)} {)} {\$} {(} {x} {+} {y} {)} {*} {p} } \\
                            \bitboxes[]{1}{
                                {} {} {}
                                {} {} {} 
                                {} {} {$\uparrow$} 
                                {} {} {} 
                                {} {} {} 
                                {} {} {} 
                                {} {} {} 
                                {} {} {} 
                                {} {} {} }
                        \end{bytefield}
                    \end{center}
                }
                \end{minipage}
            } \\
        
        %----------------------------------------------------------------------------------------
        \hline
        \hline
            {
                \begin{minted}[autogobble]{c}
                    infix_string[i=8] = '-';
                \end{minted}
            }
            \begin{itemize}
                \item \mintinline{c}{'-'} is an operator, since the stack is not empty, we call the precedence function, the stack top is \mintinline{c}{'('} and the token is \mintinline{c}{'-'}, this evaluates to false.
            \end{itemize}
            {
                \begin{minted}[autogobble]{c}
                    precedence('(','-'); // -> false
                \end{minted}
            }
            &  
            \begin{itemize}
                \item The stack looks like this: 
                    {
                    \begin{center}
                        \begin{tabular}{ c }
                            \mintinline{c}{OperandStack =} \\ \\
                        \end{tabular}
                        \begin{bytefield}{10}
                            \bitheader{0-9} \\
                            \bitboxes{1}{ {+} {(} {+} {(} {(} {-} {} {} {} {}}
                        \end{bytefield}
                    \end{center}
                    }
                
                \item The post-fix string looks like this: 
                    {
                        \begin{minted}[autogobble]{c}
                            postfix = "mab";
                        \end{minted}
                    }
            \end{itemize}
            \\
        %----------------------------------------------------------------------------------------
        \hline
        \hline
            \multicolumn{2}{|l|}{
                \begin{minipage}{\linewidth}
                {
                    \begin{center}
                        \begin{bytefield}{27}
                            \\
                            \large 
                            \bitheader{0-26} \\ 
                            \bitboxes{1}{{m} {+} {(} {a} {+} {(} {(} {b} {-} {c} {)} {*} {(} {d} {+} {k} {)} {)} {)} {\$} {(} {x} {+} {y} {)} {*} {p} } \\
                            \bitboxes[]{1}{
                                {} {} {}
                                {} {} {} 
                                {} {} {} 
                                {$\uparrow$} {} {} 
                                {} {} {} 
                                {} {} {} 
                                {} {} {} 
                                {} {} {} 
                                {} {} {} }
                        \end{bytefield}
                    \end{center}
                }
                \end{minipage}
            } \\
        \hline
        \hline
        %----------------------------------------------------------------------------------------
            {
                \begin{minted}[autogobble]{c}
                    infix_string[i=9] = 'c';
                \end{minted}
            }
            \begin{itemize}
                \item \mintinline{c}{'c'} is an operand, thus we just append it to the post-fix string.
            \end{itemize}
            &  
            \begin{itemize}
                \item The stack looks like this: 
                    {
                    \begin{center}
                        \begin{tabular}{ c }
                            \mintinline{c}{OperandStack =} \\ \\
                        \end{tabular}
                        \begin{bytefield}{10}
                            \bitheader{0-9} \\
                            \bitboxes{1}{ {+} {(} {+} {(} {(} {-} {} {} {} {}}
                        \end{bytefield}
                    \end{center}
                    }
                
                \item The post-fix string looks like this: 
                    {
                        \begin{minted}[autogobble]{c}
                            postfix = "mabc";
                        \end{minted}
                    }
            \end{itemize}
            \\
        %----------------------------------------------------------------------------------------
        \hline
        \hline
            \multicolumn{2}{|l|}{
                \begin{minipage}{\linewidth}
                {
                    \begin{center}
                        \begin{bytefield}{27}
                            \\
                            \large 
                            \bitheader{0-26} \\ 
                            \bitboxes{1}{{m} {+} {(} {a} {+} {(} {(} {b} {-} {c} {)} {*} {(} {d} {+} {k} {)} {)} {)} {\$} {(} {x} {+} {y} {)} {*} {p} } \\
                            \bitboxes[]{1}{
                                {} {} {}
                                {} {} {} 
                                {} {} {} 
                                {} {$\uparrow$} {} 
                                {} {} {} 
                                {} {} {} 
                                {} {} {} 
                                {} {} {} 
                                {} {} {} }
                        \end{bytefield}
                    \end{center}
                }
                \end{minipage}
            } \\
        
        %----------------------------------------------------------------------------------------
        \hline
        \hline
            {
                \begin{minted}[autogobble]{c}
                    infix_string[i=10] = ')';
                \end{minted}
            }
            \begin{itemize}
                \item \mintinline{c}{')'} is an operator, the stack is not empty, but, upon checking precedence this evaluates to true.  
                \item The rule when the precedence function returns true is to pop the stack top until the precedence is false again.
            \end{itemize}
            {
                \begin{minted}[autogobble]{c}
                    precedence('-',')'); // -> true
                \end{minted}
            }
            &  
            \begin{itemize}
                \item The stack looks like this: 
                    {
                    \begin{center}
                        \begin{tabular}{ c }
                            \mintinline{c}{OperandStack =} \\ \\
                        \end{tabular}
                        \begin{bytefield}{10}
                            \bitheader{0-9} \\
                            \bitboxes{1}{ {+} {(} {+} {(} {(} {-} {} {} {} {}}
                        \end{bytefield}
                    \end{center}
                    \begin{minted}[autogobble]{c}
                        postfix.append(pop(&OperandStack));
                    \end{minted}
                    \begin{center}
                        \begin{tabular}{ c }
                            \mintinline{c}{OperandStack =} \\ \\
                        \end{tabular}
                        \begin{bytefield}{10}
                            \bitheader{0-9} \\
                            \bitboxes{1}{ {+} {(} {+} {(} {(} {} {} {} {} {}}
                        \end{bytefield}
                    \end{center}
                    \begin{itemize}
                        \item Now we must compare the stack top that is now \mintinline{c}{'('} with the token which is \mintinline{c}{')'}, in that case what the precedence function returns is false, but the case will be special.
                    \end{itemize}
                    \begin{minted}[autogobble]{c}
                        // check precedence again
                        precedence('(',')'); // -> false
                    \end{minted}
                    \begin{itemize}
                        \item In the case that stack top is \verb|'('| and the token is \verb|')'|, this case evaluates to false but it will be treated differently. 
                        \item What must be done is pop the stack once and discard the token.
                    \end{itemize}
                    \begin{minted}[autogobble]{c}
                        pop(&OperandStack);
                    \end{minted}
                    \begin{tabular}{ c }
                        \mintinline{c}{OperandStack =} \\ \\
                    \end{tabular}
                    \begin{bytefield}{10}
                        \bitheader{0-9} \\
                        \bitboxes{1}{ {+} {(} {+} {(} {} {} {} {} {} {}}
                    \end{bytefield}
                    }
                
                \item The post-fix string looks like this: 
                    {
                        \begin{minted}[autogobble]{c}
                            postfix = "mabc-";
                        \end{minted}
                    }
            \end{itemize}
            \\
        %----------------------------------------------------------------------------------------
        \hline
        \hline
            \multicolumn{2}{|l|}{
                \begin{minipage}{\linewidth}
                {
                    \begin{center}
                        \begin{bytefield}{27}
                            \\
                            \large 
                            \bitheader{0-26} \\ 
                            \bitboxes{1}{{m} {+} {(} {a} {+} {(} {(} {b} {-} {c} {)} {*} {(} {d} {+} {k} {)} {)} {)} {\$} {(} {x} {+} {y} {)} {*} {p} } \\
                            \bitboxes[]{1}{
                                {} {} {}
                                {} {} {} 
                                {} {} {} 
                                {} {} {$\uparrow$} 
                                {} {} {} 
                                {} {} {} 
                                {} {} {} 
                                {} {} {} 
                                {} {} {} }
                        \end{bytefield}
                    \end{center}
                }
                \end{minipage}
            } \\
        
        %----------------------------------------------------------------------------------------
        \hline
        \hline
            {
                \begin{minted}[autogobble]{c}
                    infix_string[i=11] = '*';
                \end{minted}
            }
            \begin{itemize}
                \item The \mintinline{c}{'*'} character has higher precedence, since the precedence function returns false, we push the character to the stack.
            \end{itemize}
            {
                \begin{minted}[autogobble]{c}
                    precedence('(','*'); // -> false
                \end{minted}
            }
            &
            \begin{itemize}
                \item The stack looks like this: 
                    {
                        \begin{center}
                           \begin{tabular}{ c }
                        \mintinline{c}{OperandStack =} \\ \\
                        \end{tabular}
                        \begin{bytefield}{10}
                            \bitheader{0-9} \\
                            \bitboxes{1}{ {+} {(} {+} {(} {*} {} {} {} {} {}}
                        \end{bytefield}
                        \end{center}
                    }
            
            \item The post-fix string looks like this: 
                {
                    \begin{minted}[autogobble]{c}
                        postfix = "mabc-";
                    \end{minted}
                }
        \end{itemize}
            \\
        %----------------------------------------------------------------------------------------
        \hline
        \hline
            \multicolumn{2}{|l|}{
                \begin{minipage}{\linewidth}
                {
                    \begin{center}
                        \begin{bytefield}{27}
                            \\
                            \Huge 
                            \bitheader{0-26} \\ 
                            \bitboxes{1}{{m} {+} {(} {a} {+} {(} {(} {b} {-} {c} {)} {*} {(} {d} {+} {k} {)} {)} {)} {\$} {(} {x} {+} {y} {)} {*} {p} } \\
                            \bitboxes[]{1}{{} {} {} {} {} {} {} {} {} {} {} {} {$\uparrow$} {} {} {} {} {} {} {} {} {} {} {} {} {} {} }
                        \end{bytefield}
                    \end{center}
                }
                \end{minipage}
            } \\
        
        %----------------------------------------------------------------------------------------
        \hline
        \hline
            {
                \begin{minted}[autogobble]{c}
                    infix_string[i=12] = '(';
                \end{minted}
            }
            \begin{itemize}
                \item The character \mintinline{c}{'('} has higher precedence than \mintinline{c}{*} thus the precedence function returns false. 
                \item We must push in to the stack. 
            \end{itemize}
            {
                \begin{minted}[autogobble]{c}
                    precedence('*','('); // -> false 
                \end{minted}
            }
            &  
                \begin{itemize}
                    \item The stack looks like this: 
                        {
                            \begin{center}
                               \begin{tabular}{ c }
                            \mintinline{c}{OperandStack =} \\ \\
                            \end{tabular}
                            \begin{bytefield}{10}
                                \bitheader{0-9} \\
                                \bitboxes{1}{ {+} {(} {+} {(} {*} {(} {} {} {} {}}
                            \end{bytefield}
                            \end{center}
                        }
                
                \item The post-fix string looks like this: 
                    {
                        \begin{minted}[autogobble]{c}
                            postfix = "mabc-";
                        \end{minted}
                    }
            \end{itemize}
            \\
        %----------------------------------------------------------------------------------------
        \hline
        \hline
            \multicolumn{2}{|l|}{
                \begin{minipage}{\linewidth}
                {
                    \begin{center}
                        \begin{bytefield}{27}
                            \\
                            \Huge 
                            \bitheader{0-26} \\ 
                            \bitboxes{1}{{m} {+} {(} {a} {+} {(} {(} {b} {-} {c} {)} {*} {(} {d} {+} {k} {)} {)} {)} {\$} {(} {x} {+} {y} {)} {*} {p} } \\
                            \bitboxes[]{1}{{} {} {} {} {} {} {} {} {} {} {} {} {} {$\uparrow$} {} {} {} {} {} {} {} {} {} {} {} {} {} }
                        \end{bytefield}
                    \end{center}
                }
                \end{minipage}
            } \\
        %----------------------------------------------------------------------------------------
        \hline
        \hline
            {
                \begin{minted}[autogobble]{c}
                    infix_string[i=13] = 'd';
                \end{minted}
            }
            \begin{itemize}
                \item This is an operand, we just append to the post-fix string.
            \end{itemize}
            &
            \begin{itemize}
                \item The stack looks like this: 
                    {
                        \begin{center}
                            \begin{tabular}{ c }
                                \mintinline{c}{OperandStack =} \\ \\
                            \end{tabular}
                            \begin{bytefield}{10}
                                    \bitheader{0-9} \\
                                    \bitboxes{1}{ {+} {(} {+} {(} {*} {(} {} {} {} {}}
                            \end{bytefield}
                        \end{center}
                    }
                
                \item The post-fix string looks like this:
                    {
                        \begin{minted}[autogobble]{c}
                            postfix = "mabc-d";
                        \end{minted}
                    }
            \end{itemize}
            \\ 
        \hline 
        \hline
        
        %----------------------------------------------------------------------------------------
        \multicolumn{2}{|l|}{
            \begin{minipage}{\linewidth}
            {
                \begin{center}
                    \begin{bytefield}{27}
                        \\
                        \Huge 
                        \bitheader{0-26} \\ 
                        \bitboxes{1}{{m} {+} {(} {a} {+} {(} {(} {b} {-} {c} {)} {*} {(} {d} {+} {k} {)} {)} {)} {\$} {(} {x} {+} {y} {)} {*} {p} } \\
                        \bitboxes[]{1}{{} {} {} {} {} {} {} {} {} {} {} {} {} {} {$\uparrow$} {} {} {} {} {} {} {} {} {} {} {} {} }
                    \end{bytefield}
                \end{center}
            }
            \end{minipage}
            } \\
            
            %----------------------------------------------------------------------------------------
            \hline
            \hline
            {
                \begin{minted}[autogobble]{c}
                    infix_string[i=14] = '+';
                \end{minted}
            }
            \begin{itemize}
                \item The \mintinline{c}{'+'} has higher precedence, this means the precedence function returns false. 
            \end{itemize}
            {
                \begin{minted}[autogobble]{c}
                    precedence('(','+'); // -> false
                \end{minted}
            }
            &
            \begin{itemize}
                \item The stack looks like this: 
                    {
                        \begin{center}
                            \begin{tabular}{ c }
                                \mintinline{c}{OperandStack =} \\ \\
                            \end{tabular}
                            \begin{bytefield}{10}
                                    \bitheader{0-9} \\
                                    \bitboxes{1}{ {+} {(} {+} {(} {*} {(} {+} {} {} {}}
                            \end{bytefield}
                        \end{center}
                    }
                
                \item The post-fix string looks like this:
                    {
                        \begin{minted}[autogobble]{c}
                            postfix = "mabc-d";
                        \end{minted}
                    }
            \end{itemize}
        \\ 
        \hline
        \hline
        %----------------------------------------------------------------------------------------
        \multicolumn{2}{|l|}{
            \begin{minipage}{\linewidth}
            {
                \begin{center}
                    \begin{bytefield}{27}
                        \\
                        \Huge 
                        \bitheader{0-26} \\ 
                        \bitboxes{1}{{m} {+} {(} {a} {+} {(} {(} {b} {-} {c} {)} {*} {(} {d} {+} {k} {)} {)} {)} {\$} {(} {x} {+} {y} {)} {*} {p} } \\
                        \bitboxes[]{1}{{} {} {} {} {} {} {} {} {} {} {} {} {} {} {} {$\uparrow$} {} {} {} {} {} {} {} {} {} {} {} }
                    \end{bytefield}
                \end{center}
            }
            \end{minipage}
            } \\
            \hline
            \hline
        %----------------------------------------------------------------------------------------
            {
                \begin{minted}[autogobble]{c}
                    infix_string[i=15] = 'k';
                \end{minted}
            }
            \begin{itemize}
                \item \mintinline{c}{'k'} is an operator, thus we just append it to the post-fix string.
            \end{itemize}
        & 
        \begin{itemize}
            \item The stack looks like this: 
                {
                    \begin{center}
                        \begin{tabular}{ c }
                            \mintinline{c}{OperandStack =} \\ \\
                        \end{tabular}
                        \begin{bytefield}{10}
                                \bitheader{0-9} \\
                                \bitboxes{1}{ {+} {(} {+} {(} {*} {(} {+} {} {} {}}
                        \end{bytefield}
                    \end{center}
                }
            
            \item The post-fix string looks like this:
                {
                    \begin{minted}[autogobble]{c}
                        postfix = "mabc-dk";
                    \end{minted}
                }
        \end{itemize}
    \\ 
    \hline
    \hline
    %----------------------------------------------------------------------------------------
    \multicolumn{2}{|l|}{
        \begin{minipage}{\linewidth}
        {
            \begin{center}
                \begin{bytefield}{27}
                    \\
                    \Huge 
                    \bitheader{0-26} \\ 
                    \bitboxes{1}{{m} {+} {(} {a} {+} {(} {(} {b} {-} {c} {)} {*} {(} {d} {+} {k} {)} {)} {)} {\$} {(} {x} {+} {y} {)} {*} {p} } \\
                    \bitboxes[]{1}{{} {} {} {} {} {} {} {} {} {} {} {} {} {} {} {} {$\uparrow$} {} {} {} {} {} {} {} {} {} {} }
                \end{bytefield}
            \end{center}
        }
        \end{minipage}
        } \\
    \hline
    \hline
    %----------------------------------------------------------------------------------------
            {
            \begin{minted}[autogobble]{c}
                infix_string[i=16] = ')';
            \end{minted}
            }
            \begin{itemize}
                \item The stack top is \mintinline{c}{'+'} and the token is \mintinline{c}{')'}, meaning that the precedence function will return true. 
                \item This means that we must pop the stack, append the operators poped to the post-fix string until the precedence function returns 
            \end{itemize}
            {
                \begin{minted}[autogobble]{c}
                    precedence('+',')'); // -> true
                \end{minted}
            }
            &
            \begin{itemize}
                \item The stack looks like this: 
                    {
                        \begin{center}
                            \begin{tabular}{ c }
                                \mintinline{c}{OperandStack =} \\ \\
                            \end{tabular}
                            \begin{bytefield}{10}
                                    \bitheader{0-9} \\
                                    \bitboxes{1}{ {+} {(} {+} {(} {*} {(} {+} {} {} {}}
                            \end{bytefield}
                            \begin{minted}[autogobble]{c}
                                postfix.append(pop(&OperandStack));
                            \end{minted}
                            \begin{tabular}{ c }
                                \mintinline{c}{OperandStack =} \\ \\
                            \end{tabular}
                            \begin{bytefield}{10}
                                    \bitheader{0-9} \\
                                    \bitboxes{1}{ {+} {(} {+} {(} {*} {(} {} {} {} {}}
                            \end{bytefield}
                            \begin{itemize}
                                \item Continue with the precedence checking until true. 
                            \end{itemize}
                            \begin{minted}[autogobble]{c}
                                precedence('(',')');
                            \end{minted}
                            \begin{itemize}
                                \item This is false, but this is the exceptional case, for the case \mintinline{c}{'('} as the stack top and token \mintinline{c}{')'} we pop the stack and discard the token as well. 
                            \end{itemize}
                            \begin{minted}[autogobble]{c}
                                pop(&OperandStack);
                            \end{minted}
                            \begin{tabular}{ c }
                                \mintinline{c}{OperandStack =} \\ \\
                            \end{tabular}
                            \begin{bytefield}{10}
                                    \bitheader{0-9} \\
                                    \bitboxes{1}{ {+} {(} {+} {(} {*} {} {} {} {} {}}
                            \end{bytefield}
                        \end{center}
                    }
                
                \item The post-fix string looks like this:
                    {
                        \begin{minted}[autogobble]{c}
                            postfix = "mabc-dk+";
                        \end{minted}
                    }
            \end{itemize}
        \\ 
        \hline
        \hline
        %----------------------------------------------------------------------------------------
        %----------------------------------------------------------------------------------------
    \multicolumn{2}{|l|}{
        \begin{minipage}{\linewidth}
        {
            \begin{center}
                \begin{bytefield}{27}
                    \\
                    \Huge 
                    \bitheader{0-26} \\ 
                    \bitboxes{1}{{m} {+} {(} {a} {+} {(} {(} {b} {-} {c} {)} {*} {(} {d} {+} {k} {)} {)} {)} {\$} {(} {x} {+} {y} {)} {*} {p} } \\
                    \bitboxes[]{1}{{} {} {} {} {} {} {} {} {} {} {} {} {} {} {} {} {} {$\uparrow$} {} {} {} {} {} {} {} {} {} }
                \end{bytefield}
            \end{center}
        }
        \end{minipage}
        } \\
    \hline
    \hline
    
    %----------------------------------------------------------------------------------------
        
            {
                \begin{minted}[autogobble]{c}
                    infix_string[i=17] = ')';
                \end{minted}
            }
            \begin{itemize}
                \item This is an operator, and needs to evaluated for precedence, in this case it is true, thus we need to pop the stack and append the poped element to the post-fix string until the precedence function returns false. 
            \end{itemize}
            {
                \begin{minted}[autogobble]{c}
                    precedence('*',')'); // -> true
                \end{minted}
            }
            &
            \begin{itemize}
                \item The stack looks like this: 
                    {
                        \begin{center}
                            \begin{tabular}{ c }
                                \mintinline{c}{OperandStack =} \\ \\
                            \end{tabular}
                            \begin{bytefield}{10}
                                    \bitheader{0-9} \\
                                    \bitboxes{1}{ {+} {(} {+} {(} {*} {} {} {} {} {}}
                            \end{bytefield}
                        \end{center}
                        \begin{itemize}
                            \item The \mintinline{c}{'*'} character constituting the stack top and the token \mintinline{c}{')'} will cause a true to be returned from the precedence function. Thus pop it and append it to the post-fix string.
                        \end{itemize}
                        \begin{minted}[autogobble]{c}
                            postfix.append(pop(&OperandStack));
                        \end{minted}
                        \begin{center}
                            \begin{tabular}{ c }
                                \mintinline{c}{OperandStack =} \\ \\
                            \end{tabular}
                            \begin{bytefield}{10}
                                    \bitheader{0-9} \\
                                    \bitboxes{1}{ {+} {(} {+} {(} {} {} {} {} {} {}}
                            \end{bytefield}
                        \end{center}
                        \begin{itemize}
                            \item The stack top is \mintinline{c}{'('} and the token is \mintinline{c}{')'}, this is the exception case, we pop and discard.
                        \end{itemize}
                        \begin{minted}[autogobble]{c}
                            pop(&OperandStack);
                        \end{minted}
                        \begin{center}
                            \begin{tabular}{ c }
                                \mintinline{c}{OperandStack =} \\ \\
                            \end{tabular}
                            \begin{bytefield}{10}
                                    \bitheader{0-9} \\
                                    \bitboxes{1}{ {+} {(} {+} {} {} {} {} {} {} {}}
                            \end{bytefield}
                        \end{center}
                    }
                
                \item The post-fix string looks like this:
                    {
                        \begin{minted}[autogobble]{c}
                            postfix = "mabc-dk+*";
                        \end{minted}
                    }
            \end{itemize}
            \\
            \hline
            \hline
            
            %----------------------------------------------------------------------------------------
            \multicolumn{2}{|l|}{
                \begin{minipage}{\linewidth}
                {
                    \begin{center}
                        \begin{bytefield}{27}
                            \\
                            \Huge 
                            \bitheader{0-26} \\ 
                            \bitboxes{1}{{m} {+} {(} {a} {+} {(} {(} {b} {-} {c} {)} {*} {(} {d} {+} {k} {)} {)} {)} {\$} {(} {x} {+} {y} {)} {*} {p} } \\
                            \bitboxes[]{1}{{} {} {} {} {} {} {} {} {} {} {} {} {} {} {} {} {} {} {$\uparrow$} {} {} {} {} {} {} {} {} }
                        \end{bytefield}
                    \end{center}
                }
                \end{minipage}
                } \\
            \hline
            \hline
            %----------------------------------------------------------------------------------------
            
            {
                \begin{minted}[autogobble]{c}
                    infix_string[i=18] = ')';
                \end{minted}
            }
            \begin{itemize}
                \item The stack top is \mintinline{c}{'+'} and the token is \mintinline{c}{')'}, the precedence function returns true, thus we pop the stack and append to the post-fix string.
            \end{itemize}
            {
                \begin{minted}[autogobble]{c}
                    precedence('+',')'); // -> true
                \end{minted}
            }
            &
            \begin{itemize}
                \item The stack looks like this: 
                    {
                        \begin{center}
                            \begin{tabular}{ c }
                                \mintinline{c}{OperandStack =} \\ \\
                            \end{tabular}
                            \begin{bytefield}{10}
                                \bitheader{0-9} \\
                                \bitboxes{1}{ {+} {(} {+} {} {} {} {} {} {} {}}
                            \end{bytefield}
                            \begin{minted}[autogobble]{c}
                                postfix.append(pop(&OperandStack));
                            \end{minted}
                            \begin{tabular}{ c }
                                \mintinline{c}{OperandStack =} \\ \\
                            \end{tabular}
                            \begin{bytefield}{10}
                                \bitheader{0-9} \\
                                \bitboxes{1}{ {+} {(} {} {} {} {} {} {} {} {}}
                            \end{bytefield}
                            \begin{itemize}
                                \item Again we check for precedence.
                            \end{itemize}
                            \begin{minted}[autogobble]{c}
                                precedence('(',')');
                            \end{minted}
                            \begin{itemize}
                                \item This case we discard.
                            \end{itemize}
                            \begin{minted}[autogobble]{c}
                                pop(&OperandStack);
                            \end{minted}
                            \begin{tabular}{ c }
                                \mintinline{c}{OperandStack =} \\ \\
                            \end{tabular}
                            \begin{bytefield}{10}
                                \bitheader{0-9} \\
                                \bitboxes{1}{ {+} {} {} {} {} {} {} {} {} {}}
                            \end{bytefield}
                        \end{center}
                    }
                
                \item The post-fix string looks like this:
                    {
                        \begin{minted}[autogobble]{c}
                            postfix = "mabc-dk+*+";
                        \end{minted}
                    }
            \end{itemize}
        \\ 
        \hline 
        \hline
        
        %----------------------------------------------------------------------------------------
        \multicolumn{2}{|l|}{
                \begin{minipage}{\linewidth}
                {
                    \begin{center}
                        \begin{bytefield}{27}
                            \\
                            \Huge 
                            \bitheader{0-26} \\ 
                            \bitboxes{1}{{m} {+} {(} {a} {+} {(} {(} {b} {-} {c} {)} {*} {(} {d} {+} {k} {)} {)} {)} {\$} {(} {x} {+} {y} {)} {*} {p} } \\
                            \bitboxes[]{1}{{} {} {} {} {} {} {} {} {} {} {} {} {} {} {} {} {} {} {} {$\uparrow$} {} {} {} {} {} {} {} }
                        \end{bytefield}
                    \end{center}
                }
                \end{minipage}
                } \\
            \hline
            \hline
            
        %----------------------------------------------------------------------------------------
            {
                \begin{minted}[autogobble]{c}
                    infix_string[i=19] = '$';
                \end{minted}
            }
            \begin{itemize}
                \item The token \mintinline{c}{'$'} and the stack top \mintinline{c}{+} will cause the precedence function to return false.
                \item We push the \$ to the stack. 
            \end{itemize}
            {
                \begin{minted}[autogobble]{c}
                    precedence('+','$'); // -> false
                \end{minted}
            }
            &
            \begin{itemize}
                \item The stack looks like this: 
                    {
                        \begin{center}
                            \begin{tabular}{ c }
                                \mintinline{c}{OperandStack =} \\ \\
                            \end{tabular}
                            \begin{bytefield}{10}
                                \bitheader{0-9} \\
                                \bitboxes{1}{ {+} {\$} {} {} {} {} {} {} {} {}}
                            \end{bytefield}
                        \end{center}
                    }
                
                \item The post-fix string looks like this:
                    {
                        \begin{minted}[autogobble]{c}
                            postfix = "mabc-dk+*+";
                        \end{minted}
                    }
            \end{itemize}
        \\ 
        \hline
        \hline
        
        %----------------------------------------------------------------------------------------
        \multicolumn{2}{|l|}{
                \begin{minipage}{\linewidth}
                {
                    \begin{center}
                        \begin{bytefield}{27}
                            \\
                            \Huge 
                            \bitheader{0-26} \\ 
                            \bitboxes{1}{{m} {+} {(} {a} {+} {(} {(} {b} {-} {c} {)} {*} {(} {d} {+} {k} {)} {)} {)} {\$} {(} {x} {+} {y} {)} {*} {p} } \\
                            \bitboxes[]{1}{{} {} {} {} {} {} {} {} {} {} {} {} {} {} {} {} {} {} {} {} {$\uparrow$} {} {} {} {} {} {} }
                        \end{bytefield}
                    \end{center}
                }
                \end{minipage}
                } \\
        \hline
        \hline
        %----------------------------------------------------------------------------------------
            {
                \begin{minted}[autogobble]{c}
                    infix_string[i=20] = '(';
                \end{minted}
            }
            \begin{itemize}
                \item The \mintinline{c}{'('} has higher precedence, so the function will return false, if it is false we push it to the stack as always.
            \end{itemize}
            {
                \begin{minted}[autogobble]{c}
                    precedence('$','('); // -> false
                \end{minted}
            }
            &
            \begin{itemize}
                \item The stack looks like this: 
                    {
                        \begin{tabular}{ c }
                            \mintinline{c}{OperandStack =} \\ \\
                        \end{tabular}
                        \begin{bytefield}{10}
                            \bitheader{0-9} \\
                            \bitboxes{1}{ {+} {\$} {(} {} {} {} {} {} {} {}}
                        \end{bytefield}
                    }
                
                \item The post-fix string looks like this:
                    {
                        \begin{minted}[autogobble]{c}
                            postfix = "mabc-dk+*+";
                        \end{minted}
                    }
            \end{itemize}
        \\
        \hline 
        \hline
        %----------------------------------------------------------------------------------------
        \multicolumn{2}{|l|}{
                \begin{minipage}{\linewidth}
                {
                    \begin{center}
                        \begin{bytefield}{27}
                            \\
                            \Huge 
                            \bitheader{0-26} \\ 
                            \bitboxes{1}{{m} {+} {(} {a} {+} {(} {(} {b} {-} {c} {)} {*} {(} {d} {+} {k} {)} {)} {)} {\$} {(} {x} {+} {y} {)} {*} {p} } \\
                            \bitboxes[]{1}{{} {} {} {} {} {} {} {} {} {} {} {} {} {} {} {} {} {} {} {} {} {$\uparrow$} {} {} {} {} {} }
                        \end{bytefield}
                    \end{center}
                }
                \end{minipage}
                } \\
        \hline
        \hline        
        %----------------------------------------------------------------------------------------
            {
                \begin{minted}[autogobble]{c}
                    infix_string[i=21] = 'x';
                \end{minted}
            }
            \begin{itemize}
                \item \mintinline{c}{'x'} is an operand. Just append it to the post-fix string.
            \end{itemize}
            &
            \begin{itemize}
                \item The stack looks like this: 
                    {
                        \begin{tabular}{ c }
                            \mintinline{c}{OperandStack =} \\ \\
                        \end{tabular}
                        \begin{bytefield}{10}
                            \bitheader{0-9} \\
                            \bitboxes{1}{ {+} {\$} {(} {} {} {} {} {} {} {}}
                        \end{bytefield}
                    }
                
                \item The post-fix string looks like this:
                    {
                        \begin{minted}[autogobble]{c}
                            postfix = "mabc-dk+*+x";
                        \end{minted}
                    }
            \end{itemize}
        \\ 
        \hline
        \hline
        
        %----------------------------------------------------------------------------------------
        \multicolumn{2}{|l|}{
                \begin{minipage}{\linewidth}
                {
                    \begin{center}
                        \begin{bytefield}{27}
                            \\
                            \Huge 
                            \bitheader{0-26} \\ 
                            \bitboxes{1}{{m} {+} {(} {a} {+} {(} {(} {b} {-} {c} {)} {*} {(} {d} {+} {k} {)} {)} {)} {\$} {(} {x} {+} {y} {)} {*} {p} } \\
                            \bitboxes[]{1}{{} {} {} {} {} {} {} {} {} {} {} {} {} {} {} {} {} {} {} {} {} {} {$\uparrow$} {} {} {} {} }
                        \end{bytefield}
                    \end{center}
                }
                \end{minipage}
                } \\
        \hline
        \hline  
        
        %----------------------------------------------------------------------------------------
        
            {
                \begin{minted}[autogobble]{c}
                    infix_string[i=22] = '+';
                \end{minted}
            }
            \begin{itemize}
                \item The \mintinline{c}{'+'} operator has more precedence than \mintinline{c}{'('} so the function returns false. 
                \item Thus we just push the token to the stack.
            \end{itemize}
            {
                \begin{minted}[autogobble]{c}
                    precedence('(','+'); // -> false
                \end{minted}
            }
            &
            \begin{itemize}
                \item The stack looks like this: 
                    {
                        \begin{center}
                            \begin{tabular}{ c }
                                \mintinline{c}{OperandStack =} \\ \\
                            \end{tabular}
                            \begin{bytefield}{10}
                                \bitheader{0-9} \\
                                \bitboxes{1}{ {+} {\$} {(} {+} {} {} {} {} {} {}}
                            \end{bytefield}
                        \end{center}
                    }
                
                \item The post-fix string looks like this:
                    {
                        \begin{minted}[autogobble]{c}
                            postfix = "mabc-dk+*+x";
                        \end{minted}
                    }
            \end{itemize}
        \\ 
        \hline 
        \hline
        
        %----------------------------------------------------------------------------------------
        \multicolumn{2}{|l|}{
                \begin{minipage}{\linewidth}
                {
                    \begin{center}
                        \begin{bytefield}{27}
                            \\
                            \Huge 
                            \bitheader{0-26} \\ 
                            \bitboxes{1}{{m} {+} {(} {a} {+} {(} {(} {b} {-} {c} {)} {*} {(} {d} {+} {k} {)} {)} {)} {\$} {(} {x} {+} {y} {)} {*} {p} } \\
                            \bitboxes[]{1}{{} {} {} {} {} {} {} {} {} {} {} {} {} {} {} {} {} {} {} {} {} {} {} {$\uparrow$} {} {} {} }
                        \end{bytefield}
                    \end{center}
                }
                \end{minipage}
                } \\
        \hline 
        \hline
        %----------------------------------------------------------------------------------------
        
            {
                \begin{minted}[autogobble]{c}
                    infix_string[i=23] = 'y';
                \end{minted}
            }
            \begin{itemize}
                \item This is an operand, we just append it to the post-fix string.
            \end{itemize}
            &
            \begin{itemize}
                \item The stack looks like this: 
                    {
                        \begin{center}
                            \begin{tabular}{ c }
                                \mintinline{c}{OperandStack =} \\ \\
                            \end{tabular}
                            \begin{bytefield}{10}
                                \bitheader{0-9} \\
                                \bitboxes{1}{ {+} {\$} {(} {+} {} {} {} {} {} {}}
                            \end{bytefield}
                        \end{center}
                    }
                
                \item The post-fix string looks like this:
                    {
                        \begin{minted}[autogobble]{c}
                            postfix = "mabc-dk+*+xy";
                        \end{minted}
                    }
            \end{itemize} \\ 
        \hline 
        \hline
        %----------------------------------------------------------------------------------------
        \multicolumn{2}{|l|}{
                \begin{minipage}{\linewidth}
                {
                    \begin{center}
                        \begin{bytefield}{27}
                            \\
                            \Huge 
                            \bitheader{0-26} \\ 
                            \bitboxes{1}{{m} {+} {(} {a} {+} {(} {(} {b} {-} {c} {)} {*} {(} {d} {+} {k} {)} {)} {)} {\$} {(} {x} {+} {y} {)} {*} {p} } \\
                            \bitboxes[]{1}{{} {} {} {} {} {} {} {} {} {} {} {} {} {} {} {} {} {} {} {} {} {} {} {} {$\uparrow$} {} {} }
                        \end{bytefield}
                    \end{center}
                }
                \end{minipage}
                } \\
        \hline 
        \hline
        %----------------------------------------------------------------------------------------
        {
            \begin{minted}[autogobble]{c}
                infix_string[i=24] = ')';
            \end{minted}
        }
        \begin{itemize}
            \item The \mintinline{c}{')'} character has lower precedence than \mintinline{c}{+} thus we pop the stack and append the popped item to the post-fix string.
        \end{itemize}
        {
            \begin{minted}[autogobble]{c}
                precedence('+',')'); // -> true
            \end{minted}
        }
        & 
        \begin{itemize}
            \item The stack looks like this: 
                {
                    \begin{center}
                        \begin{tabular}{ c }
                            \mintinline{c}{OperandStack =} \\ \\
                        \end{tabular}
                        \begin{bytefield}{10}
                            \bitheader{0-9} \\
                            \bitboxes{1}{ {+} {\$} {(} {+} {} {} {} {} {} {}}
                        \end{bytefield}
                        \begin{minted}[autogobble]{c}
                            postfix.append(pop(&OperandStack));
                        \end{minted}
                        \begin{tabular}{ c }
                            \mintinline{c}{OperandStack =} \\ \\
                        \end{tabular}
                        \begin{bytefield}{10}
                            \bitheader{0-9} \\
                            \bitboxes{1}{ {+} {\$} {(} {} {} {} {} {} {} {}}
                        \end{bytefield}
                        \begin{itemize}
                            \item Check precedence again.
                        \end{itemize}
                        \begin{minted}[autogobble]{c}
                            precedence('(',')'); // -> false
                        \end{minted}
                        \begin{itemize}
                            \item This is an exception case, we discard the \mintinline{c}{'('}.
                        \end{itemize}
                        \begin{minted}[autogobble]{c}
                            pop(&OperandStack);
                        \end{minted}
                        \begin{tabular}{ c }
                            \mintinline{c}{OperandStack =} \\ \\
                        \end{tabular}
                        \begin{bytefield}{10}
                            \bitheader{0-9} \\
                            \bitboxes{1}{ {+} {\$} {} {} {} {} {} {} {} {}}
                        \end{bytefield}
                    \end{center}
                }
            
            \item The post-fix string looks like this:
                {
                    \begin{minted}[autogobble]{c}
                        postfix = "mabc-dk+*+xy+";
                    \end{minted}
                }
        \end{itemize}
        \\ 
        \hline
        \hline
        
        %----------------------------------------------------------------------------------------
        \multicolumn{2}{|l|}{
                \begin{minipage}{\linewidth}
                {
                    \begin{center}
                        \begin{bytefield}{27}
                            \\
                            \Huge 
                            \bitheader{0-26} \\ 
                            \bitboxes{1}{{m} {+} {(} {a} {+} {(} {(} {b} {-} {c} {)} {*} {(} {d} {+} {k} {)} {)} {)} {\$} {(} {x} {+} {y} {)} {*} {p} } \\
                            \bitboxes[]{1}{{} {} {} {} {} {} {} {} {} {} {} {} {} {} {} {} {} {} {} {} {} {} {} {} {} {$\uparrow$} {} }
                        \end{bytefield}
                    \end{center}
                }
                \end{minipage}
                } \\
        \hline 
        \hline
        
        %----------------------------------------------------------------------------------------
        {
            \begin{minted}[autogobble]{c}
                infix_string[i=25] = '*';
            \end{minted}
        }
        \begin{itemize}
            \item The \mintinline{c}{'$'} returns true because it has higher precedence than \mintinline{c}{'*'}.
        \end{itemize}
        {
            \begin{minted}[autogobble]{c}
                precedence('$','*'); // -> true
            \end{minted}
        }
        & 
        \begin{itemize}
            \item The stack looks like this: 
                {
                    \begin{center}
                        \begin{itemize}
                            \item We pop the stack until the precedence function returns true. 
                        \end{itemize}
                        \begin{tabular}{ c }
                            \mintinline{c}{OperandStack =} \\ \\
                        \end{tabular}
                        \begin{bytefield}{10}
                            \bitheader{0-9} \\
                            \bitboxes{1}{ {+} {\$} {} {} {} {} {} {} {} {}}
                        \end{bytefield}
                        \begin{minted}[autogobble]{c}
                            postfix.append(pop(&OperandStack));
                        \end{minted}
                        \begin{tabular}{ c }
                            \mintinline{c}{OperandStack =} \\ \\
                        \end{tabular}
                        \begin{bytefield}{10}
                            \bitheader{0-9} \\
                            \bitboxes{1}{ {+} {} {} {} {} {} {} {} {} {}}
                        \end{bytefield}
                        \begin{itemize}
                            \item We check precedence again.
                        \end{itemize}
                        \begin{minted}[autogobble]{c}
                            precedence('+','*'); // -> false
                        \end{minted}
                        \begin{itemize}
                            \item Since this is false, we just push it to the stack. 
                        \end{itemize}
                        \begin{tabular}{ c }
                            \mintinline{c}{OperandStack =} \\ \\
                        \end{tabular}
                        \begin{bytefield}{10}
                            \bitheader{0-9} \\
                            \bitboxes{1}{ {+} {*} {} {} {} {} {} {} {} {}}
                        \end{bytefield}
                    \end{center}
                }
            
            \item The post-fix string looks like this:
                {
                    \begin{minted}[autogobble]{c}
                        postfix = "mabc-dk+*+xy+$";
                    \end{minted}
                }
        \end{itemize}
        \\ 
        \hline
        \hline
        
        %----------------------------------------------------------------------------------------
        \multicolumn{2}{|l|}{
                \begin{minipage}{\linewidth}
                {
                    \begin{center}
                        \begin{bytefield}{27}
                            \\
                            \Huge 
                            \bitheader{0-26} \\ 
                            \bitboxes{1}{{m} {+} {(} {a} {+} {(} {(} {b} {-} {c} {)} {*} {(} {d} {+} {k} {)} {)} {)} {\$} {(} {x} {+} {y} {)} {*} {p} } \\
                            \bitboxes[]{1}{{} {} {} {} {} {} {} {} {} {} {} {} {} {} {} {} {} {} {} {} {} {} {} {} {} {} {$\uparrow$} }
                        \end{bytefield}
                    \end{center}
                }
                \end{minipage}
                } \\
        \hline 
        \hline

        %----------------------------------------------------------------------------------------
        {
            \begin{minted}[autogobble]{c}
                infix_string[i=26] = 'p';
            \end{minted}
        }
        \begin{itemize}
            \item This is an operand, thus we just appendt it to the post-fix string. 
        \end{itemize}
        & 
        \begin{itemize}
            \item The stack looks like this: 
                {
                    \begin{center}
                        \begin{tabular}{ c }
                            \mintinline{c}{OperandStack =} \\ \\
                        \end{tabular}
                        \begin{bytefield}{10}
                            \bitheader{0-9} \\
                            \bitboxes{1}{ {+} {*} {} {} {} {} {} {} {} {}}
                        \end{bytefield}
                    \end{center}
                }
            
            \item The post-fix string looks like this:
                {
                    \begin{minted}[autogobble]{c}
                        postfix = "mabc-dk+*+xy+$p";
                    \end{minted}
                }
        \end{itemize}
        \\ 
        \hline
        \hline
        \begin{itemize}
            \item We have reached the top of the string. Now whatever content is in the stack must be emptyed and appended to the post-fix string.
        \end{itemize}
        & 
        \begin{itemize}
            \item The stack looks like this: 
                {
                    \begin{center}
                        \begin{tabular}{ c }
                            \mintinline{c}{OperandStack =} \\ \\
                        \end{tabular}
                        \begin{bytefield}{10}
                            \bitheader{0-9} \\
                            \bitboxes{1}{ {+} {*} {} {} {} {} {} {} {} {}}
                        \end{bytefield}
                        \begin{minted}[autogobble]{c}
                            postfix.append(pop(&OperandStack));
                        \end{minted}
                        \begin{tabular}{ c }
                            \mintinline{c}{OperandStack =} \\ \\
                        \end{tabular}
                        \begin{bytefield}{10}
                            \bitheader{0-9} \\
                            \bitboxes{1}{ {+} {} {} {} {} {} {} {} {} {}}
                        \end{bytefield}
                        \begin{minted}[autogobble]{c}
                            postfix.append(pop(&OperandStack));
                        \end{minted}
                        \begin{tabular}{ c }
                            \mintinline{c}{OperandStack =} \\ \\
                        \end{tabular}
                        \begin{bytefield}{10}
                            \bitheader{0-9} \\
                            \bitboxes{1}{ {} {} {} {} {} {} {} {} {} {}}
                        \end{bytefield}
                        \begin{itemize}
                            \item The stack is now empty.
                        \end{itemize}
                    \end{center}

                }
            
            \item The post-fix string looks like this:
                {
                    \begin{minted}[autogobble]{c}
                        postfix = "mabc-dk+*+xy+$p*+";
                    \end{minted}
                }
        \end{itemize}
    \end{longtable}
\end{center}



%----------------------------------------------------------------------------------------
\section{Writing the algorithm for converting infix expression to equivalent post-fix}
\begin{itemize}
    \item Arrays:
        \begin{itemize}
            \item infix: to hold the infix string. 
            \item Post-fix: to hold the post-fix string, initially empty.
        \end{itemize}
    
    \item Other data structures:
        \begin{itemize}
            \item OperatorStack: top is initialized and push and pop operation works on this. 
        \end{itemize}
\end{itemize}
Pseudo-code for the infix to post-fix: 
\begin{center}
    \begin{algorithm}[H]
        \SetAlgoLined
        % \DontPrintSemiColon
        % \Input{}
        % \Output{}
        % \BlankLine
        \large
        Initialize\;
        \While{not the end of the infix string}{
            token = get the next element from infix string\; 
            \uIf{token is an operand}{
                append the token with the postfix string\;
            }
            \uElseIf{token is an operator}{
                \While{not empty operator stack and prcd(stacktop,token)}{
                    top\_operator = pop(operatorstack)\; 
                    append top\_operator with the postfix string\;     
                }
                \eIf{token = ')'}{
                    pop(operatorstack)\; 

                }{
                    push(operatorstack,token)\;
                }
            }
        }
        \While{not empty operatorstack}{
            top\_operator = pop(operatorstack)\;
            append top\_operator with the postfix string\; 
        }
        print postfix\;
        \caption{Algorithm for converting infix to post-fix}
    \end{algorithm}
\end{center}


%----------------------------------------------------------------------------------------
\section{Combine the conversion and evaluation function in a single program}
\begin{itemize}
    \item Now we can combine the program that evaluates the post-fix notation and the infix to post-fix notation.
\end{itemize}
\inputcode{c}{\code/calculator.c}


