\section{Overview}
\begin{itemize}
    \item You can't put a char in double quotes, the compiler will interpret anything between double quotes is a string constant. 
    \item Single quotes = chars, Double quotes = strings.
    \item To display a double quote character use \mintinline{c}{\"}.
    \item The string in memory always one character more, this is a null character represented by \mintinline{c}{\0}.
    \item Null character with the code value 0 is added to the end of each string to mark where it ends in memory. \mintinline{c}{\0} character is always included. 
    \item The length of a string is always one greater than the number of characters in the string. 
    \item Don't confuse the null character with the null characters: 
        \begin{itemize}
            \item null character is a string terminator. 
            \item \mintinline{c}{NULL} is a symbol that represents a memory address that doesn't reference anything. 
        \end{itemize}
    
    \item You can add a \mintinline{c}{\0} character to a string explicitly, in this case it will create two strings. If you add \mintinline{c}{\0} to a string in the middle the string will only be counted or recognized up until that null character. 
\end{itemize}
\subsection{Example}
\inputcode{\lang}{\code/null_character.c}


%----------------------------------------------------------------------------------------
\section{Defining a string}
\begin{itemize}
    \item C doesn't have a special type for strings, this means you can't use any of the operators with C for strings, you'll have to use a library.  
    \item Strings in C are stored in an array of type char. 
    \item Character strings are stored in adjacent memory cells. 
    \item Example: \mintinline{c}{char my_str[20];} 19 chars taking in to account the \mintinline{c}{\0} character. 
    \item The compiler will add the null character automatically at the end of the string constant. 
\end{itemize}
Initializing: 
\begin{itemize}
    \item To initialize a character array you put the characters in between the brackets. 
    \item Example: 
        \mint{c}{char word[] = {'H','e','l','l','o','!'};}
    The compiler will automatically add the null character, declaring a char array like a string doesn't require a number of elements in between the square braces, the compiler will add that automatically. 
    
    \item Short hand notation: 
        \mint{c}{char word[] = {"Hello!"}; }
    Let the compiler figure out the size for you. Leave the braces empty. 
    
    \item You can also initialize just part of the array such as: \mintinline{c}{char str[40] = "To be";} 5 will be used, the others are actually empty. 
\end{itemize}
Assigning a value to a string after initializing: 
\begin{itemize}
    \item Since strings are arrays, you can't assign strings either. 
    \item Things such as the following are not allowed: 
        \begin{minted}[autogobble]{c}
            char s[100]; 
            s = "hello"; // error 
        \end{minted}
    
    \item You can't assign an array of characters to another array of characters. To do that use a special function from the string library called strncpy().
\end{itemize}
Displaying a string: 
\begin{itemize}
    \item To display a string with printf use the format specifier \%s: 
    \begin{Verbatim}[breaklines=true, breakanywhere=true]
        printf("\nThe message is: %s",message);
    \end{Verbatim}
    
    \item \%s expects a null terminated string. 
    \item To print a character array you can use the format specifier to format the printf(), using just the name of the character array, this is unique, this is the only type of array that enables this type of printing. 
\end{itemize}
Inputting a string: 
\begin{itemize}
    \item To input string using scanf():
        \begin{minted}[autogobble]{c}
            char input[10]; 
            printf("Please input your name");
            scanf("%s",input); // no ampersand because it's an array. 
        \end{minted}
    
    \item \%s is for inputting and outputing strings. 
    \item Don't use the \& in character arrays for scanf().
\end{itemize}
Testing if two strings are equal: 
\begin{itemize}
    \item You can't do this: \mintinline{c}{if (str1 == str2);}, strings are character arrays so operators can't be applied to them. Don't use == in strings. Remember "x" is two chars because of the null char, 'x' is just one. 
    \item If you wanted to check if two strings are equal you would actually have to use a for loop and compare each character, so there is a standard library function called strcmp(), use this to compare.
\end{itemize}

\subsection{Example}
\inputcode{\lang}{\code/defining_a_str.c}


%----------------------------------------------------------------------------------------
\section{Constant strings} 
\begin{itemize}
    \item the \#define is used to define constants, there are constant strings. 
    \item Constant strings remain hence the name, constant all throughout the program. You may want to use constant strings for readability. You can later change the constant string in one place instead of everywhere you used it. Avoid magic numbers, use constants. 
    \item \#define is a preprocessor statement to define constants. 
    \item Syntax is: \#define \placeholder{var\_name} \placeholder{value}, no semicolon or equals operator. 
    \item All \#define directives are globals, there is no thing as a local \#define.
    \item \#define can be used for char and string constants. 
        \begin{minted}[autogobble]{c}
            #define BEEP 'a'
            #define TEE 'T'
            #define ESC '\033'
            #define OPPS "Now you have done it" 
        \end{minted}
    
    \item Another way to create constant is to use the \mintinline{c}{const} keyword. This means data defined as \mintinline{c}{const} will not be able to change, its another equivalent for \#define. \verb|const| is a read-only value, you can use consts in calculations an whatever as long as you don't alter the value, this is not allowed. const can replace magic numbers. 
    \item Const is more flexible to \#define, it lets you declare a type and allows better control over whch parts of a program can use the constant, is to say that const can be local and global. 
    \item Example: \mintinline{c}{const int MONTHS = 12;}
    \item Third way to declare constants are using enums. 
    \item For initialization of a const string:
        \begin{minted}[autogobble]{c}
            const char message[] = "The end of the world is nigh.";
        \end{minted}
    
    \item Any attempt to modify it by doing a strcopy() will cause an error. 
\end{itemize}


%----------------------------------------------------------------------------------------
\section{String functions}
\begin{itemize}
    \item C standard library includes functions that can operate on strings. 
\end{itemize}
C standard functions on strings \mintinline{c}{#include <string.h>}: 
\begin{itemize}
    \item strlen(): getting length of string. 
        \begin{itemize}
            \item This functions does change the string, string is not parametrized as a constant. 
        \end{itemize}
    \inputcode{\lang}{\code/strlen.c}
    \item strcpy() and strncpy(): copying one character string to another. 
        \begin{itemize}
            \item \mintinline{c}{char s[100]; s = "hello";} doesn't work in C, you'll have to use strcpy()
            \item The strcpy() doesn't check the length of the character array, thus it's prone to errors such as out of bounds errors.
            \item The strncpy() takes in an third argument, the maximum number of characters to copy. 
        \end{itemize}
    \inputcode{\lang}{\code/strcpy.c}
    \item strcat() and strncat(): used for concatenation.
        \begin{itemize}
            \item strncat() takes consideration of the size of the string, strcat() doesn't. 
        \end{itemize}
    \inputcode{\lang}{\code/strcat_strncat.c}
    \item strcmp(): compare strings to see if they are equal. 
        \begin{itemize}
            \item str1 == str2 will compare whether the strings have the same address, not the actual content.  
            \item in strcmp(), if the result is 0 means that they are the same string, if the result is negative or positive they are not equal. 
            \item If the return is negative $\rightarrow$ str1 is less than str2. 
            \item If return is positive $\rightarrow$ str2 is less than str1.
            \item You don't have to worry about the size of arrays being compared thus no strncmp() not for ensuring no buffer overflows, its for comparing substrings. For example, you want all strings that start in ``astro'' you'll use strncmp(), use strncmp() when you are interested in prefixes.
        \end{itemize}
    \inputcode{\lang}{\code/strcmp_equal.c}
\end{itemize}
\subsection{String functions examples}
\inputcode{\lang}{\code/strlen_strncpy.c}


%----------------------------------------------------------------------------------------
\section{Searching, Tokenizing, and Analyzing Strings}
\begin{itemize}
    \item Searching a string: To find a single character or substring use: strchr() and strstr().
    \item Tokenizing: Sequence of characters within a string bounded by a delimeter (space,comma,period,etc); Breaking a sentence into words is called tokenizing. Use: strtok() 
    \item Analyzing strings: islower(), isupper(), isalpha(), etc. 
\end{itemize}

\subsection{Pointers}
\begin{itemize}
    \item A pointer is a useful type of variable, related to memory. A pointer is a variable that stores an address, its value is the address of another location in memory that can contain value. 
        \begin{minted}[autogobble]{c}
            int Number = 25; 
            int *pNumber = &Number; 
        \end{minted}
    
    \item *pNumber is a pointer, it points to address of Number. 
\end{itemize}
\begin{center}
    \begin{tabular}{ |c|c|c| }
        \hline
            Variable & Address & Value \\
        \hline \hline
            Number & 1000 & 25 \\ 
        \hline
            pNumber & 1004 & 1000 \\ 
        \hline
    \end{tabular}
\end{center}

\subsection{Searching a string for a character}
\begin{itemize}
    \item strchr() function searches a given string for a specified character. The first argument is the string to search (given as an address) and the second is the character to look for. 
    \item The function will return a pointer to the first position in the string where the character is found. The address to this position in memory and it is of type char* or ``pointer to char''.
    \item To store a value returned you must store it in a variable that can store the address of a character. 
    \item If the character is not found, the function returns a null value. For a pointer null represents a pointer that is not pointing to anything. 
\end{itemize}
\begin{minted}[autogobble]{c}
    char str[] = "The quick brown fox"; 
    char ch = 'q';
    char *pGot_char = NULL; 
    pGot_Char = strchr(str,ch); // pointer will be pointed to the q address. 
\end{minted}
\begin{itemize}
    \item In the example you can also use an int as a char to look for, it will be converted to ASCII equivalent. 
\end{itemize}

\subsection{Searching for a substring}
\begin{itemize}
    \item strstr() searches one string for the first occurrence of a substring. It returns a null pointer if non is found. 
    \item Returns a pointer to the position in the first string where the substring is found. 
    \item The search is case-sensitive. 
\end{itemize}
\begin{minted}[autogobble]{c}
    char text[] = "Every dog has his day";
    char word[] = "dog"; 
    char *pFound = NULL; 
    pFound = strstr(text,word);
\end{minted}

\subsection{Tokenizing a string}
\begin{itemize}
    \item Token is a sequence of characters within a string that is bound by a delimiter, the delimiter can be anything, best if the delimiter is something meaningful.
    \item strtok(): first argument is the string to be tokenized, a string containing all the possible delimiter characters. 
\end{itemize}
\inputcode{\lang}{\code/strtok.c}

\subsection{Analyzing strings}
\begin{itemize}
    \item \mintinline{c}{islower()}: lower case.
    \item \mintinline{c}{isupper()}: upper case.
    \item \mintinline{c}{isalpha()}: lower or upper case.
    \item \mintinline{c}{isalnum()}: upper case or lower case or a digit. 
    \item \mintinline{c}{iscntrl()}: control character. 
    \item \mintinline{c}{isprint()}: any printing character including a space. 
    \item \mintinline{c}{isgraph()}: any printing character except a space.
    \item \mintinline{c}{isdigit()}: decimal digit (0-9)
    \item \mintinline{c}{isxdigit()}: hexadecimal digit (0-9,A-F,a-f)
    \item \mintinline{c}{isblank()}: standard blank characters, space and \verb|\t|
    \item \mintinline{c}{isspace()}: white space character such as: \verb|space, \n, \t, \v, \r, \f|
    \item \mintinline{c}{ispunct()}: printing character for shitch isspace is isalnum return false. 
\end{itemize}
\inputcode{\lang}{\code/analyzing_str.c}


%----------------------------------------------------------------------------------------
\section{Converting strings}
\begin{itemize}
    \item Use toupper() and tolower(). 
    \item Syntax is: \mintinline{c}{for (int i = 0; (buff[i]=(char)toupper(buff[i])) != '\0'; ++i);}
    \item This loop will cover the entire string in the buff array to uppercase by stepping through the strung one character at a time. 
    \item toupper() returns a type int, thats why we convert it. 
\end{itemize}
\inputcode{\lang}{\code/str_toupper.c}
This is all included in the stdio.h.
\begin{center}
    \begin{tabular}{ |c|p{7cm}| }
        \hline
            Function & Returns \\
        \hline
            \mintinline{c}{atof()} & A value of type double that is produced from  the string argument. Infinity as a double value is recognized from the strings ``inf'' or ``INFINITY'' where any character can be in uppercase or lowercase and ``not a number'' is recognized from the string ``NAN'' in uppercase or lowercase. \\  
            \mintinline{c}{atoi()} & A value of type int that is produced from the string argument. \\  
            \mintinline{c}{atol()} & A value of type long that is produced from the string argument. \\  
            \mintinline{c}{atoll()} & A value of type long long that is produced from the string argument.  \\ 
            \mintinline{c}{strtod()} & A value of type double is produced from the initial part of the string specified by the first argument. The second argument is a pointer to a variable, ptr say, of type char* in which the fucntion will store the address of the first character following the substring that was converted to the double value. If no string was found that could be converted to type double, the variable ptr will contain the address passed as the first argument. \\ 
            \mintinline{c}{strof()} & A value of type float. In all other respects it works as strtod(). \\ 
            \mintinline{c}{strtold()} & A value of type long double. In all other respects it works as strtod(). \\ 
        \hline
    \end{tabular}
\end{center}
\inputcode{\lang}{\code/convertion_function_example.c}

%----------------------------------------------------------------------------------------

\section{Challenge - Understanding char arrays}
\inputcode{\lang}{\code/challenge_write_strlen_compare_concat.c}

\subsection{Reverse a string}
\inputcode{\lang}{\code/challenge_reverse_a_str.c}


