\section{Hello world project}
\begin{itemize}
    \item Java is case sensative.
    \item The class is structured like so:
        \fvset{frame=single,numbers=left,numbersep=3pt}
        \VerbatimInput{Clases/Examples/src/Hello.java}
    
    \item The keyword ``public'' is an acces modifier, this defines the scope.
    \item The class name will be the one following the keyword.
    \item The curly braces define the class body, also called block.
    \item A method: is a collection of statements that perform an operation. The main method is the entry point of the program.
    \item Void: is a method that the method that will not return anything.
    \item The code block is defined with curly braces, and contains statements corresponding to certain parts of the code.
    \item Statement: is a complete command to be excecuted and can include one or more expressions.
\end{itemize}

%%%%%%%%%%%%%%%%%%%%%%%%%%%%%%%%%%%%%%%%%%%%%%%%%%%%%%%%%%%%%%%%%%%%%%%%%%%%%%%%%%%%%%%%%%%%%%%%
\section{Variables}
\begin{itemize}
    \item A variables is a way to store information in a computer, they can ba accesed by the name reference and the computer does the work of allocating the memory to store that information, this happens in the RAM.
    \item The contents stored can be changed.
    \item Aspects: we must declare the data type for each variable, use the data type keywords, then initialize them.
    \item Follow:
        \fvset{frame=single,numbers=left,numbersep=3pt}
        \VerbatimInput{Clases/Examples/src/Variables_01.java}
    
    \item In principle a variable needs to be initializad before it's used.
\end{itemize}

%%%%%%%%%%%%%%%%%%%%%%%%%%%%%%%%%%%%%%%%%%%%%%%%%%%%%%%%%%%%%%%%%%%%%%%%%%%%%%%%%%%%%%%%%%%%%%%%
\section{Primitive data types}
\begin{itemize}
    \item Primitive data types are the most basic.
    \item There are eight primitive data types:
        \begin{enumerate}
            \item Boolean 
            \item Byte 
            \item Char 
            \item Short 
            \item Int 
            \item Long 
            \item Float 
            \item Double 
        \end{enumerate}
\end{itemize}

\subsection{Int}
\begin{itemize}
    \item The int data type are for whole numbers, they are not infinite, there is a maximum and a lower value. Follow:
        \fvset{frame=single,numbers=left,numbersep=3pt}
        \VerbatimInput{Clases/Examples/src/ByteShortIntLong.java}
    
    \item ``Integer'' is a wrapper class; A wrapper class is a concept that is applied to all primitive types and allow us to perform operations in those data types.
    \item Intigers can be separated with underscores.
\end{itemize}




%%%%%%%%%%%%%%%%%%%%%%%%%%%%%%%%%%%%%%%%%%%%%%%%%%%%%%%%%%%%%%%%%%%%%%%%%%%%%%%%%%%%%%%%%%%%%%%%%%%
\section{Byte \& Short}
\begin{itemize}
    \item For the byte the data type can store much less values than the int, it is still a number based data type, the interval value is:  [-128,127].
    \item For the Short is the same as the byte but stores even less memory.
\end{itemize} 
