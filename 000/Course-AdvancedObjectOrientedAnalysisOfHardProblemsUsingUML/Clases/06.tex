\section{What is a model?}
\begin{itemize}
    \item UML, Unified Modeling Language.
    \item Why do we model? 
    \begin{itemize} 
            \item It's important to diferentiate the subject matter and the model.  
            \item To make complex subject matters more comprehensible and possible to grasp on a higher level.
            \item We create a model to understand, we use it to simulate.
            \item To be able to simulate real-world scenarios to get an approximation on how the subject may react and work.
            \item We create models to understand complex things and once we have the model to run simulations.
            \item Specify, not yet build tangible subjects, we use a model to specify a subject.
            \item Specify, not yet developed intengible subjects. Like bussinesses or business models.
        \end{itemize}
    
    \item Models:
        \begin{enumerate}
            \item To understand complex subject matters.
            \item To do simulations and try to predict the future.
            \item To specify not yet built physical things and intangible things.
        \end{enumerate}
\end{itemize}


%----------------------------------------------------------------------------------------
\section{Domain models}
\begin{itemize}
    \item What is a domain model:
        \begin{itemize}
            \item Domain: encloses the content of a particular field of knowledge. Example: the domain of the Udemy example is the things that encapsulates the whole area, it's the knowledge that limits that field.
            \item This is recursive, a system is a part of a bigger system, the scope can be played with, it's also layered, I can describe something using biology or chemistry or physics or quantum physics. You can broaden the scope and tweek the abstraction.
            \item A conceptual domain model is: can be described on the conceptual level, it includes the concepts used by the experts in that particular domain, used to create, reason understand adn develop theories and systems within that domain.
        \end{itemize}
    
    \item Depending on who you are there will be diferent concepts in the field.
    \item Problem and solution domain: 
        \begin{itemize}
            \item The conceptual model is explicitly chosen to be independent of design or solution concern. 
            \item This problem-solution domain is relative, it depends on who you ask. 
        \end{itemize}
    
    \item What is a business domain model:
        \begin{itemize}
            \item It includes concepts the business experts use. 
            \item It's usually independent on any organization or software system.
            \item It is detailed to a level that makes sense for the business experts. 
            \item Sufficient to describe the business logic. 
            \item Not only information entities.
            \item Business concept models do not just include ``what we have in our databases''
        \end{itemize}
\end{itemize}



%----------------------------------------------------------------------------------------
\section{What is a model more formally?}
\begin{itemize}
    \item Definition of model in UML:
        \begin{itemize}
            \item Three categories of elements: Classifiers,  events, behaviours.
            \item Each major category models individuals in an incarnation of the system being modeled.
            \item An incarnation of the system being modeled.
            \item An incarnation is one instance of the concept. 
            \item Example: a McDonald's restaurant is an instance of the McDonald's concept.
        \end{itemize}
    
    \item Classifiers: A classifier describes a set of objects, an object is an individual thing with a state and relationships to other objects. Example: car.
    \item Events: describes a set of posible occurrences. An occurrance is something that happens that has some consecuences within the system. Example: accident.
    \item Behavior: describes of posible executions. An excecution is the performance of an algorithm according to set of rules. Example: A $\rightarrow$ B, A $\rightarrow$ C, A behaviour is described as A can go to B or C.
    \item Models do not contain objects, occurrences, and executions, because those things are the subjects of models, not their content. 
    \item You have subjects and you have models, A model can model aspects of individual objects, events or behavior. Example: occurrence of an accident, we model the event, type of accident, time of accident, occurrance specification; the occurrance will be complete the model will be incomplete. 
    \item Value specifications, occurrences specifications, and excecution specifications model individual objects, occurrences, and executions within a particular context. 
    \item Is a diagram a model:  
        \begin{itemize}
            \item No, they are not. 
            \item The model is the ``store'' of the complete, often recursively composed, set of elements that describes a system.
            \item A diagram/map is a view of specific aspects of the model's elements.
        \end{itemize}
    
    \item Bare in mind ``we often use the term model as a shorthand for diagram, maps or subset of a model'', we must understand the distinction.
    \item One diagram showing one aspect of the some of the elements. A model is thus composed of diagrams but the model isn't a diagram.
\end{itemize}


%----------------------------------------------------------------------------------------
\section{The unified modeling language}
\begin{itemize}
    \item A standard general-purpose modeling language in the field of software and business engineering. Managed by, the object management group.
    \item Includes a set of graphical notation techniques to create visual models of systems.
    \item Brief history:
        \begin{itemize}
            \item Formed 1996, UML 1.0, released by the three amigos. They merged their methods for object oriented modeling.
            \item Version 1.3,1.4 fixed bugs. 
            \item 2005 major revision was adopted.
            \item 2012 is version 2.5, newest.
        \end{itemize}
    
    \item UML is big, it's a tool set used to specify a complete systems. A lot of diagram types. 
    \item The basics needed in this course: 
        \begin{itemize}
            \item Class: has attributes, they can have associations to others, they can have generalisations, they can have relations in the diferent types of clases.
            \item A class is an instance, we can have links between instances. 
            \item This is linked to the phylosofical course content. 
            \item We can have domains, can have dependences. 
            \item There is hundreds of elements in UML.
        \end{itemize}
\end{itemize}


%----------------------------------------------------------------------------------------
\section{Classes, attributes and objects}
\begin{itemize}
    \item Classes:
        \begin{itemize}
            \item Classes are rendered as boxes with a name indicating the ``term'' for the concept.
            \item Concepts are modeled as classes in UML.
            \item Classes have compartments that hold the fetures (such as attributes) of the class.
            \item Compartments can be shown or hidden in a specific diagram. 
        \end{itemize}
    
    \item Attributes:
        \begin{itemize}
            \item They model the non relational properties become attributes on the class in UML. 
            \item Examples: a group of people that have the concept car, color pink; 
            \item The attribute's instantiations describe the objects of the class.
            \item Attributes could be set to a specific type, for example in an agreement wi'll have the attribute date, valid date, identifier.
            \item Get the concepts straight first. We notice attributes in the breakdown fase.
        \end{itemize}
    
    \item Objects:
        \begin{itemize}
            \item A classifier describes a set of instances that have features in common, classfier is the generic term for a class.
            \item The special type of instances that are instances of classes are called objects. 
            \item The objects in the wordl are represented in UML as instances of a class. Instance specifications, are like constructors in java.
            \item A box with an underlined name is used as notation for an object, the class name is stated after the colon. 
            \item Objects can be anonymous. The name could be left blank. 
            \item Objects can be unclassified, you can set the class later in the process, you can have names and no object instantiation, you then add the class instantiation as you start to understan what you need. 
            \item Attribute's instantiations describe the objects of the class. 
        \end{itemize}
    
    \item Stereotypes:
        \begin{itemize}
            \item What if I want to classify my classes? 
            \item You can add a stereotype to enrich the predefined semantics. 
            \item A common classification in business modelling is a stereotype business entities, business workers and business actors. 
            \item You can create ``icons'' for the stereotypes. You can have the default like this: \verb|<<something>>|
            \item Two equivalent UML diagrams of the same model.
        \end{itemize}
    
    \item Summary:
        \begin{itemize}
            \item Classes are boxes with a name indicating the ``term'' for the concept, elements can be stereotyped. The attributes, which could have a type, are shown in the class compartements.
            \item Objects are rendered as a box, with underlined name and optionally the instantiating class. The attributes instances that describe the object could optionally be shown in the objects compartment.
        \end{itemize}
\end{itemize}


%----------------------------------------------------------------------------------------
\section{UML Associations - ``The lines''}
\begin{itemize}
    \item Assosiations are the most common relationship used in class modeling, associations are shown as a solid line between two classes.
    \item Associations (ends) could have multiplicity.
        \begin{itemize}
            \item It regulates which type of object constellations that are posible.
            \item ``An agreement must have at least one party but a party doesn't have to be involved in an agreement.''
            \item The 1..* declares the dependencies, one depedency in this case. 
        \end{itemize}
    
    \item Associations can have role names. The association end has a role, in UML Language.
    \item Association end could have navigability, this is represented as an arrow.
        \begin{itemize}
            \item Navigability specifies if the class ``knows'' about the relationship.
            \item If no arrows are shown, both ends are navigable.
        \end{itemize}
    
    \item Very common confution about directions:
        \begin{itemize}
            \item Static: meaning, the agreement concept ``depends'' on the existance of the product concept, the direction of the arrow implies the dependence. 
            \item Dynamic meaning, the process of defining products to sell is prior to the writing of agrements. In concept modeling mixing behavior and dependencies can be confusing.
        \end{itemize}
    
    \item Reading directions are like comments in the diagram flow.
    \item Links:
        \begin{itemize}
            \item Links specify specific properties become links between objects in UML. 
            \item Links are shown as a solid line between two objects. 
            \item Exampl: Romeo:Person $\rightarrow$ Loves $\rightarrow$ Juliet:Person (agent) vice versa.
            \item Links are instances of associations. 
        \end{itemize}
    
    \item An associations declares that there can be links between the objects.  So associations are also classifiers, as classes... An association can be instanciated thus they are classifiers as well.
    \item What if I need to describe attributes for the associations: 
        \begin{itemize}
            \item Associations classes are shown as a regular class which has a dotted line that conncets them to the associations that they describe. 
            \item It's like a comment for attribute links.
        \end{itemize}
    
    \item Modeling compositions of objects:
        \begin{itemize}
            \item Typically, a part instance is only ``attached'' to one whole instance at a time.
            \item The composite aggregate are shown as a solid black ``diamond'' at the associations end for the ``whole'' class. 
            \item Objects that are composed of other objects are called aggregations (phylosophy mereology).
            \item Example: a table is aggregating the top and the table legs.
            \item Composite classfiers provides mechanisms for specifying structures of interconnected classifiers created within an instance of a containing classifier. Instead of showing the table as three diferent classes you can have an anidated box composed of the objects that are composed to aggregate the master object. 
        \end{itemize}
    
    \item Summary:
        \begin{itemize}
            \item Associations are shown as a solid line between two classes, a name and ``reading'' direction can be shown.
            \item Association ends could have multiplicity, role names, and navigability. 
            \item Composite aggregations are shown as a black diamons on a side ``owning'' class.
            \item Links between objects which are instances of associatons are \textbf{also} shown as solid lines between objects.
            \item How we model depends on what is considered important in the domain.
        \end{itemize}
\end{itemize}


%----------------------------------------------------------------------------------------
\section{Generalizations}
\begin{itemize}
    \item Generalizations:
        \begin{itemize}
            \item Are another type of relation, generalizations are the way to express general implications. Remember:
                \[
                  \forall x  (\text{ Man }(x)) \rightarrow x = \text{ Animal }
                \]
            
            \item General implications become just generalizations in UML.
            \item The generalization relationship is solid line with a non solid triangle in the parent side.
            \item This is the way we relate concepts when the links are generalizations.
        \end{itemize}
    
    \item Associations and links:
        \begin{itemize}
            \item Example: Cows and barns, mammal is a generalization, a generalization of mammal is animal, so on. 
            \item Hence by definition there are only objcets of the lowest level clases.
            \item All superordinate classes in a generalization structure are abstract, this means you can't instantiate them directly. 
            \item The name of an abstract class is put in italics, this means that this class can't be instantiated directly. 
        \end{itemize}
    
    \item Two orthogonal ways to subtype accounts: 
        \begin{itemize}
            \item A class can have multiple generalization sets. 
            \item Each set can also be further described with constraints within curly brackets \{\} and the set could be given a name, such as Liability Type and Account Type.
        \end{itemize}
    
    \item What if you need to speak about the type before you have an object to speak about:
        \begin{itemize}
            \item This is called powertypes.
            \item Powertypes are shown as a class with the same name as the generalization set, stereotyped to ``powertype''.
            \item A powertype is a class whose instances are also subclasses of another class. 
            \item Example:
                \begin{itemize}
                    \item Maple: Tree Species 
                    \item a specific tree: Maple 
                \end{itemize}
        \end{itemize}
    
    \item Multiple inheritance:
        \begin{itemize}
            \item A class might specialize more than one class.
            \item Be careful, it makes the model harder to interpret.
            \item Example: 
                \begin{itemize}
                    \item Vehicle $\rightarrow$ car $\rightarrow$ boat $\rightarrow$ amphibious vehicle.
                \end{itemize}
            
            \item One class inherits form another. 
        \end{itemize}
    
    \item Summary:
        \begin{itemize}
            \item The generalization relationship is shown as a solid line with a non-filled arrow head.
            \item The roles is the relationship is called generalizaion (the parent) and specialization (the child).
            \item A class can have multiple generalization sets which could have a constraint (such as incomplete, disjoint) and the set can have a name.
            \item If needed a specific class, called a powertype can be modeled for each generalization set. 
        \end{itemize}
\end{itemize}


%----------------------------------------------------------------------------------------
\section{Packages - Divide and conquer}
\begin{itemize}
    \item Packages are used to decompose large UML models.
    \item Packaging are used to deal with complexity and modificability of model by decomposing it in to smaller chunks. 
    \item Packages depend on each other, dependencies are shown as dotted lines with an open arrow in UML. 
    \item The direction of the dependency is derived from the ingoing elements' relationships. 
    \item The packages are dependent.
    \item Sound packing is so important that the whole last section of this cours talking about this. 
\end{itemize}

