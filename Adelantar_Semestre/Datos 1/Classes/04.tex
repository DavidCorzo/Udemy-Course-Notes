\section{Required software}
\begin{itemize}
    \item MySQL: is a relational database management system.
        \begin{itemize}
            \item We can set a mySQL database server, which is a server were when it is running you can search for things in the database and perform the usual CRUD operations.
        \end{itemize}

    \item popsql: is an IDE, we will use this to have ease of use with the mySQL server.
\end{itemize}

\section{Data types}
\begin{itemize}
    \item There are lots of data types you can use on mySQL.
    \item The most common are:
        \begin{center}
            \begin{tabular}{ |c|c| }
                \hline
                    \mintinline{sql}{INT} & Whole number. \\
                    \mintinline{sql}{DECIMAL(M,N)} & Decimal Numbers - Exact value. The (whole number part, how many decimals it will have). \\
                    \mintinline{sql}{VARCHAR(l)} & String of text of length 1. Inside the parenthesis you put the length of the string you want, for example (100) means 100 characters. \\
                    \mintinline{sql}{BLOB} & Binary Large Object, Stores large data, such as images or files. \\
                    \mintinline{sql}{DATE} & 'YYYY-MM-DD' \\
                    \mintinline{sql}{TIMESTAMP} & 'YYYY-MM-DD HH:MM:SS' Used for recording when something happened. \\
                \hline
            \end{tabular}
        \end{center}
\end{itemize}

\section{Creating a table}
\begin{itemize}
    \item To create a table we must instruct the program which data types the database will store.
    \item Writing things in all caps is optional but as convention it is preferred because it makes it easy to distinguish SQL from other elements.
    \item Every command in SQL needs to be terminated with a semicolon.
    \item Create a table:
        \begin{minted}[autogobble]{sql}
            CREATE TABLE student (
                -- columns:
                student_id INT PRIMARY KEY,
                name VARCHAR(20),
                major VARCHAR(20)
            );
        \end{minted}
    
    \item You can define the primary key after table creation:
        \begin{minted}[autogobble]{sql}
            CREATE TABLE student (
                -- columns:
                student_id INT,
                name VARCHAR(20),
                major VARCHAR(20)
                PRIMARY KEY(student_id)
            );
        \end{minted}
    
    \item You can use the DESCRIBE to list the table elements:
        \begin{minted}[autogobble]{sql}
            DESCRIBE student;
        \end{minted}
    
    \item To delete the table you can use DROP TABLE:
        \begin{minted}[autogobble]{sql}
            DROP TABLE student;
        \end{minted}
    
    \item To add a column:
        \begin{minted}[autogobble]{sql}
            ALTER TABLE student ADD gpa DECIMAL(3,2); -- store student's GPA
        \end{minted}
    
    \item To delete a column:
        \begin{minted}[autogobble]{sql}
            ALTER TABLE student DROP COLUMN gpa;
        \end{minted}
\end{itemize}
