\section{Object-oriented programming}
\begin{itemize}
    \item Methods: act as functions that use information about the object, as well as the object itself to return results, or change the current object. 
\end{itemize}


%----------------------------------------------------------------------------------------
\section{Attributes and class keyword}
\begin{itemize}
    \item By convention class names are written in upper camel case convention. 
\end{itemize}
\begin{minted}[autogobble]{python}
    class Sample():
        pass
    my_sample = Sample()
    type(my_sample)
    # output: __main__Sample

    class Dog():
        # This is the constructor method.
        def __init__(self,breed,name,spots): 
            # By convention the attribute name and the parameter name are the same.
            # Attributes 
            # We take in the argument 
            # Assign it using self.attribute_name
            self.breed = breed
            self.name = name
            self.spots = spots

    my_dog = Dog(breed="Lab",name="Sammy",spots=False)
    type(my_dog)
    # output: __main__.Dog
    my_dog.breed
    # output: 'Lab'
    my_dog.name
    # output: 'Sammy'
    my_dog.spots
    # output: False
\end{minted}

%----------------------------------------------------------------------------------------
\section{OOP Class object attributes}
\begin{itemize}
    \item Class object attributes are defined before the constructor.
    \item They are the same for any instance of the class. 
    \item Class object attributes can be referenced in two ways: 
        \begin{enumerate}
            \item \placeholder{Class name}.\placeholder{class object attribute} 
            \item self.\placeholder{class object attribute}     
        \end{enumerate}
\end{itemize}
\begin{minted}[autogobble]{python}
    class Circle():
        # CLASS OBJECT ATTRIBUTE 
        pi = 3.14

        def __init__(self, radious=1):
            self.radious = radious
            self.area = (radious ** 2) * Circle.pi

        def get_circumference(self):
            return self.radious * self.pi * 2

    my_circle = Circle(30)
    my_circle.pi
    # output: 3.14
    my_circle.radious
    # output: 30
    my_circle.get_circumference()
    # output: 188.4
    my_circle.area
    # output: 2826.0
\end{minted}


%----------------------------------------------------------------------------------------
\section{OOP Inheritance and polymorphism}
\begin{itemize}
    \item Inheritance: form new classes using classes that have already been defined, it's goal is to reuse code and to reduce complexity. 
        \begin{itemize}
            \item All methods from the base class can be overwritten and used in the inheritance class. 
        \end{itemize}
        \begin{minted}[autogobble]{python}
            # BASE CLASS
            class Animal():
                def __init(self):
                    print("Animal created")
                
                def who_im_i(self):
                    print("I'm an animal")
                
                def eat(self):
                    print("I'm eating")

            # INHERITANCE OF DOG FROM THE ANIMAL CLASS
            class Dog(Animal):
                def __init__(self): 
                    Animal.__init__(self) # Copy Animal class constructor
                    print("Dog created")
                def who_im_i(self):
                    print("I'm a dog!")
                def bark(self):
                    print("I'm a dog eating")

            mydog = Dog()
            # output: Animal created
            # output: Dog created
            mydog.who_im_i()
            # output: I'm a dog!
            mydog.bark()
            # output: I'm a dog eating
        \end{minted}
    
    \item Polymorphism: diferent object classes with identical method names. 
        \begin{minted}[autogobble]{python}
            class Dog():
                def __init__(self,name):
                    self.name = name 
                def speak(self):
                    print(self.name + " says woof!")

            class Cat():
                def __init__(self,name):
                    self.name = name 
                def speak(self):
                    print(self.name + " says meow!")

            niko = Dog("niko")
            felix = Cat("felix")
            print(niko.speak())
            # output: niko says woof!
            print(felix.speak())
            # output: felix says meow!

            for pet in [niko,felix]:
                print(type(pet))
                print(pet.speak())
            # output: <class '__main__.Dog'>
            # output: niko says woof!
            # output: <class '__main__.Dog'>
            # output: felix says meow!
        \end{minted}
\end{itemize}
\begin{itemize}
    \item An abstract class is a class which is not intended to be instanciated, it's only used as a base class. 
        \begin{minted}[autogobble]{python}
            # ABSTRACT CLASSES
            class Animal():
                def __init__(self,name):
                    self.name = name 
                def speak(self):
                    raise NotImplementedError("Subclass must implement this abstract method.")
            class Dog(Animal):
                def __init__(self,name):
                    self.name = name 
                def speak(self):
                    print(self.name + " says woof!")

            class Cat(Animal):
                def __init__(self,name):
                    self.name = name 
                def speak(self):
                    print(self.name + " says meow!")

            fido = Dog("Fido")
            isis = Cat("Isis")
            print(fido.speak())
            print(isis.speak())
        \end{minted}
\end{itemize}


%----------------------------------------------------------------------------------------
\section{Special (Magic/Dunder) methods}
\begin{itemize}
    \item Make python built-in methods interact with the user defined objects and classes. 
\end{itemize}
\begin{minted}[autogobble]{python}
    class Book():
        def __init__(self,title,author,pages):
            self.title = title
            self.author = author 
            self.pages = pages 
        
    b = Book("Python","David",200)
    print(b)
    # output: <__main__.Book object at 0x000001B7E92102B0>
    # I want to be able to print this in a string form so...

    class Book():
        def __init__(self,title,author,pages):
            self.title = title
            self.author = author 
            self.pages = pages 
        
        def __str__(self):
            return f"{self.title} by {self.author}"
        
        def __len__(self):
            return self.pages

        def __del__(self):
            print(f"The book object '{self.__str__()}'' has been deleted")

    b = Book("Python","David",200)
    print(b)
    # output: Python by David
    print(len(b))
    # output: 200
    del b 
    # output: The book object 'Python by David' has been deleted
\end{minted}
